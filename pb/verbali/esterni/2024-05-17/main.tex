
# requisiti opzionali

incontro per discutere i requisiti opzionali da implementare e quali no

I requisiti opzioni che è stato deciso di implementare saranno i seguenti:

...

I requisiti opzionali che non saranno implementati al fine dell'MVP sono i seguenti:

...

Il propontente si è trovato d'accordo con queste decisioni.


# aiuti generali

\begin{itemize}
\item \textbf{Domanda}: Nei Dockerfile bisogna installare npm e composer per esempio manualmente tramite comando o non serve ? 
\item \textbf{Risposta}: È norma per ogni servizio fornire un immagine docker già preparata ad hoc, utilizzando ad esempio FROM all'interno del dockerfile, ed è necessario solo importare la propria libreria nel container. \\ Ad esempio per node viene copiato il package.json e il proprio codice ed eseguire `npm build`, il container sarà già fornito di tutti gli strumenti necessari completare la costruzione con successo. \\ \\ Sono stati poi menzionati i bundle (Es. MySQL più Apache) e i dockerfile multistage. \\ Per questi utlimi il sig. Staffolani ci ha spiegato a voce il seguente \href{https://medium.com/@mohamedbenkhemiswork576/how-to-dockerize-a-react-app-with-multi-stage-build-and-nginx-minimize-react-image-size-by-80-33a09ae20118}{tutoriarl}.
\end{itemize}

\begin{itemize}
\item \textbf{Domanda}: È meglio avere un indicatore di code coverage unico per backend e frontend o averne due separati ?
\item \textbf{Risposta}: Ci è stato consigliato di utilizzare due code coverage separate per backend e frontend, quanto contano come due applicazione separate e in un contesto reale sarebbero sviluppate da team diversi
\end{itemize}
