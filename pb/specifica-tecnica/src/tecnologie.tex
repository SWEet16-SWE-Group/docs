\section{Tecnologie}

In questa sezione viene fornita una panoramica generale delle tecnologie utilizzate per la
realizzazione del prodotto in questione. Vengono infatti descritte le procedure, gli strumenti e le
librerie necessari per lo sviluppo, il test e la distribuzione del prodotto.\\
In particolare, verranno trattate le tecnologie impiegate per la realizzazione del front-end e del back-end, per la gestione del
database e per l'integrazione con i servizi previsti.

\subsection{Linguaggi}

\subsubsection{}

\textbf{Descrizione:}

\textbf{Versione:}



\subsubsection{HTML}

\textbf{Descrizione:} Linguaggio di annotazione (markup) utilizzato per impostare la struttura delle
singole pagine e definire gli elementi dell’interfaccia;

\textbf{Versione:} 5


\subsubsection{CSS}

\textbf{Descrizione:} Linguaggio utilizzato per la formattazione e la gestione dello stile degli elementi HTML;

\textbf{Versione:} 3


\subsubsection{JavaScript}

\textbf{Descrizione:} Linguaggio utilizzato per la gestione di eventi incocati dall'utente;

\textbf{Versione:} ECMAScript 2023;

\subsubsection{PHP}

\textbf{Descrizione:} Linguaggio per la codifica di applicazioni web lato server, utilizzato per la creazione di \emph{API Rest}$^{G}$;

\textbf{Versione:} 8.x

% fine linguaggi

\subsection{Librerie e framework}

\subsubsection{ReactJs}

\textbf{Descrizione:} Libreria grafica per facilitare lo
sviluppo front-end gestendo modularmente le componenti grafiche,
permettendo performance buone grazie all'efficacia della sua renderizzazione;

\textbf{Versione:} 18.2.x

\subsubsection{Laravel}

\textbf{Descrizione:} Framework PHP utilizzato per facilitare la creazione di API Rest;

\textbf{Versione:} 11

\subsubsection{Axios}

\textbf{Descrizione:} Libreria JavaScript che viene utilizzata per effettuare richieste HTTP sia negli ambienti browser che Node.js;

\textbf{Versione:} 11.x

\subsubsection{Material UI}

\textbf{Descrizione:} Framework di componenti React preconfezionati per la creazione di interfacce
utente gradevoli, funzionali e personalizzabili;

\textbf{Versione:} 4.1.x

\subsubsection{Shadcn/ui}

\textbf{Descrizione:} Libreria di componenti React utilizzata per facilitare la codifica del front-end, incorporando
nell’interfaccia grafica componenti prefabbricate, personalizzabili e altamente riutilizzabili.

\textbf{Versione:} 0.8.0

% fine librerie e framework

\subsection{Strumenti e servizi}

\subsubsection{Node.js}

\textbf{Descrizione:} Runtime system per esecuzione di codice Javascript;

\textbf{Versione:} 20.11.0

\subsubsection{NPM}

\textbf{Descrizione:} Gestore di pacchetti per il linguaggio JavaScript e l'ambiente di esecuzione Node.js;

\textbf{Versione:} 9.6.x

\subsubsection{Docker}

\textbf{Descrizione:} Piattaforma di sviluppo e gestione di applicazioni che permette di creare, distribuire e eseguire in software in container virtualizzati.

\textbf{Versione:} 24.0.7


\subsubsection{Git}

\textbf{Descrizione:} Sistema di controllo versione distribuito utilizzato per la gestione del codice sorgente dal parte del gruppo di progetto;

\textbf{Versione:} /

% fine strumenti e servizi

\subsection{Tecnologie per l'analisi statica del codice}

\subsubsection{Prettier}

\textbf{Descrizione:} Strumento di formattazione del codice che aiuta a mantenere uno stile di codifica coerente e leggibile;

\textbf{Versione:} 3.0.x

\subsubsection{Jest}

\textbf{Descrizione:} Framework di testing basato su JavaScript utilizzato per l'implementazione ed esecuzione dei test di unità e di integrazione;

\textbf{Versione:} 29.1.x

\subsubsection{PHPUnit}

\textbf{Descrizione:} Framework di unit test per il linguaggio di programmazione PHP.

\textbf{Versione:} 11.x


\subsection{Tecnologie per l'analisi dinamica del codice}
\subsubsection{React Testing Library}

\textbf{Descrizione:} Libreria di test integrata nativamente che consente di testare il
comportamento dei componentiG React da una prospettiva degli utenti finali;

\textbf{Versione:} 14.0.x

\subsubsection{GitHub Actions}

\textbf{Descrizione:} Servizio di CI/CDG per automatizzare il processo di buildG, test e deployG del
progetto software;

\textbf{Versione:} /
