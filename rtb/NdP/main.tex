\documentclass[a4paper, 11pt]{article}
\usepackage{graphicx} % Required for inserting images
\usepackage{amsmath}
\usepackage{geometry}
\usepackage{hyperref}
\usepackage{setspace}
\usepackage{array}
\usepackage[usenames,dvipsnames]{xcolor}
\usepackage{colortbl} 
\usepackage{tabularray}
\usepackage[italian]{babel}

 \geometry{
 a4paper,
 left=25mm,
 right=25mm,
 top=20mm,
 bottom=20mm,
 }

\setlength{\parskip}{1em}
\setlength{\parindent}{0pt}
\graphicspath{{media/}{../media/}}

\setcounter{secnumdepth}{5}
\setcounter{tocdepth}{5}

\makeatletter
\newcommand\subsubsubsection{\@startsection{paragraph}{4}{\z@}{-2.5ex\@plus -1ex \@minus -.25ex}{1.25ex \@plus .25ex}{\normalfont\normalsize\bfseries}}
\newcommand\subsubsubsubsection{\@startsection{subparagraph}{5}{\z@}{-2.5ex\@plus -1ex \@minus -.25ex}{1.25ex \@plus .25ex}{\normalfont\normalsize\bfseries}}
\makeatother

% comandi per permettere di aggiungere altre sub section innestate

\begin{document}

\begin{minipage}{0.35\linewidth}
    \includegraphics[width=\linewidth]{Logo_Università_Padova.svg.png}
\end{minipage}\hfil
\begin{minipage}{0.55\linewidth}
\textbf{Università degli Studi di Padova} \\
Laurea in Informatica \\
Corso di Ingegneria del Software \\
Anno Accademico 2023/2024
\end{minipage}

\vspace{5mm}

\begin{minipage}{0.35\linewidth}
    \includegraphics[width=\linewidth]{logo rotondo.jpg}
\end{minipage}\hfil
\begin{minipage}{0.55\linewidth}
\textbf{Gruppo:} SWEet16 \\
\textbf{Email:} 
\href{mailto:sweet16.unipd@gmail.com}{\nolinkurl{sweet16.unipd@gmail.com}}
\end{minipage}

\vspace{15mm}

\begin{center}
\begin{Huge}
        \textbf{Norme di Progetto} \\
        \vspace{4mm}
        
\end{Huge}

\vspace{20mm}

\begin{large}
\begin{spacing}{1.4}
\begin{tabular}{c c c}
   Redattori:  &  Alberto M. & \\
   Verificatori: &  &  \\
   Amministratore: & Ales S. & \\
   Destinatari: & T. Vardanega & R. Cardin \\  
   Versione: & 0.1.0 & 
\end{tabular}
\end{spacing}
\end{large}
\end{center}

\pagebreak

\begin{huge}
    \textbf{Registro delle modifiche}
\end{huge}
\vspace{5pt}

\begin{tblr}{
colspec={|X[1.5cm]|X[2cm]|X[2.5cm]|X[2.5cm]|X[5cm]|},
row{odd}={bg=white},
row{even}={bg=lightgray},
row{1}={bg=black,fg=white}
}
    Versione & Data & Autore & Verificatore & Descrizione \\
    0.1.0 & 2024/02/14 & Alberto M. &  & Stesura struttura del documento \\
    \hline
  
\end{tblr}

\pagebreak
\tableofcontents
\pagebreak 


\section{Introduzione e scopo del documento}

\subsection{Scopo del documento}

\subsection{Scopo del prodotto}

\subsection{Glossario}

\subsection{Maturità del documento}

\subsection{Riferimenti}

\subsubsection{Riferimenti normativi}

\subsubsection{Riferimenti informativi}


\section{Processi Primari}

\subsection{Fornitura}

\subsubsection{Descrizione, scopo ed aspettative}

\subsubsection{Proponente}

\subsubsection{Documenti}

\subsubsubsection{Piano di Progetto}

\subsubsubsection{Piano di Qualifica}

%Aggiungere sezione strumenti?

%fine subsection fornitura

\subsection{Sviluppo}

\subsubsection{Descrizione, scopo ed aspettative}

\subsubsection{Analisi dei Requisiti}

\subsubsubsection{Requisiti}

\subsubsubsection{Denominazione e Legenda}

\subsubsubsubsection{Struttura Casi d'Uso}

\subsubsubsubsection{Denominazione Casi d'Uso}

\subsubsubsubsection{Struttura dei Requisiti}

\subsubsubsubsection{Denominazione dei Requisiti}

%fine subsection sviluppo

\subsection{Progettazione}

\subsubsection{Descrizione, scopo ed aspettative}

\subsubsection{Requirements \& Technology Baseline (RTB)}

\subsubsection{Product Baseline (PB)}

%fine progettazione

\subsection{Codifica}

\subsubsection{Descrizione, scopo ed aspettative}

\subsubsection{Stile della Codifica}

\subsubsection{Strumenti utilizzati}

\subsubsubsection{Strumenti per la codifica}

\subsubsubsection{Strumenti di supporto alla codifica}

%fine codifica

%fine processi primari

\section{Processi di Supporto}











\end{document}