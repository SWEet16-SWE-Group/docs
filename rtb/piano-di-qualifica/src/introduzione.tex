\nonstopmode

\section{Introduzione}
\subsection{Scopo del documento}
Lo scopo di questo documento è stabilire standard e obiettivi per valutare la qualità
dei processi e dei prodotti lungo l'intero ciclo di progetto. \\
L'obiettivo è stabilire la qualità del prodotto in modo accurato, attraverso un processo di miglioramento continuo che evolve nel tempo, soprattutto quando viene stabilita
una linea di base. \\
La qualità è definita da una serie di processi che mirano a definire metriche per valutare
l'efficacia ed efficienza. \\
Queste misure quantitative saranno utilizzate per valutare il progresso del progetto.\\
In pratica questo documento si propone di:
\begin{itemize}
    \item Stabilire metriche e procedure di controllo e misurazione appropriate;
    \item Definire la quantità e la qualità dei test e le relative metriche;
    \item Applicare i test e documentare i risultati ottenuti, verificando se corrispondono alle aspettative basate sulle metriche definite.
\end{itemize}

\subsection{Glossario}
Al fine di evitare possibili ambiguità o incomprensioni riguardanti la terminologia usata nel documento, è stato deciso di adottare un glossario in cui vengono riportate le varie definizioni. \\
In questa maniera in esso verranno posti tutti i termini specifici del dominio d’uso con relativi significati. \\
La presenza di un termine all’interno del glossario viene indicata applicando una " $^{G}$ " ad apice della parola.
\subsection{Riferimenti}
\subsubsection{Riferimenti normativi}
    \begin{itemize}
      \item \emph{Capitolato d’appalto}$^{G}$ C3 - Easy Meal: \\
        \url{https://www.math.unipd.it/~tullio/IS-1/2023/Progetto/C3.pdf}
    \end{itemize}
\subsubsection{Riferimenti informativi}
\begin{itemize}
    \item Qualità del software: \\
    \url{https://www.math.unipd.it/~tullio/IS-1/2023/Dispense/T7.pdf}
    \item Qualità di processo:\\
    \url{https://www.math.unipd.it/~tullio/IS-1/2023/Dispense/T8.pdf}
    \item Verifica e validazione: \\
    \url{https://www.math.unipd.it/~tullio/IS-1/2023/Dispense/T9.pdf}
    \item Verifica e validazione, analisi statica:\\
    \url{https://www.math.unipd.it/~tullio/IS-1/2023/Dispense/T10.pdf}
    \item Verifica e validazione, analisi dinamica (aka testing): \\
    \url{https://www.math.unipd.it/~tullio/IS-1/2023/Dispense/T11.pdf}
    \item ISO/IEC 9126:2001: \\
    \url{https://www.iso.org/standard/35733.html}
    \item Indice di Gulpease: \\
    \url{https://it.wikipedia.org/wiki/Indice_Gulpease}
    \item Glossario: \\
    \url{https://github.com/SWEet16-SWE-Group/docs/blob/main/RTB/Documentazione%20Esterna/Glossario.pdf}
\end{itemize}

Tutti i riferimenti (normativi e informativi) a risorse web soggette a variazione sono stati consultati il 2024/04/03.

