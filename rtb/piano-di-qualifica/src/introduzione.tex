
\section{Introduzione}
\subsection{Scopo del documento}
Lo scopo di questo documento è stabilire standard e obiettivi per valutare la qualità
dei processi e dei prodotti lungo l'intero ciclo di progetto. \\
L'obiettivo è stabilire la qualità del prodotto in modo accurato, attraverso un processo di miglioramento continuo che evolve nel tempo, soprattutto quando viene stabilita
una linea di base. \\
La qualità è definita da una serie di processi che mirano a definire metriche per valutare 
l'efficacia ed efficienza. \\
Queste misure quantitative saranno utilizzate per valutare il progresso del progetto.\\
In pratica questo documento si propone di:
\begin{itemize}
    \item Stabilire metriche e procedure di controllo e misurazione appropriate;
    \item Definire la quantità e la qualità dei test e le relative metriche;
    \item Applicare i test e documentare i risultati ottenuti, verificando se corrispondono alle aspettative basate sulle metriche definite.
\end{itemize}

\subsection{Scopo del prodotto}
Lo scopo dell'applicazione è quello di creare una piattaforma che permetta di gestire e semplificare
il processo di prenotazione$^{G}$ di tavoli all'interno dei ristoranti. \\
Sarà inoltre possibile anticipare l'esperienza culinaria visionando prima il menù ed 
andando ad effettuare la propria ordinazione$^{G}$ prima di arrivare al ristorante. \\
Il prodotto offre inoltre un’esperienza di ordinazione delle pietanze$^{G}$ collaborativa e coinvolgente, 
permettendo di condividerla con amici ed, in caso di dubbi, interagire direttamente con lo staff del ristorante.
L’idea è una piattaforma SaaS$^{G}$ (Software as a Service), in cui saranno presenti due tipi di utenti:
\begin{itemize}
    \item Clienti$^{G}$: utente registrato all’interno dell’applicazione, può cercare ristoranti, effettuare prenotazioni, ordinazioni e inserire feedback e recensioni;
    \item Ristoratori$^{G}$: utente registrato all’interno dell’applicazione, può gestire uno o più ristoranti, controllando le prenotazioni e le ordinazioni dei clienti ed i menù del/i ristorante/i.
\end{itemize}
La piattaforma dovrà essere  disponibile attraverso una WebApp$^{G}$ accessibile da qualsiasi dispositivo, esso sia Desktop$^{G}$ o Mobile$^{G}$.
\subsection{Glossario}
Al fine di evitare possibili ambiguità o incomprensioni riguardanti la terminologia usata nel documento, è stato deciso di adottare un glossario in cui vengono riportate le varie definizioni.  \\
In questa maniera in esso verranno posti tutti i termini specifici del dominio d’uso con relativi significati. \\
La presenza di un termine all’interno del glossario viene indicata applicando una " $^{G}$ " ad apice della parola.
\subsection{Riferimenti}
\subsubsection{Riferimenti normativi}
    \begin{itemize}
        \item Capitolato d’appalto$^{G}$ C3 - Easy Meal: \\
        \url{https://www.math.unipd.it/~tullio/IS-1/2023/Progetto/C3.pdf}
    \end{itemize}
\subsubsection{Riferimenti informativi}
\begin{itemize}
    \item Qualità del software: \\
    \url{https://www.math.unipd.it/~tullio/IS-1/2023/Dispense/T7.pdf}
    \item Qualità di processo:\\
    \url{https://www.math.unipd.it/~tullio/IS-1/2023/Dispense/T8.pdf}
    \item Glossario: \\
    \url{} %aggiungere link a glossario
\end{itemize}

Tutti i riferimenti (normativi e informativi) a risorse web soggette a variazione sono stati consultati il 2024/02/22.

