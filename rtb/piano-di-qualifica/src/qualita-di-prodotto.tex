
\section{Qualità di prodotto}
\subsection{Scopo}
Facendo riferimento allo standard ISO/IEC 9126:2001, vengono di seguito riportate le caratteristiche che il prodotto deve avere per essere considerato di qualità.
Vengono inoltre riportate le relative metriche atte a definire un metodo di valutazione del prodotto finale.

\subsection{Usabilità}
Creazione di un prodotto che sia semplice da usare e di facile comprensione nel suo utilizzo da parte di ogni utente.
L'obiettivo è quindi quello di fornire una user experience di elevata qualità.
\subsubsection{Metriche}
\textbf{TA- Tempo Apprendimento}:\\ Tempo necessario all'utente per apprendere l'utilizzo del prodotto \\
\textbf{NP- Numero Passi}:\\ Numero di passi necessari per raggiungere lo scopo voluto \\
\textbf{TE- Tempo Esplorazione}:\\ Tempo spesso nell'esplorazione del prodotto \\
\textbf{NE - Numero Errori}:\\ Numero di errori commessi dall'utente prima di raggiungere lo scopo voluto \\

\subsubsection{Obiettivi}
\begin{table}[h!]
    \begin{tblr}{
        colspec={|X[3cm]|X[4cm]|X[4cm]|X[4cm]|},
        row{odd}={bg=white},
        row{even}={bg=lightgray},
        row{1}={bg=black,fg=white}
        }
        Metrica & Descrizione & Valore accettabile & Valore ideale \\
        TA & Tempo Apprendimento & 10 minuti & 5 minuti \\
        NP & Numero Passi & 20 click & 10 click \\
        TE & Tempo Esplorazione & 5 minuti & 3 minuti \\
        NE & NUmero Errori & 5 & 0 \\
        \hline
     \end{tblr}
    \caption{Metriche e obiettivi}
    \label{tab:1}
\end{table}


\subsection{Manutenibilità}
Facilità di apportare modifiche al prodotto, che sia riusabile e aperto a miglioramenti.
\subsubsection{Metriche}
\subsubsection{Obiettivi}

\subsection{Affidabilità}
Prodotto che sia sempre disponibile, che sia in grado di svolgere le funzionalità implementate e che sia tollerante agli errori.
\subsubsection{Metriche}
\subsubsection{Obiettivi}

\subsection{Efficienza}
Prodotto che fornisca e soddisfi gli obiettivi prefissati con il minor utilizzo di risorse possibile.
\subsubsection{Metriche}
\subsubsection{Obiettivi}

\subsection{Funzionalità}
Soddisfare tutti i requisiti richiesti e descritti all'interno dell'Analisi dei Requisiti v1.0.0.
\subsubsection{Metriche}
\textbf{QPR-RC}: Requirements Coverage \\
Indica la percentuale dei requisiti soddisfatti. Per il calcolo del valore accettato si considerano solo i requisiti obbligatori.\\
Formula valore accettato:
$$RC_{obb} = \frac{NR_{os}}{NR_{ot}} \cdot 100$$

\begin{itemize}
\item $RC_{obb}$ : Requirements Coverage obbligatori;
\item $NR_{os}$ : numero di requisiti obbligatori soddisfatti;
\item $NR_{ot}$ : numero di requisiti obbligatori totali;
\end{itemize}

Formula valore ideale:
$$RC_{i} = \frac{NR_{s}}{NR_{t}} \cdot 100$$

\begin{itemize}
\item $RC_{i}$ : Requirements Coverage ideali;
\item $NR_{s}$ : numero di requisiti ideali soddisfatti;
\item $NR_{t}$ : numero di requisiti ideali totali;
\end{itemize}

\subsubsection{Obiettivi}
\begin{table}[h!]
    \begin{tblr}{
        colspec={|X[3cm]|X[4cm]|X[4cm]|X[4cm]|},
        row{odd}={bg=white},
        row{even}={bg=lightgray},
        row{1}={bg=black,fg=white}
        }
        Metrica & Descrizione & Valore accettabile & Valore ideale \\
        QPR-RC & Requirements Coverage & 100\% $RC_{obb}$ & 100\% $RC_{i}$ \\
        \hline
     \end{tblr}
    \caption{Metriche e obiettivi}
    \label{tab:2}
\end{table}

\subsection{Compatibilità}
Prodotto accessibile al maggior numero di utenti possibile, garantendo la compatibilità con i browser più diffusi.
\subsubsection{Metriche}
\subsubsection{Obiettivi}

\subsection{Portabilità}
Capacità del software di essere portato da un ambiente di sviluppo ad un altro.
\subsubsection{Metriche}
\subsubsection{Obiettivi}

\subsection{Documentazione}
I documenti dovranno essere di facile comprensione, corretti ortograficamente e sintatticamente.
\subsubsection{Metriche}
\subsubsection{Obiettivi}


