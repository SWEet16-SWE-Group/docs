\section{Qualità di processo}
\subsection{Scopo}
La qualità di un prodotto è influenzata dalla qualità dei processi che lo compongono. \\
È quindi necessario dotarsi di metriche che permettano di valutare tali processi e garantire che essi
raggiungano gli obiettivi di qualità fissati.\\
Per garantire una corretta implementazione ed un mantenimento costante, si seguirà
il \textit{ciclo di Deming}, meglio conosciuto come PDCA, che prevede un approccio iterativo funzionale
all'attuazione di un miglioramento continuo.\\
In questa sezione si espongono le metriche scelte ed i livelli di qualità accettabili e ottimali per ciascuna
di esse.\\
Le metriche hanno un identificativo avente come prefisso l'acronimo QPC (Qualità Processo) seguito dal codice della singola metrica.\\

\subsection{Processi primari}

\subsubsection{Fornitura}
\subsubsubsection{Metriche}
\begin{itemize}
    \item \textbf{Budget at Completion (QPC-BAC)}: Totale preventivato del progetto.
\end{itemize}

\begin{itemize}
    \item \textbf{QPC-AC - Actual Cost}: Costo sostenuto per il progetto al momento del calcolo;
    \item \textbf{QPC-ETC - Estimated to Completion}: Stima del valore per la realizzazione delle rimanenti attività;
    \item \textbf{QPC-EAC - Estimated at Completion}: Costo finale stimato alla data della misurazione, revisione del \textbf{QPC-BAC}; $$Formula: \textrm{QPC-AC} + \textrm{QPC-ETC}$$
    \item \textbf{QPC-EV - Earned Value}: Importo guadagnato per il lavoro svolto al momento del calcolo; $$Formula: (\%Lavoro \enspace svolto) \cdot QPC-EAC$$
    \item \textbf{QPC-PV - Planned Value}: Importo pianificato in base al lavoro svolto, al momento del calcolo; $$Formula: (\%Lavoro \enspace pianificato) \cdot \textrm{QPC-BAC}$$
    \item \textbf{QPC-SV - Schedule Variance}: Stato (anticipo/ritardo) della pianificazione. Un valore negativo indica che si è in ritardo rispetto alla pianificazione; $$Formula \enspace base: \textrm{QPC-EV} - \textrm{QPC-PV}$$ $$Formula \enspace in \enspace \%: \textrm{QPC-EV} / \textrm{QPC-PV} \cdot 100$$
    \item \textbf{QPC-CV - Cost Variance}: Differenza tra il budget a disposizione e quello effettivamente utilizzato. Un valore negativo indica che si sta lavorando in perdita. $$Formula: \textrm{QPC-EV} - \textrm{QPC-AC}$$ $$Formula \enspace in \enspace \%: \textrm{QPC-EV} / \textrm{QPC-AC} \cdot 100$$
\end{itemize}
\subsubsubsection{Obiettivi}
\begin{table}[H]
    \begin{tblr}{
        colspec={|X[3cm]|X[5cm]|X[4cm]|X[3cm]|},
        row{odd}={bg=white},
        row{even}={bg=lightgray},
        row{1}={bg=black, fg=white}
}
        Metrica & Descrizione & Valore accettabile & Valore ideale \\
        QPC-AC & Actual Cost & & \\
        QPC-ETC & Estimated to Completion & ${\geq}$ 0\% & ${\leq}$ QPC-EAC\\
        QPC-EAC & Estimated at Completion & Errore del ${\pm}$ 3\% rispetto a QPC-BAC & = QPC-BAC\\
        QPC-EV & Earned Value & ${\geq}$ 0 & ${\leq}$ QPC-EAC \\
        QPC-PV & Planned Value & ${\geq}$ 0 & ${\leq}$ QPC-BAC \\
        QPC-SV & Schedule Variance & ${\geq}$ -10\% & ${\geq}$ 0 \\
        QPC-CV & Cost Variance & ${\geq}$ -5\% & ${\geq}$ 0 \\
        \hline
     \end{tblr}
    \caption{Metriche e obiettivi fornitura}
    \label{tab:20}
\end{table}

\subsubsection{Sviluppo}
\subsubsubsection{Progettazione architetturale}
\subsubsubsubsection{Metriche}

\begin{itemize}
    \item \textbf{QPC-SFIN - Structural Fan-in}: Indice di utilità, indica quante componenti utilizzano un determinato modulo. Un valore alto indica che il componente è molto usato;
    \item \textbf{QPC-SFOUT - Structural Fan-out}: Indice di dipendenza, indica quante componenti sono utilizzate dalla componente in esame. Un valore alto indica un elevato utilizzo della componente.
\end{itemize}

\subsubsubsubsection{Obiettivi}
\begin{table}[H]
    \begin{tblr}{
        colspec={|X[3cm]|X[5cm]|X[4cm]|X[3cm]|},
        row{odd}={bg=white},
        row{even}={bg=lightgray},
        row{1}={bg=black, fg=white}
}
        Metrica & Descrizione & Valore accettabile & Valore ideale \\
        QPC-SFIN & Structural Fan-in & - & - \\
        QPC-SFOUT & Structural Fan-out & - & - \\
        \hline
     \end{tblr}
    \caption{Metriche e obiettivi progettazione architetturale}
    \label{tab:21}
\end{table}

\subsubsubsection{Progettazione di dettaglio}
\subsubsubsubsection{Metriche}
\begin{itemize}
    \item \textbf{QPC-NM - Number of Methods}: Indica il numero medio di metodi per package. Un numero eccessivo potrebbe indicare la necessità di refactoring.
\end{itemize}

\subsubsubsubsection{Obiettivi}
\begin{table}[H]
    \begin{tblr}{
        colspec={|X[3cm]|X[5cm]|X[4cm]|X[3cm]|},
        row{odd}={bg=white},
        row{even}={bg=lightgray},
        row{1}={bg=black, fg=white}
}
        Metrica & Descrizione & Valore accettabile & Valore ideale \\
        QPC-NM & Number of Methods & 3-11 & 3-8 \\
        \hline
     \end{tblr}
    \caption{Metriche e obiettivi progettazione di dettaglio}
    \label{tab:22}
\end{table}

\subsubsubsection{Codifica}
\subsubsubsubsection{Metriche}
\begin{itemize}
    \item \textbf{QPC-BLC: Bugs for Line of Code}: Indice del numero di righe di codice contenenti bug ed errori al proprio interno;
    \item \textbf{QPC-VNU: Variabili Non Utilizzate}: Indice di un errore di programmazione: Le variabile non utilizzate sporcano il codice e fanno allocare memoria inutilmente;
    \item \textbf{QPC-VND: Variabili Non Definite}: Fonte comune di bug nel software: Sono variabili dichiarate ma non inizializzate ad un valore noto definito prima di essere utilizzate.
\end{itemize}

\subsubsubsubsection{Obiettivi}
\begin{table}[H]
    \begin{tblr}{
        colspec={|X[3cm]|X[5cm]|X[4cm]|X[3cm]|},
        row{odd}={bg=white},
        row{even}={bg=lightgray},
        row{1}={bg=black, fg=white}
}
        Metrica & Descrizione & Valore accettabile & Valore ideale \\
        QPC-BLC & Bugs for Line of Code & 0-70 & 0-25 \\
        QPC-VNU & Variabili Non Utilizzate & 0 & 0 \\
        QPC-VND & Variabili Non Definite & 0 & 0 \\
        \hline
     \end{tblr}
    \caption{Metriche e obiettivi codifica}
    \label{tab:23}
\end{table}


\subsection{Processi di supporto}

\subsubsection{Documentazione}
\subsubsubsection{Metriche}
\textbf{QPR-DOC}: Indice di Gulpease \\
L'Indice di Gulpease è un indice di leggibilità di un testo tarato sulla lingua italiana.
Si basa sulla seguente formula:
$$GULPEASE = 89+\frac{(NF \cdot 300) - (10 \cdot NL)}{NP}$$
\begin{itemize}
    \item \textbf{NF:} Numero frasi;
    \item \textbf{NL:} Numero lettere;
    \item \textbf{NP:} Numero parole.
\end{itemize}

In generale risulta la seguente suddivisione:
\begin{itemize}
   \item GULPEASE ${\le}$ 80: Difficile da leggere per chi ha un licenza elementare;
    \item GULPEASE ${\le}$ 60: Difficile da leggere per chi ha un licenza media;
    \item GULPEASE ${\le}$ 40: Difficile da leggere per chi ha un diploma superiore.
\end{itemize}


\subsubsubsection{Obiettivi}
\begin{table}[H]
    \begin{tblr}{
        colspec={|X[3cm]|X[4cm]|X[4cm]|X[4cm]|},
        row{odd}={bg=white},
        row{even}={bg=lightgray},
        row{1}={bg=black, fg=white}
}
        Metrica & Descrizione & Valore accettabile & Valore ideale \\
        QPR-DOC & Indice di Gulpease & GULPEASE ${\geq}$ 40 & GULPEASE ${\geq}$ 60 \\
        \hline
     \end{tblr}
    \caption{Metriche Documentazione}
    \label{tab:24}
\end{table}

\subsubsection{Gestione delle qualità}
\subsubsubsection{Metriche}
\begin{itemize}
    \item \textbf{QPC-QMS: Quality Metrics Satisfied}: Percentuale di metriche di qualità soddisfatte. $$\textrm{QPC-QMS} = \frac{NQMS}{TQM} \cdot 100$$
\end{itemize}

\subsubsubsection{Obiettivi}
\begin{table}[H]
    \begin{tblr}{
        colspec={|X[3cm]|X[5cm]|X[4cm]|X[3cm]|},
        row{odd}={bg=white},
        row{even}={bg=lightgray},
        row{1}={bg=black, fg=white}
}
        Metrica & Descrizione & Valore accettabile & Valore ideale \\
        QPC-QMS & Quality Metrics Satisfied & ${\geq}$ 90\% & 100\% \\
        \hline
     \end{tblr}
    \caption{Metriche e obiettivi gestione della qualità}
    \label{tab:25}
\end{table}


\subsubsection{Verifica}
\subsubsubsection{Metriche}
\begin{itemize}
    \item \textbf{QPC-CC - Code Coverage}: Definisce la misura della quantità di codice di un programma che viene eseguita durante uno specifico test. 
    Una percentuale alta indica che il codice è stato testato in modo approfondito nelle sue diverse parti e quindi vi è una minore probabilità che ci siano bug;
    \item \textbf{QPC-SC - Statement Coverage}: Tecnica di test di tipo white box che prevede l'esecuzione di tutte le istruzioni presenti nel codice sorgente almeno una volta;
    \item \textbf{QPC-BC - Branch Coverage}: Indice di quante diramazioni del codice vengono eseguite dai test. 
    Un "ramo" è uno dei possibili percorsi di esecuzione che il codice può seguire dopo che un'istruzione decisionale (es. if) viene valutata;
    \item \textbf{QPC-DCC - Decision/Condition Coverage}: Il Decision/Condition Coverage è un criterio di copertura del codice utilizzato nei test del software.
\end{itemize}

\subsubsubsection{Obiettivi}
\begin{table}[H]
    \begin{tblr}{
        colspec={|X[3cm]|X[5cm]|X[4cm]|X[3cm]|},
        row{odd}={bg=white},
        row{even}={bg=lightgray},
        row{1}={bg=black, fg=white}
}
        Metrica & Descrizione & Valore accettabile & Valore ideale \\
        QPC-CC & Code Coverage & ${\geq}$ 80\% & 90\% \\
        QPC-SC & Statement Coverage & ${\geq}$ 70\% & 85\% \\
        QPC-BC & Branch Coverage & ${\geq}$ 50\% & 75\% \\
        QPC-DCC & Decision/Condition Coverage & - & - \\
        \hline
     \end{tblr}
    \caption{Metriche e obiettivi verifica}
    \label{tab:26}
\end{table}
