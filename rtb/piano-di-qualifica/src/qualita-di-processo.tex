

\section{Qualità di processo}
\subsection{Scopo}
La qualità di un prodotto è influenzata dalla qualità dei processi che lo compongono.
È quindi necessario dorarsi di metriche che permettano di valutare tali processi e garantire che essi
raggiungano gli obiettivi di qualità fissati.\\
Per garantire una corretta implementazione ed un mantenimento costante, si seguirà 
il \textit{ciclo di Deming} , meglio conosciuto come PDCA, che prevede un approccio iterativo funzionale
all'attuazione di un miglioramento continuo.
In questa sezione si espongono le metriche scelte ed i livelli di qualità accettabili e ottimali per ciascuna
di esse.\\
Le metriche hanno un identificativo avente come prefisso l'acronimo QPC (Qualità Processo) seguito dal codice della singola metrica.\\

\subsection{Processi primari}

\subsubsection{Fornitura}
Processo che consiste nell'individuare e decidere quali sono le procedure e risorse adatte a soddisfare le necessità del cliente.\\
\subsubsection{Metriche}
\begin{itemize}
    \item \textbf{Budget at Completion (BAC)}: Costo totale preventivato del progetto.
\end{itemize}
\begin{itemize}
    \item \textbf{QPC-AC - Actual Cost}: Costo sostenuto per il progetto al momento del calcolo;
    \item \textbf{QPC-EV - Earned Value}: Importo guadagnato per il lavoro svolto al momento del calcolo;
    \item \textbf{QPC-PV - Planned Value}: Importo pianificato in base al lavoro svolto, al momento del calcolo:
    \item \textbf{QPR-NE - Numero Errori}: Numero di errori commessi dall'utente prima di raggiungere lo scopo voluto.
\end{itemize}
\subsubsection{Obiettivi}
\begin{table}[h!]
    \begin{tblr}{
        colspec={|X[3cm]|X[4cm]|X[4cm]|X[4cm]|},
        row{odd}={bg=white},
        row{even}={bg=lightgray},
        row{1}={bg=black, fg=white}
}
        Metrica & Descrizione & Valore accettabile & Valore ideale \\
        QPR-TA & Tempo Apprendimento & 10 minuti & 5 minuti \\
        QPR-NP & Numero Passi & 20 click & 10 click \\
        QPR-TE & Tempo Esplorazione & 5 minuti & 3 minuti \\
        QPR-NE & NUmero Errori & 5 & 0 \\
        \hline
     \end{tblr}
    \caption{Metriche usabilità}
    \label{tab:1}
\end{table}

\subsubsection{Sviluppo}

\subsection{Processi di supporto}

\subsubsection{Documentazione}

\subsubsection{Gestione delle qualità}

\subsubsection{Verifica}
