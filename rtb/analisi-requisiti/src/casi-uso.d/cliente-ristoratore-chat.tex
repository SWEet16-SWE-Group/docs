
\textbf{UC49-Avvia chat}
\begin{itemize}
\item \textbf{Attore principale:} Cliente.
\item \textbf{Precondizioni:} È in corso una prenotazione e il cliente non ha ancora avviato la chat. % TODO non so come mettere a parole che deve essere il cliente a voler comunicare con il ristoratore
\item \textbf{Postcondizioni:} Diventa possibile comunicare con il ristoratore.
\item \textbf{Scenario principale:}
\begin{enumerate}
    \item Il cliente ha dei dubbi che vuole chiarire.
    \item Il cliente preme il bottone di avvio della chat.
    \item Il sistema crea la chat.
    \item Viene visualizzato un bottone di apertura della chat al posto dell'avvio chat.
    \item Diventa possibile comunicare bidirezionalmente tra il/i cliente/clienti e il ristoratore. 
\end{enumerate}
\end{itemize}

\textbf{UC50-Apertura chat}
\begin{itemize}
\item \textbf{Attore principale:} Cliente / Ristoratore.
\item \textbf{Precondizioni:} Il cliente ha avviato la chat.
\item \textbf{Postcondizioni:} La viene aperta. % TODO anche qua non so come metterlo a parole
\item \textbf{Scenario principale:}
\begin{enumerate}
    \item L'attore principale vuole visionare i messaggi in chat o inviare ulteriori messaggi.
    \item L'attore clicca sul tasto della chat.
    \item Diventa possibile interagire con la chat.
\end{enumerate}
\end{itemize}

\textbf{UC50.1-Invio messaggio}
\begin{itemize}
\item \textbf{Attore principale:} Cliente / Ristoratore
\item \textbf{Precondizioni:}
\item \textbf{Postcondizioni:}
\item \textbf{Scenario principale:}
\begin{enumerate}
    \item
\end{enumerate}
\end{itemize}

\textbf{UC50.2-Lettura messaggio}
\begin{itemize}
\item \textbf{Attore principale:} Cliente / Ristoratore
\item \textbf{Precondizioni:}
\item \textbf{Postcondizioni:}
\item \textbf{Scenario secondario:}
\begin{enumerate}
    \item
\end{enumerate}
\end{itemize}
