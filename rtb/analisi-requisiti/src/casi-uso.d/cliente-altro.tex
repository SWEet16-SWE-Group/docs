
\textbf{UC-}
\begin{itemize}
\item \textbf{Attore principale:}
\item \textbf{Precondizioni:}
\item \textbf{Postcondizioni:}
\item \textbf{Scenario principale:}
\begin{enumerate}
    \item
\end{enumerate}
\end{itemize}

\textbf{UC29-Inserimento di feedback e recensioni}
\begin{itemize}
\item \textbf{Attore principale:} Cliente.
\item \textbf{Precondizioni:} Esiste almeno una prenotazione pagata con quel ristorante.
\item \textbf{Postcondizioni:} È stato lasciato il feedback.
\item \textbf{Scenario principale:}
\begin{enumerate}
    \item Il cliente clicca sul tasto per lasciare la recensione;
    \item Il sistema mostra un form contente tre voti da una a cinque stelle, rispettivamente per:
  \begin{itemize}
    \item Menù;
    \item Servizio;
    \item Prezzo.
  \end{itemize}
      Segue poi un campo di testo;
    \item Il cliente può riempie i tre voti (obbligatori) e il campo di testo (facoltatico);
    \item Il cliente fa il submit del form;
    \item Il sistema reindirizza alla pagina del ristorante dove è possibile vedere la propria recensione.
\end{enumerate}
\end{itemize}

\textbf{UC30-Visualizzazione notifica cambio stato prenotazione}
\begin{itemize}
\item \textbf{Attore principale:} Cliente.
\item \textbf{Precondizioni:} Il cliente ha una prenotazione in attesa.
\item \textbf{Postcondizioni:} Il cliente visualizza una bolla con il nuovo stato della prenotazione
\item \textbf{Scenario principale:}
\begin{enumerate}
    % TODO non so come descriverlo meglio, ci ho provato
    \item Il ristoratore accetta o rifiuta la prenotazione;
    \item Al cliente compare una bolla contentente il nuovo stato della prenotazione.
\end{enumerate}
\end{itemize}

\textbf{UC32-Visualizzazione notifica accettazione invito da parte di altri utenti}
\begin{itemize}
\item \textbf{Attore principale:} Cliente.
\item \textbf{Precondizioni:} Il cliente ha una prenotazione attiva e ha condiviso il link di invito tramite terze parti.
\item \textbf{Postcondizioni:} Il cliente visualizza una bolla con il nome del profilo di chi ha accettato l'invito.
\item \textbf{Scenario principale:}
\begin{enumerate}
    \item Il cliente mittente copia il link fornito per invitare altri profili alla propria prenotazione e lo diffonde tramite
      servizi di terze parti;
    \item Un destinatario del link lo clicca;
    \item Il sistema chiede il login caso l'utente destinatario non sia già loggato;
    \item Il sistema i profili cliente disponibili con cui aggiungersi alla prenotazione;
    \item L'utente destinatario seleziona un profilo;
    \item Al cliente mittente compare una bolla contentente il nuovo profilo aggiuntosi alla prenotazione.
\end{enumerate}
\end{itemize}

% TODO forse;  mettere UC48-Visualizzazione messaggio in chat 
