\subsubsection{UC19-Prenotazione di un tavolo}
\begin{itemize}
    \item \textbf{Attore principale: }Cliente.
    \item \textbf{Precondizioni: }Il cliente sta visualizzando le informazioni di un ristorante.
    \item \textbf{Postcondizioni: }La prenotazione è salvata nella lista delle prenotazioni del cliente.
    \item \textbf{Scenario principale:} 
        \begin{enumerate}
            \item Il cliente seleziona la funzionalità di richiesta di una nuova prenotazione;
            \item Il sistema illustra i giorni in cui è possibile effettuare la richiesta;
            \item Il cliente seleziona il giorno di calendario;
            \item Il sistema illustra le fasce orarie disponibili;
            \item Il cliente seleziona l'orario;
            \item Il cliente inserisce il numero di persone che parteciperanno alla prenotazione;
            \item Il cliente dà la conferma ed effettua la richiesta;
            \item il cliente visualizza la lista delle sue prenotazioni,ove è presente la prenotazione appena
            effettuata.
        \end{enumerate}
\end{itemize}

\subsubsection{UC20-Visualizzazione lista prenotazioni}
\begin{itemize}
    \item \textbf{Attore principale: } Cliente.
    \item \textbf{Precondizioni: }Il cliente sta visualizzando le sue informazioni personali.
    \item \textbf{Postcondizioni: }Il cliente visualizza la lista delle sue prenotazioni.
    \item \textbf{Scenario principale:} 
        \begin{enumerate}
            \item L'utente seleziona la funzionalità di visualizzazione delle sue prenotazioni;
            \item L'utente visualizza la lista delle sue prenotazioni divise in due sezioni:
              \begin{itemize}
                \item Attive: prenotazioni future negli stati "In attesa" o "Accettata";
                \item Scadute: prenotazioni passate o nello stato di "Rifiutata";
              \end{itemize}
        \end{enumerate}
\end{itemize}

\subsubsection{UC21-Visualizzazione singola prenotazione}
\begin{itemize}
    \item \textbf{Attore principale: } Cliente.
    \item \textbf{Precondizioni: }Il cliente sta visualizzando la lista delle sue prenotazioni.
    \item \textbf{Postcondizioni: }Il cliente visualizza le informazioni relative ad una singola prenotazione.
    \item \textbf{Scenario principale:} 
        \begin{enumerate}
            \item L'utente seleziona una prenotazione in particolare;
            \item L'utente visualizza la lista delle informazioni relative alla prenotazione selezionata:
            \begin{itemize}
                \item Il nome del ristorante;
                \item L'indirizzo del ristorante;
                \item L'orario;
                \item Il numero di persone ;
                \item Lo stato della prenotazione (se è in attesa di conferma da parte del ristoratore
                o se è già stata confermata).
            \item Il cliente può andare a generare il link per la condivisione della prenotazione (si veda UC23).
            \end{itemize}
        \end{enumerate}
    \item \textbf{Estensione: }UCE21-Prenotazione rifiutata
\end{itemize}

        %L'estensione avviene in seguito all'UC34 del ristoratore, ma credo che dobbiamo tenere la notazione
        %UCE che usiamo per le estensioni ? 
\subsubsection{UCE21-Prenotazione rifiutata}
\begin{itemize}
    \item \textbf{Descrizione: }Se il ristoratore rifiuta una delle prenotazioni del cliente in attesa di conferma,
    mentre quest'ultimo sta visualizzando la lista delle sue prenotazioni, il cliente si accorge del rifiuto quando
    seleziona la suddetta prenotazione.
    \item \textbf{Scenario alternativo: }
    \begin{enumerate}
        \item Il cliente seleziona una prenotazione rifiutata mentre sta visualizzando la lista 
        delle sue prenotazioni;
        \item Il sistema comunica un messaggio di errore al cliente,dicendogli che la prenotazione selezionata 
        è stata rifiutata e che non è più possibile visualizzarne le informazioni.
    \end{enumerate}
\end{itemize}

\subsubsection{UC22-Cancellazione prenotazione}
\begin{itemize}
    \item \textbf{Attore principale: } Cliente.
    \item \textbf{Precondizioni: }Il cliente sta visualizzando le informazioni di una singola prenotazione
    e c'è una differenza di almeno 48 ore tra il datetime in cui il cliente fa la richiesta di cancellazione e quello della prenotazione.
    \item \textbf{Postcondizioni: }La prenotazione è stata cancellata e non è più presente nella lista di prenotazioni
    del cliente.
    \item \textbf{Scenario principale:} 
        \begin{enumerate}
            \item Il cliente seleziona l'opzione di cancellazione della prenotazione;
            \item Il cliente dà la conferma ed effettua la cancellazione;
            \item Il sistema notifica il ristoratore dell'avvenuta cancellazione;
            \item Il cliente visualizza la lista delle sue prenotazioni aggiornata,priva di quella appena
            cancellata.
        \end{enumerate}
\end{itemize}

\subsubsection{UC23-Condivisione della prenotazione}
\begin{itemize}
    \item \textbf{Attore principale: } Cliente.
    \item \textbf{Precondizioni: }Il cliente sta visualizzando le informazioni di una sua singola prenotazione (si veda UC21).
    \item \textbf{Postcondizioni: }Il cliente ha una copia testuale del link relativo alla prenotazione.
    \item \textbf{Scenario principale:} 
        \begin{enumerate}
            \item L'utente seleziona la funzionalità di condivione del link afferente alla prenotazione;
            \item Il sistema genera il link relativo alla prenotazione;
            \item Il cliente selezione l'opzione di copia testuale del link.
            %Come ad esempio si fa con il link delle repo di git hub;
            %Non so se aggiungere dettagli ,tipo condivisione del link su servizio di messaggistica,
            %che però teoricamente è fuori dal sistema;è anche vero che il sistema può gestire come su telegram
            %o whatsapp fino alla scelta del servizio di messaggistica per la condivisione di un contenuto.
        \end{enumerate}
\end{itemize}

\subsubsection{UC24-Accettazione di invito di partecipazione ad una prenotazione}
\begin{itemize}
    \item \textbf{Attore principale: } Utente non riconosciuto.
    \item \textbf{Precondizioni: }L'utente possiede il link per partecipare ad una prenotazione.
    \item \textbf{Postcondizioni: }La prenotazione è salvata nella lista delle prenotazioni dell'utente.
    \item \textbf{Scenario principale:} 
        \begin{enumerate}
            \item L'utente seleziona il link di partecipazione alla prenotazione;
            \item L'utente effettua il login per accedere al sistema;
            \item L'utente seleziona il profilo cliente con il quale partecipare alla prenotazione;
            \item L'utente visualizza la prenotazione alla quale è stato invitato nella lista delle sue
            prenotazioni.
        \end{enumerate}
        \item \textbf{Estensione: }UCE24-Raggiunto numero massimo di partecipanti
    \end{itemize}

\subsubsection{UCE24-Raggiunto numero massimo di partecipanti}
\begin{itemize}
    \item \textbf{Descrizione: }Possono accedere alla prenotazione tramite il link di
    invito un numero massimo di utenti uguale al numero di partecipanti indicato al momento della
    prenotazione.
    \item \textbf{Scenario alternativo: }
    \begin{enumerate}
        \item L'utente riconosciuto seleziona il profilo con il quale partecipare alla prenotazione;
        \item Il sistema notifica all'utente un messaggio di errore,in cui gli comunica che è stato
        già raggiunto il numero massimo di partecipanti ;
        %Forse l'estensione può terminare anche con il messaggio di errore.
        \item L'utente visualizza la sua lista di prenotazioni senza la prenotazione alla quale voleva
        partecipare.
    \end{enumerate}
\end{itemize}

