\subsubsection{UC1-Registrazione account}
\begin{itemize}
    \item \textbf{Attore principale: }Utente non riconosciuto.
    \item \textbf{Precondizioni: }L'utente è connesso al sistema.
    \item \textbf{Postcondizioni: }L'account dell'utente e le informazioni ad esso collegate, sono registrati nel sistema.
    \item \textbf{Scenario principale:} 
        \begin{enumerate}
            \item L'utente sceglie l'opzione di registrazione di un account;
            \item Il sistema chiede all'utente di inserire una email ed una password;
            \item L'utente inserisce l'email e la password;
            \item L'utente conferma i valori inseriti;
            \item Il sistema comunica all'utente che la registrazione è andata 
            a buon fine.
            \item Il cliente visualizza un profilo di tipo cliente creato di default dal sistema.
        \end{enumerate}
    \item \textbf{Estensioni:}
        \begin{itemize}
                \item UCE1.1-Campo mancante;
                \item UC1.2-Email già registrata nel sistema.
        \end{itemize}
\end{itemize}
    
\subsubsection{UCE1.1-Campo mancante}
\begin{itemize}
    \item \textbf{Descrizione: }Per effettuare la registrazione,deve essere inserito un valore sia per la password che per l'email.
    \item \textbf{Scenario alternativo:}
    \begin{enumerate}
        \item Al momento della conferma,il sistema rileva che uno dei campi (o entrambi) risulta non compilato;
        \item Il sistema comunica la natura dell'errore all'utente;
        \item L'utente visualizza i campi di mail e password con i valori inseriti precedentemente.
    \end{enumerate}
\end{itemize}

\subsubsection{UC1.2-Email già registrata nel sistema}
\begin{itemize}
    \item \textbf{Descrizione: }L'utente non può registrare con una email già presente nel sistema.
    \item \textbf{Scenario alternativo:}
    \begin{itemize}
        \item L'email inserita dall'utente al momento della registrazione risulta associata ad un
        account già registrato nel sistema;
        \item Il sistema comunica la natura dell'errore all'utente;
        \item L'utente è reindirizzato alla homepage di registrazione.
    \end{itemize}
\end{itemize}
\break

\subsubsection{UC2-Login}
\begin{itemize}
\item \textbf{Attore principale:} Utente non riconosciuto.
\item \textbf{Precondizioni:} L'utente è connesso al sistema.
\item \textbf{Postcondizioni:} L'utente è autenticato presso il sistema.
\item \textbf{Scenario principale:}
\begin{enumerate}
    \item L'utente sceglie l'opzione di login;
    \item L'utente inserisce la sua password;
    \item l'utente inserisce la sua email;
    \item L'utente conferma i dati inseriti al sistema;
    \item Il sistema verifica l'esistenza di un account con i suddetti dati;
    \item L'utente visualizza la lista dei profili afferenti al suo account.
\end{enumerate}
    \item \textbf{Estensione: }UCE2-Credenziali non corrette.
\end{itemize}

\subsubsection{UCE2-Credenziali non corrette}
\textbf{Descrizione: }Il login può non andare a buon fine se l'email inserita dall'utente non è registrata 
o la password non è corretta; per questioni di sicurezza, all'utente viene notificato un errore generico.
\textbf{Scenario secondario:}
\begin{enumerate}
    \item Dopo la conferma dei dati inseriti dall'utente,il sistema verifica 
    che l'email non è presente nel sistema o che la password associata non è corretta;
    \item Il sistema comunica all'utente che le credenziali inserite non sono corrette;
    \item L'utente può riprovare ad effettuare il login ,inserendo di nuovo l'email e la password.
\end{enumerate}

\subsubsection{UC16-Visualizzazione lista ristoranti}
\begin{itemize}
\item \textbf{Attore principale:} Utente non riconosciuto / cliente.
\item \textbf{Precondizioni:} L'utente è connesso al sistema.
\item \textbf{Postcondizioni:} L'utente visualizza una lista di ristoranti.
\item \textbf{Scenario principale:}
\begin{enumerate}
    \item L'utente seleziona la funzionalità di visualizzazione di una lista di ristoranti;
    \item L'utente visualizza una lista di ristoranti,ordinati secondo la loro valutazione media in ordine decrescente.
\end{enumerate}
\end{itemize}

\subsubsection{UC17-Ricerca ristorante}
\begin{itemize}
\item \textbf{Attore principale:}Utente non riconosciuto / cliente.
\item \textbf{Precondizioni:} L'utente è connesso al sistema.
\item \textbf{Postcondizioni:} L'utente visualizza la lista dei ristoranti corrispondenti ai criteri inseriti
dell'utente.
\item \textbf{Scenario principale:}
\begin{enumerate}
    \item L'utente seleziona la funzionalità di ricerca di un ristorante;
    \item L'utente può effettuare la ricerca inserendo uno o più parametri ,corrispondenti
    ai seguenti criteri:
    \begin{itemize}
        \item Il nome del ristorante (vedi UC17.1-Ricerca per nome);
        \item La città del ristorante (vedi UC17.2-Ricerca per città);
        \item La valutazione media del ristorante (vedi UC17.3-Ricerca per valutazione);
        \item La tipologia di cucina (vedi UC17.4-Ricerca per tipologia di cucina);
        \item L'orario (vedi UC17.5-Ricerca ristorante per orario) ;
        \item La data (vedi UC17.6-Ricerca ristorante per data); 
    \end{itemize}
    \item  Il sistema filtra la lista di ristoranti secondo i criteri inseriti;
    \item L'utente visualizza la lista dei ristoranti che rispettano i criteri da lui inseriti.
\end{enumerate}
\end{itemize}

\subsubsection{UC17.1-Ricerca ristorante per nome}
\begin{itemize}
\item \textbf{Attore principale:}Utente non riconosciuto / cliente.
\item \textbf{Precondizioni:} L'utente è connesso al sistema.
\item \textbf{Postcondizioni:} L'utente visualizza la lista dei ristoranti corrispondenti 
alla ricerca per nome da lui inserito.
\item \textbf{Scenario principale:}
\begin{enumerate}
    \item L'utente seleziona la funzionalità di ricerca di un ristorante;
    \item L'utente inserisce il testo che deve essere contenuto nel nome; 
    \item Il sistema filtra la lista di ristoranti secondo il criterio inserito;
    \item L'utente visualizza la lista dei ristoranti corrispondenti al nome da lui inserito.
\end{enumerate}
\end{itemize}

\subsubsection{UC17.2-Ricerca ristorante per città}
\begin{itemize}
\item \textbf{Attore principale:}Utente non riconosciuto / cliente.
\item \textbf{Precondizioni:} L'utente è connesso al sistema.
\item \textbf{Postcondizioni:} L'utente visualizza la lista dei ristoranti corrispondenti alla città da lui inserita.
\item \textbf{Scenario principale:}
\begin{enumerate}
    \item L'utente seleziona la funzionalità di ricerca di un ristorante;
    \item L'utente inserisce la città come parametro di ricerca;
    \item Il sistema filtra la lista di ristoranti secondo il criterio inserito;
    \item L'utente visualizza la lista dei ristoranti corrispondenti alla città inserita.
\end{enumerate}
\end{itemize}

\subsubsection{UC17.3-Ricerca ristorante per valutazione}
\begin{itemize}
\item \textbf{Attore principale:}Utente non riconosciuto / cliente.
\item \textbf{Precondizioni:} L'utente è connesso al sistema.
\item \textbf{Postcondizioni:} L'utente visualizza la lista dei ristoranti la cui valutazione è maggiore o uguale alla
valutazione da lui inserita.
\item \textbf{Scenario principale:}
\begin{enumerate}
    \item L'utente seleziona la funzionalità di ricerca di un ristorante;
    \item L'utente inserisce il valore della valutazione che desidera;
    \item Il sistema filtra la lista di ristoranti secondo il criterio inserito;
    \item L'utente visualizza la lista dei ristoranti con valutazione maggiore o uguale a quella da lui inserita.
\end{enumerate}
\end{itemize}

\subsubsection{UC17.4-Ricerca ristorante per tipologia di cucina}
\begin{itemize}
\item \textbf{Attore principale:}Utente non riconosciuto / cliente.
\item \textbf{Precondizioni:} L'utente è connesso al sistema.
\item \textbf{Postcondizioni:} L'utente visualizza la lista dei ristoranti che offrono la tipologia di cucina
da lui inserita.
\item \textbf{Scenario principale:}
\begin{enumerate}
    \item L'utente seleziona la funzionalità di ricerca di un ristorante;
    \item L'utente seleziona una o più tipologie di cucina alle quali è interessato;
    \item Il sistema filtra la lista di ristoranti secondo il criterio inserito; 
    \item L'utente visualizza la lista dei ristoranti che offrono la/e tipologia/e di cucina da egli cercata.
\end{enumerate}
\end{itemize}

\subsubsection{UC17.5-Ricerca ristorante per orario}
\begin{itemize}
\item \textbf{Attore principale: }Utente non riconosciuto / cliente.
\item \textbf{Precondizioni:} L'utente è connesso al sistema.
\item \textbf{Postcondizioni:} L'utente visualizza la lista dei ristoranti che hanno posti disponibili nell'orario 
da lui scelto.
\item \textbf{Scenario principale:}
\begin{enumerate}
    \item L'utente seleziona la funzionalità di ricerca di un ristorante;
    \item L'utente seleziona l'orario ;
    \item Il sistema filtra la lista di ristoranti secondo il criterio inserito;
    \item L'utente visualizza la lista dei ristoranti che hanno posti disponibili nell'orario da lui
    selezionato.
\end{enumerate}
\end{itemize}

\subsubsection{UC17.6-Ricerca ristorante per data}
\begin{itemize}
\item \textbf{Attore principale:} Utente non riconosciuto / cliente.
\item \textbf{Precondizioni:} L'utente è connesso al sistema.
\item \textbf{Postcondizioni:} L'utente visualizza la lista dei ristoranti che hanno posti disponibili
nella data da lui selezionata.
\item \textbf{Scenario principale:}
\begin{enumerate}
    \item L'utente seleziona la funzionalità di ricerca di un ristorante;
    \item L'utente seleziona la data;
    \item Il sistema filtra la lista di ristoranti secondo il criterio inserito;
    \item L'utente visualizza la lista dei ristoranti che hanno posti disponibili
    nella data selezionata .
\end{enumerate}
\end{itemize}

\textbf{U18-Visualizzazione ristorante}
\begin{itemize}
\item \textbf{Attore principale:} Utente non riconosciuto / cliente.
\item \textbf{Precondizioni:} L'utente è connesso al sistema e sta visualizzando una lista di ristoranti.
\item \textbf{Postcondizioni:} L'utente visualizza le informazioni relative al ristorante selezionato.
\item \textbf{Scenario principale:}
\begin{enumerate}
    \item L'utente seleziona un ristorante dalla lista che sta visualizzando ;
    \item L'utente visualizza le informazioni relative al ristorante :
    \begin{itemize}
        \item Il nome;
        \item il recapito telefonico;
        \item L'indirizzo;
        \item gli orari di servizio;
        \item la/le tipologie di cucine offerte;
        \item la valutazione media.
    \end{itemize}
    \item L'utente può scegliere di visualizzare il menù completo (vedi UC18.1-Visualizzazione menù);
    \item L'utente può scegliere di visualizzare le recensioni rilasciate da altri utenti (vedi UC18.2-Visualizzazione recensioni).
\end{enumerate}
\end{itemize}

\subsubsection{UC18.1-Visualizzazione menù}
\begin{itemize}
\item \textbf{Attore principale:} Utente non riconosciuto / cliente.
\item \textbf{Precondizioni:} L'utente sta visualizzando le informazioni di un ristorante.
\item \textbf{Postcondizioni:} L'utente visualizza il menù del ristorante.
\item \textbf{Scenario principale:}
\begin{enumerate}
    \item L'utente seleziona la funzionalità di visualizzazione del menù del ristorante;
    \item L'utente visualizza la lista completa delle pietanze presenti nel menù;
    \item L'utente può effettuare la ricerca di una pietanza (vedi UC18.1.1-Ricerca pietanza);
    \item L'utente può visualizzare i dettagli relativi ad una singola pietanza presente
     nel menù (vedi UC18.1.2-Visualizzazione pietanza).
\end{enumerate}
\end{itemize}

\textbf{UC18.1.1-Ricerca pietanza}
\begin{itemize}
\item \textbf{Attore principale:} Utente non riconosciuto / cliente.
\item \textbf{Precondizioni:} L'utente sta visualizzando il menù di un ristorante.
\item \textbf{Postcondizioni:} L'utente visualizza la lista delle pietanze corrispondenti ai parametri di ricerca.
\item \textbf{Scenario principale:}
\begin{enumerate}
    \item L'utente seleziona la funzionalità di ricerca di una pietanza ;
    \item L'utente può effettuare la ricerca per nome;
    \item L'utente può effettuare la ricerca selezionando gli allergeni che non vuole 
    siano presenti nelle pietanze del menù;
    \item Il sistema filtra la lista delle pietanze secondo i parametri inseriti;
    \item L'utente visualizza la lista delle pietanze corrispondenti ai criteri di ricerca
    (la lista eventualmente può essere vuota).
\end{enumerate}
\end{itemize}



\textbf{UC18.1.2-Visualizzazione pietanza}
\begin{itemize}
\item \textbf{Attore principale:} Utente non riconosciuto / cliente.
\item \textbf{Precondizioni:} L'utente sta visualizzando il menù di un ristorante.
\item \textbf{Postcondizioni:} L'utente visualizza le informazioni relative alla pietanza selezionata.
\item \textbf{Scenario principale:}
\begin{enumerate}
    \item L'utente seleziona una pietanza in particolare, presente nel menù che sta visualizzando;
    \item L'utente visualizza le seguenti informazioni ad essa relativa:
    \begin{itemize}
        \item La lista degli ingredienti in essa presenti;
        \item La lista degli allergeni;
        \item Il prezzo.
    \end{itemize}
\end{enumerate}
\end{itemize}

\subsubsection{UC18.2-Visualizzazione recensioni}
\begin{itemize}
\item \textbf{Attore principale:} Utente non riconosciuto / cliente.
\item \textbf{Precondizioni:} L'utente sta visualizzando le informazioni relative ad un ristorante.
\item \textbf{Postcondizioni:} L'utente visualizza le ultime 5 recensioni fatte in ordine temporale decrescente da altri utenti.
\item \textbf{Scenario principale:}
\begin{enumerate}
    \item L'utente seleziona la funzionalità di visualizzazione delle recensioni fatte da altri utenti
    sulla loro esperienza presso il ristorante;
    \item Di default,se presenti più di 5 recensioni, il sistema ne presenta una lista con le ultime 5 fatte in ordine temporale decrescente;
    \item L'utente visualizza le suddette recensioni ,corredate dalle seguenti informazioni:
    \begin{itemize}
        \item Un voto da 1 a 5 sul menù;
        \item Un voto da 1 a 5 sul servizio;
        \item Un voto da 1 a 5 sul prezzo;
        \item Un eventuale commento testuale.
    \end{itemize}
\end{enumerate}
\end{itemize}
