\textbf{UC1-Registrazione account}
\begin{itemize}
    \item \textbf{Attore principale: }utente non riconosciuto.
    \item \textbf{Precondizioni: }l'utente è connesso al sistema.
    \item \textbf{Postcondizioni: }l'account dell'utente e le informazioni ad esso collegate, sono registrati nel sistema.
    \item \textbf{Scenario principale:} 
        \begin{enumerate}
            \item L'utente sceglie l'opzione di registrazione di un account;
            \item Il sistema chiede all'utente di inserire una email ed una password;
            \item L'utente inserisce l'email e la password;
            \item L'utente conferma i valori inseriti;
            \item Il sistema comunica all'utente che la registrazione è andata 
            a buon fine.
            \item Il cliente visualizza un profilo di tipo cliente creato di default dal sistema,il cui username è l'email dell'account appena registrato.
        \end{enumerate}
    \item \textbf{Estensioni:}
        \begin{itemize}
                \item UCE1.1-Campo mancante;
                \item UC1.2-Email già registrata nel sistema.
        \end{itemize}
\end{itemize}
    
\textbf{UCE1.1-Campo mancante}
\begin{itemize}
    \item \textbf{Descrizione: }per effettuare la registrazione,devono essere inseriti un valore per la password che per l'email.
    \item \textbf{Scenario alternativo:}
    \begin{enumerate}
        \item Al momento della conferma,il sistema rileva che uno dei campi (od entrambi) risulta non compilato;
        \item Il sistema comunica la natura dell'errore all'utente;
        \item Il cliente visualizza i campi della password e della email;se è stato precedentemente inserito un valore in uno
        dei due campi,esso è presente.
    \end{enumerate}
\end{itemize}

\textbf{UC1.2-Email già registrata nel sistema}
\begin{itemize}
    \item \textbf{Descrizione: }l'utente non può registrare un account inserendo una email già usata da un altro utente.
    \item \textbf{Scenario alternativo:}
    \begin{itemize}
        \item L'email inserita dall'utente al momento della registrazione risulta già presente nel sistema;
        \item Il sistema comunica la natura dell'errore all'utente;
        \item Il cliente è reindirizzato alla homepage di registrazione.
    \end{itemize}
\end{itemize}
\break

\textbf{UC2-Login}
\begin{itemize}
\item \textbf{Attore principale:} utente non riconosciuto.
\item \textbf{Precondizioni:} l'utente è connesso al sistema.
\item \textbf{Postcondizioni:} l'utente è autenticato presso il sistema.
\item \textbf{Scenario principale:}
\begin{enumerate}
    \item L'utente sceglie l'opzione di autenticazione;
    \item L'utente inserisce la sua password;
    \item l'utente inserisce la sua email;
    \item L'utente conferma i dati inseriti al sistema;
    \item Il sistema verifica l'esistenza di un account con i suddetti dati;
    \item L'utente visualizza la lista dei profili afferenti al suo account.
\end{enumerate}
    \item \textbf{Estensione: }UCE2-Credenziali non corrette.
\end{itemize}

\textbf{UCE2-Credenziali non corrette}
\textbf{Descrizione: }il login può non andare a buon fine se l'email inserita dall'utente è errata
o ,nel caso sia presente un account registrato con la suddetta email, se la password inserita non è quella
ad essa associata; per questioni di sicurezza, all'utente viene notificato un errore generico.
\textbf{Scenario secondario:}
\begin{enumerate}
    \item Dopo la conferma dei dati inseriti dall'utente,il sistema verifica 
    che l' email non è presente a sistema;
    \item Il sistema comunica all'utente che le credenziali inserite non sono corrette;
    \item L'utente può riprovare ad effettuareil login ,inserendo di nuovo l'email e la password.
    email.
\end{enumerate}

\textbf{UC16-Visualizzazione lista ristoranti}
\begin{itemize}
\item \textbf{Attore principale:} utente non riconosciuto.
\item \textbf{Precondizioni:} l'utente è connesso al sistema.
\item \textbf{Postcondizioni:} l'utente visualizza una lista di ristoranti.
\item \textbf{Scenario principale:}
%Da aggiungere qualche step allo scenario principale?
\begin{enumerate}
    \item L'utente seleziona la funzionalità di visualizzazione di una lista di ristoranti;
    \item L'utente visualizza una lista di ristoranti,ordinati secondo la loro valutazione media.
\end{enumerate}
\end{itemize}

\textbf{UC17-Ricerca ristorante}
\begin{itemize}
\item \textbf{Attore principale:}utente non riconosciuto.
\item \textbf{Precondizioni:} l'utente è connesso al sistema.
\item \textbf{Postcondizioni:} l'utente visualizza la lista dei ristoranti corrispondenti ai criteri inseriti
dell'utente.
\item \textbf{Scenario principale:}
\begin{enumerate}
    \item L'utente seleziona la funzionalità di ricerca di un ristorante;
    \item L'utente può effettuare la ricerca inserendo uno o più parametri ,corrispondenti
    ai seguenti criteri:
    \begin{itemize}
        \item Il nome del ristorante (vedi UC17.1-Ricerca per nome);
        \item La città del ristorante (vedi UC17.2-Ricerca per città);
        \item La valutazione media del ristorante (vedi UC17.3-Ricerca per valutazione);
        \item La tipologia di cucina (vedi UC17.4-Ricerca per tipologia di cucina);
        \item L orario (vedi UC17.5-Ricerca ristorante per orario) ;
        \item La data (vedi UC17.6-Ricerca ristorante per data); 
    \end{itemize}
    \item L'utente conferma il o i valori inseriti ed effettua la ricerca;
    \item L'utente visualizza la lista dei ristoranti che rispettano i criteri da lui inseriti.
\end{enumerate}
\end{itemize}

\textbf{UC17.1-Ricerca ristorante per nome}
\begin{itemize}
\item \textbf{Attore principale:}utente non riconosciuto.
\item \textbf{Precondizioni:} l'utente è connesso al sistema.
\item \textbf{Postcondizioni:} l'utente visualizza la lista dei ristoranti corrispondenti 
alla ricerca per nome da lui inserito.
\item \textbf{Scenario principale:}
\begin{enumerate}
    \item L'utente seleziona la funzionalità di ricerca di un ristorante;
    \item L'utente inserisce il nome secondo il quale effettuare la ricerca;
    \item L'utente conferma il valore inserito ed effettua la ricerca;
    \item L'utente visualizza la lista dei ristoranti corrispondenti al nome da lui inserito.
\end{enumerate}
\end{itemize}

\textbf{UC17.2-Ricerca ristorante per città}
\begin{itemize}
\item \textbf{Attore principale:}utente non riconosciuto.
\item \textbf{Precondizioni:} l'utente è connesso al sistema.
\item \textbf{Postcondizioni:} l'utente visualizza la lista dei ristoranti corrispondenti alla città da lui inserita.
\item \textbf{Scenario principale:}
\begin{enumerate}
    \item L'utente seleziona la funzionalità di ricerca di un ristorante;
    \item L'utente inserisce la città come parametro di ricerca;
    \item L'utente dà la conferma ed effettua la ricerca;
    \item L'utente visualizza la lista dei ristoranti corrispondenti alla città inserita.
\end{enumerate}
\end{itemize}

\textbf{UC17.3-Ricerca ristorante per valutazione}
\begin{itemize}
\item \textbf{Attore principale:}utente non riconosciuto.
\item \textbf{Precondizioni:} l'utente è connesso al sistema.
\item \textbf{Postcondizioni:} l'utente visualizza la lista dei ristoranti la cui valutazione è maggiore o uguale alla
valutazione da lui inserita.
\item \textbf{Scenario principale:}
\begin{enumerate}
    \item L'utente seleziona la funzionalità di ricerca di un ristorante;
    \item L'utente inserisce il valore della valutazione che desidera;
    \item L'utente conferma il valore inserito ed effettua la ricerca;
    \item  L'utente visualizza la lista dei ristoranti con valutazione maggiore o uguale a quella da lui inserita.
\end{enumerate}
\end{itemize}

\textbf{UC17.4-Ricerca ristorante per tipologia di cucina}
\begin{itemize}
\item \textbf{Attore principale:}utente non riconosciuto.
\item \textbf{Precondizioni:} l'utente è connesso al sistema.
\item \textbf{Postcondizioni:} l'utente visualizza la lista dei ristoranti che offrono la tipologia di cucina
da lui inserita.
\item \textbf{Scenario principale:}
\begin{enumerate}
    \item L'utente seleziona la funzionalità di ricerca di un ristorante;
    \item L'utente seleziona la tipologia di cucina alla quale è interessato;
    \item L'utente conferma la scelta ed effettua la ricerca; 
    \item L'utente visualizza la lista dei ristoranti che offrono la cucina da egli cercata.
\end{enumerate}
\end{itemize}

\textbf{UC17.5-Ricerca ristorante per orario}
\begin{itemize}
\item \textbf{Attore principale:}utente non riconosciuto.
\item \textbf{Precondizioni:} l'utente è connesso al sistema.
\item \textbf{Postcondizioni:} l'utente visualizza la lista dei ristoranti che hanno posti disponibili nell'orario 
da lui scelto.
\item \textbf{Scenario principale:}
\begin{enumerate}
    \item L'utente seleziona la funzionalità di ricerca di un ristorante;
    \item L'utente seleziona l'orario ;
    \item L'utente dà la conferma ed effettua la ricerca;
    \item L'utente visualizza la lista dei ristoranti che hanno posti disponibili nell'orario da lui
    selezionato.
\end{enumerate}
\end{itemize}

\textbf{UC17.6-Ricerca ristorante per data}
\begin{itemize}
\item \textbf{Attore principale:}utente non riconosciuto.
\item \textbf{Precondizioni:} l'utente è connesso al sistema.
\item \textbf{Postcondizioni:} l'utente visualizza la lista dei ristoranti che hanno posti disponibili
nella data da lui selezionata.
\item \textbf{Scenario principale:}
\begin{enumerate}
    \item L'utente seleziona la funzionalità di ricerca di un ristorante;
    \item L'utente seleziona la data;
    \item L'utente dà la conferma ed effettua la ricerca;
    \item L'utente visualizza la lista dei ristoranti che hanno posti disponibili
    nella data selezionata .
\end{enumerate}
\end{itemize}
