\pagebreak
\textbf{UC33-Accettazione prenotazione}
\begin{itemize}
\item \textbf{Attore principale:} ristoratore.
\item \textbf{Precondizioni:} 
\begin{itemize}
        \item L'utente è connesso al sistema e sta visualizzando il dettaglio di una singola prenotazione (vedi UC36-Visualizzazione singola prenotazione);
        \item La prenotazione è nello stato "In attesa".
\end{itemize}
\item \textbf{Postcondizioni:} l'utente accetta la prenotazione.
\item \textbf{Scenario principale:}
\begin{enumerate}
    \item L'utente seleziona l'opzione di accettazione della prenotazione;
    \item Il sistema aggiorna lo stato della prenotazione.
\end{enumerate}
\end{itemize}

\textbf{UC34-Rifiuta prenotazione}
\begin{itemize}
\item \textbf{Attore principale:} ristoratore.
\item \textbf{Precondizioni:} 
\begin{itemize}
        \item L'utente è connesso al sistema e sta visualizzando il dettaglio di una singola prenotazione (vedi UC36-Visualizzazione singola prenotazione);
        \item La prenotazione è nello stato "In attesa".
\end{itemize}
\item \textbf{Postcondizioni:} l'utente rifiuta la prenotazione.
\item \textbf{Scenario principale:}
\begin{enumerate}
    \item L'utente seleziona l'opzione di rifiuto della prenotazione;
    \item L'utente inserisce opzionalmente le motivazioni del rifiuto della prenotazione;
    \item Il sistema aggiorna lo stato della prenotazione.
\end{enumerate}
\end{itemize}

\textbf{UC35-Visualizzazione lista prenotazioni}
\begin{itemize}
\item \textbf{Attore principale:} ristoratore.
\item \textbf{Precondizioni:} l'utente è connesso al sistema.
\item \textbf{Postcondizioni:} l'utente visualizza la lista delle prenotazioni (in qualsiasi stato esse si trovino) attraverso una vista a calendario, suddivise per giorno.
\item \textbf{Scenario principale:}
\begin{enumerate}
    \item Il sistema mostra una vista a calendario e per ogni giorno l'utente può consultare:
    \begin{itemize}
        \item La lista delle prenotazioni di quel particolare giorno;
        \item La lista degli ingredienti necessari per tale giorno (si veda UC35.1).
    \end{itemize}
    \item Il sistema mostra la lista delle prenotazioni inerente ad un giorno nei seguenti modi:
    \begin{itemize}
        \item Di default, vengono mostrate come prime le prenotazioni nello stato "Accettata";
        \item L'utente può inoltre applicare un filtro in base allo stato prenotazione.
    \end{itemize}
\end{enumerate}
\end{itemize}

\textbf{UC35.1-Visualizzazione dettagli ingredienti giornata}
\begin{itemize}
\item \textbf{Attore principale:} ristoratore.
\item \textbf{Precondizioni:} l'utente si trova nella sezione "Visualizzazione lista prenotazioni" (vedi UC35).
\item \textbf{Postcondizioni:} l'utente visualizza la lista degli ingredienti.
\item \textbf{Scenario principale:}
\begin{enumerate}
    \item Il sistema mostra la vista degli ingredienti relativi alla giornata, relativamente alle prenotazioni nello stato "Accettata";
    \item L'utente visualizza la lista degli ingredienti.
\end{enumerate}
\end{itemize}

\textbf{UC36-Visualizzazione singola prenotazione}
\begin{itemize}
    \item \textbf{Attore principale:} ristoratore.
    \item \textbf{Precondizioni:} 
    \begin{itemize}
        \item L'utente si trova nella sezione "Visualizzazione lista prenotazioni" (vedi UC35);
        \item L'utente seleziona una prenotazione da visualizzare in dettaglio.
    \end{itemize}
    \item \textbf{Postcondizioni:} l'utente visualizza le informazioni dettagliate della singola prenotazione.
    \item \textbf{Scenario principale:}
    \begin{enumerate}
        \item Il sistema mostra i dettagli e le informazioni relative alla prenotazione selezionata dall'utente:
        \begin{itemize}
            \item Nome e Cognome della persona che ha effettuato la prenotazione;
            \item Giorno e orario della prenotazione;
            \item Numero di persone;
            \item Stato della prenotazione.
        \end{itemize}
    \end{enumerate}
\end{itemize}

\textbf{UC36.1-Visualizzazione lista totale degli ingredienti}
\begin{itemize}
    \item \textbf{Attore principale:} ristoratore.
    \item \textbf{Precondizioni:} 
    \begin{itemize}
        \item L'utente si trova nella sezione "Visualizzazione singola prenotazione" (vedi UC36);
        \item L'utente seleziona "Lista ingredienti per questa prenotazione".
    \end{itemize}
    \item \textbf{Postcondizioni:} l'utente visualizza la lista degli ingredienti per la singola prenotazione.
    \item \textbf{Scenario principale:}
    \begin{enumerate}
        \item L'utente visualizza la lista degli ingredienti che servono per completare quella prenotazione, ogni ingrediente ha una sua quantità espressa in grammi.
    \end{enumerate}
\end{itemize}

\textbf{UC36.2-Visualizzazione lista ordinazioni attuali}
\begin{itemize}
    \item \textbf{Attore principale:} ristoratore.
    \item \textbf{Precondizioni:} 
    \begin{itemize}
        \item L'utente si trova nella sezione "Visualizzazione singola prenotazione" (vedi UC36);
        \item L'utente seleziona "Ordinazioni per questa prenotazione".
    \end{itemize}
    \item \textbf{Postcondizioni:} l'utente visualizza la lista delle ordinazioni per la singola prenotazione.
    \item \textbf{Scenario principale:}
    \begin{enumerate}
        \item L'utente visualizza la lista delle ordinazioni che sono state effettuate dai clienti collegati a questa prenotazione nei seguenti modi:
        \begin{itemize}
            \item Di default, vengono mostrate come prime le ordinazioni nello stato "Confermata";
            \item L'utente può inoltre applicare un filtro in base allo stato dell'ordinazione.
        \end{itemize}
    \end{enumerate}
\end{itemize}

\textbf{UC36.3-Visualizzazione lista Clienti}  % non sono sicuro di questo nome
\begin{itemize}
    \item \textbf{Attore principale:} ristoratore.
    \item \textbf{Precondizioni:} 
    \begin{itemize}
        \item L'utente si trova nella sezione "Visualizzazione singola prenotazione" (vedi UC36);
        \item L'utente seleziona "Lista clienti".
    \end{itemize}
    \item \textbf{Postcondizioni:} l'utente visualizza la lista dei clienti collegati alla singola prenotazione.
    \item \textbf{Scenario principale:}
    \begin{enumerate}
        \item L'utente visualizza la lista dei clienti collegati a questa prenotazione nei seguenti modi:
        \begin{itemize}
            \item Di default, vengono mostrati per primi i clienti le cui ordinazioni sono nello stato "Confermata";
            \item L'utente può inoltre applicare un filtro in base allo stato dell'ordinazione del cliente.
        \end{itemize}
    \end{enumerate}
\end{itemize}


\iffalse %commento
    \textbf{UC36.4-Visualizzazione lista Clienti}  % non sono sicuro di questo nome
\begin{itemize}
    \item \textbf{Attore principale:} ristoratore.
    \item \textbf{Precondizioni:} 
    \item \textbf{Postcondizioni:} l'utente visualizza la lista dei clienti collegati alla singola prenotazione.
    \item \textbf{Scenario principale:}
\end{itemize}

Non sono sicuro di questa perché a mio parere c'è molta somiglianza tra uc36.4 e uc40
\fi %fine commento


\textbf{UC37-Modifica menù}  
\begin{itemize}
    \item \textbf{Attore principale:} ristoratore.
    \item \textbf{Precondizioni:} l'utente ha effettuato l'accesso al sistema.
    \item \textbf{Postcondizioni:} l'utente gestisce il menù del proprio ristorante.
    \item \textbf{Scenario principale:}
    \begin{enumerate}
        \item L'utente può eseguire le seguenti operazioni all'interno della modifica del menù:
        \begin{itemize}
           \item Aggiungere una nuova pietanza al menù (si veda UC37.1);
           \item Modificare una pietanza preesistente (si veda UC37.2);
           \item Rimuovere una pietanza dal menù (si veda UC37.3);
           \item Aggiungere una nuova sezione al menù (si veda UC37.4);
           \item Modificare una sezione del menù (si veda UC37.5);
           \item Rimuovere una sezione dal menù (si veda UC37.6);
        \end{itemize}
    \end{enumerate}
\end{itemize}

\textbf{UC37.1-Nuova pietanza}  
\begin{itemize}
    \item \textbf{Attore principale:} ristoratore.
    \item \textbf{Precondizioni:} l'utente si trova nella sezione di gestione del menù (si veda UC37).
    \item \textbf{Postcondizioni:} l'utente ha inserito una nuova pietanza al menù.
    \item \textbf{Scenario principale:}
    \begin{enumerate}
        \item L'utente compila il seguente form contenente le informazioni relative alla pietanza:
        \begin{itemize}
            \item Il nome della pietanza;
            \item Seleziona gli ingredienti che compongono la pietanza dalla lista degli ingredienti, con la loro quantità espressa in grammi (si veda UC38 per la creazione di tale lista);
            \item Seleziona la sezione del menù a cui questo piatto appartiene.
        \end{itemize}
        \item Il sistema aggiorna il menù con la nuova pietanza inserita dall'utente.
    \end{enumerate}
\end{itemize}

\textbf{UC37.2-Modifica pietanza}  
\begin{itemize}
    \item \textbf{Attore principale:} ristoratore.
    \item \textbf{Precondizioni:} l'utente si trova nella sezione di gestione del menù (si veda UC37).
    \item \textbf{Postcondizioni:} l'utente ha modificato una pietanza al menù.
    \item \textbf{Scenario principale:}
    \begin{enumerate}
        \item L'utente modifica il seguente form contenente le informazioni relative alla pietanza:
        \begin{itemize}
            \item Il nome della pietanza;
            \item Seleziona gli ingredienti che compongono la pietanza dalla lista degli ingredienti, con la loro quantità espressa in grammi (si veda UC38 per la creazione di tale lista);
            \item Seleziona la sezione del menù a cui questo piatto appartiene.
        \end{itemize}
        \item Il sistema aggiorna il menù con le modifiche apportate alla pietanza dall'utente.
    \end{enumerate}
\end{itemize}
% aggiungere UCE 37.2 alcuni campi non sono stati inseriti?

\textbf{UC37.3-Rimuovi pietanza}  
\begin{itemize}
    \item \textbf{Attore principale:} ristoratore.
    \item \textbf{Precondizioni:} l'utente si trova nella sezione di gestione del menù (si veda UC37).
    \item \textbf{Postcondizioni:} l'utente ha rimosso una pietanza al menù.
    \item \textbf{Scenario principale:}
    \begin{enumerate}
        \item L'utente elimina una pietanza dal menù.
        \item Il sistema aggiorna il menù con l'eliminazione della pietanza.
    \end{enumerate}
\end{itemize}


\textbf{UC37.4-Nuova sezione menù}  
\begin{itemize}
    \item \textbf{Attore principale:} ristoratore.
    \item \textbf{Precondizioni:} l'utente si trova nella sezione di gestione del menù (si veda UC37).
    \item \textbf{Postcondizioni:} l'utente ha inserito una nuova sezione al menù.
    \item \textbf{Scenario principale:}
    \begin{enumerate}
        \item L'utente inserisce una nuova sezione al menù.
        \item Il sistema aggiorna il menù con la nuova seziona inserita dall'utente.
    \end{enumerate}
\end{itemize}

\textbf{UC37.5-Modifica sezione menù}  
\begin{itemize}
    \item \textbf{Attore principale:} ristoratore.
    \item \textbf{Precondizioni:} l'utente si trova nella sezione di gestione del menù (si veda UC37).
    \item \textbf{Postcondizioni:} l'utente ha modificato una sezione al menù.
    \item \textbf{Scenario principale:}
    \begin{enumerate}
        \item L'utente modifica una sezione del menù.
        \item Il sistema aggiorna il menù con la sezione modificata dall'utente.
    \end{enumerate}
\end{itemize}
% aggiungere UCE 37.5 alcuni campi non sono stati inseriti?

\textbf{UC37.6-Rimuovi sezione menù}  
\begin{itemize}
    \item \textbf{Attore principale:} ristoratore.
    \item \textbf{Precondizioni:} l'utente si trova nella sezione di gestione del menù (si veda UC37).
    \item \textbf{Postcondizioni:} l'utente ha rimosso una sezione del menù.
    \item \textbf{Scenario principale:}
    \begin{enumerate}
        \item L'utente elimina una sezione dal menù.
        \item Il sistema aggiorna il menù con l'eliminazione della sezione.
    \end{enumerate}
\end{itemize}

% aggiungere UC 37.6 sono presenti ancora piatti all'interno della sezione? 

\textbf{UC38-Modifica lista degli ingredienti}  
\begin{itemize}
    \item \textbf{Attore principale:} ristoratore.
    \item \textbf{Precondizioni:} l'utente ha effettuato l'accesso al sistema.
    \item \textbf{Postcondizioni:} l'utente gestisce la lista degli ingredienti del suo ristorante.
    \item \textbf{Scenario principale:}
    \begin{enumerate}
        \item L'utente può eseguire le seguenti operazioni all'interno della modifica della lista degli ingredienti:
        \begin{itemize}
           \item Aggiungere un nuovo ingrediente alla lista (si veda UC38.1);
           \item Rimuovere un ingrediente preesistente dalla lista (si veda UC38.2).
        \end{itemize}
    \end{enumerate}
\end{itemize}

\textbf{UC38.1-Nuovo ingrediente}  
\begin{itemize}
    \item \textbf{Attore principale:} ristoratore.
    \item \textbf{Precondizioni:} l'utente si trova nella sezione di gestione della lista ingredienti (si veda UC38).
    \item \textbf{Postcondizioni:} l'utente ha inserito un nuovo ingrediente nella lista.
    \item \textbf{Scenario principale:}
    \begin{enumerate}
        \item L'utente compila il seguente form contenente le informazioni relative alla pietanza:
        \begin{itemize}
            \item Il nome dell'ingrediente;
            \item Opzionalmente seleziona gli allergeni contenuti all'interno di quell'ingrediente (si veda UC38.1.1).
        \end{itemize}
        \item Il sistema aggiorna la lista degli ingredienti con il nuovo ingrediente inserito dall'utente.
    \end{enumerate}
\end{itemize}

\textbf{UC38.1-Nuovo ingrediente}  
\begin{itemize}
    \item \textbf{Attore principale:} ristoratore.
    \item \textbf{Precondizioni:} l'utente si trova nella sezione di gestione della lista ingredienti (si veda UC38).
    \item \textbf{Postcondizioni:} l'utente ha inserito un nuovo ingrediente nella lista.
    \item \textbf{Scenario principale:}
    \begin{enumerate}
        \item L'utente compila il seguente form contenente le informazioni relative alla pietanza:
        \begin{itemize}
            \item Il nome dell'ingrediente;
            \item Opzionalmente seleziona gli allergeni contenuti all'interno di quell'ingrediente (si veda UC38.1.1).
        \end{itemize}
        \item Il sistema aggiorna la lista degli ingredienti con il nuovo ingrediente inserito dall'utente.
    \end{enumerate}
\end{itemize}

\textbf{UC37.2-Rimuovi ingrediente}  
\begin{itemize}
    \item \textbf{Attore principale:} ristoratore.
    \item \textbf{Precondizioni:} l'utente si trova nella sezione di gestione della lista ingredienti (si veda UC38).
    \item \textbf{Postcondizioni:} l'utente ha rimosso un ingrediente dalla lista.
    \item \textbf{Scenario principale:}
    \begin{enumerate}
        \item L'utente elimina un ingrediente dalla lista di ingredienti.
        \item Il sistema aggiorna la lista con l'eliminazione dell'ingrediente.
    \end{enumerate}
\end{itemize}

%% inserire uc39 una volta che alex ha fatto l'uc18.2

\textbf{UC40-Gestione pagamento conto}  
\begin{itemize}
    \item \textbf{Attore principale:} ristoratore.
    \item \textbf{Precondizioni:} l'utente è connesso al sistema e si trova nella sezione di visualizzazione di una singola prenotazione (si veda UC36).
    \item \textbf{Postcondizioni:} l'utente può gestire il pagamento del conto.
    \item \textbf{Scenario principale:}
    \begin{enumerate}
        \item Il sistema mostra gli ordini o le persone che hanno pagato o non pagato, mettendo per prime quelle segnate come non pagate;
        \item L'utente può selezionare un ordine o una persona e segnarla come pagata;
        \item Il sistema aggiorna lo stato dell'ordine o della persona a "Pagato".
    \end{enumerate}
    \item \textbf{Generalizzazioni:}
        \begin{itemize}
            \item UC41-Pagamento equo;
            \item UC42-Pagamento proporzionale.
        \end{itemize}
\end{itemize}

\textbf{UC40-Gestione pagamento equo}  
\begin{itemize}
    \item \textbf{Descrizione:} pagamento equo del totale di tutti gli ordini, ovvero "alla romana".
    \item \textbf{Attore principale:} ristoratore.
    \item \textbf{Precondizioni:} l'utente è connesso al sistema e si trova nella sezione di visualizzazione di una singola prenotazione (si veda UC36).
    \item \textbf{Postcondizioni:} l'utente può gestire il pagamento equo del conto.
    \item \textbf{Scenario principale:}
    \begin{enumerate}
        \item Il sistema mostra le persone che hanno pagato o non pagato, mettendo per prime quelle che non hanno ancora pagato;
        \item L'utente può selezionare una persona e segnarla come pagata;
        \item Il sistema aggiorna lo stato della persona a "Pagato".
    \end{enumerate}
\end{itemize}

\textbf{UC41-Gestione pagamento proporzionale}  
\begin{itemize}
    \item \textbf{Descrizione:} pagamento proporzionale degli ordini, ovvero ognuno paga per quello che ha ordinato.
    \item \textbf{Attore principale:} ristoratore.
    \item \textbf{Precondizioni:} l'utente è connesso al sistema e si trova nella sezione di visualizzazione di una singola prenotazione (si veda UC36).
    \item \textbf{Postcondizioni:} l'utente può gestire il pagamento proporzionale del conto.
    \item \textbf{Scenario principale:}
    \begin{enumerate}
        \item Il sistema mostra gli ordini pagati o non pagati, mettendo per primi quelli ancora non pagati;
        \item L'utente può selezionare un ordina e segnarlo come pagato;
        \item Il sistema aggiorna lo stato dell'ordine a "Pagato".
    \end{enumerate}
\end{itemize}
