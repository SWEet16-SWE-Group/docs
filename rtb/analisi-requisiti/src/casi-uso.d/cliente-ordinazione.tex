\textbf{UC25-Creazione ordinazione}
\begin{itemize}
\item \textbf{Attore principale:} Cliente.
\item \textbf{Precondizioni:} 
\begin{itemize}
    \item Il cliente ha effettuato l'accesso al sistema;
    \item Il cliente ha effettuato una prenotazione (si veda UC19);
    \item Il cliente è nella sezione di riepilogo di una prenotazione singola (si veda UC22) e la prenotazione deve essere nello stato "Accettata" (si veda UC33).
\end{itemize}
\item \textbf{Postcondizioni:} Il cliente ha ordinato le pietanze relative alla prenotazione.
\item \textbf{Scenario principale:}
\begin{enumerate}
    \item Il cliente visualizza la sua ordinazione se già confermata, e le ordinazioni già effettuate e confermate dagli altri utenti facenti parte della prenotazione;
    \item Il cliente può aggiungere una pietanza al proprio ordine (si veda UC25.1);
    \item Il cliente può modificare una pietanza già aggiunta al prorio ordine (si veda UC25.2);
    \item Il cliente può confermare la sua ordinazione (si veda UC25.3);
\end{enumerate}
\end{itemize}

\textbf{UC25.1-Aggiungi pietanza/e}
\begin{itemize}
\item \textbf{Attore principale:} Cliente.
\item \textbf{Precondizioni:} Il cliente sta effettuando un'ordinazione (si veda UC25).
\item \textbf{Postcondizioni:} Il cliente ha aggiunto una o più pietanze alla propria ordinazione.
\item \textbf{Scenario principale:}
\begin{enumerate}
    \item Il cliente vede la lista delle pietanze del ristoratore presso il quale sta effettuando l'ordinazione;
    \item Il cliente può visualizzare il dettaglio di una singola pietanza (UC25.1.1);
    \item Il cliente può fare una ricerca tra le pietanze (si veda UC25.1.1);
    \item Il cliente dopo aver selezionato una o più pietanze conferma la selezione;
    \item Il sistema aggiorna le informazioni sull'ordinazione del cliente.
\end{enumerate}
\end{itemize}

\textbf{UC25.1.1-Visualizza pietanza singola}
\begin{itemize}
\item \textbf{Attore principale:} Cliente.
\item \textbf{Precondizioni:} il cliente sta visualizzando la lista delle pietanza (si veda UC25.1);
\item \textbf{Postcondizioni:} Il cliente visualizza il dettaglio di una singola pietanza del menù.
\item \textbf{Scenario principale:}
\begin{enumerate}
    \item Il cliente premere su una pietanza del menù;
    \item Il cliente visualizza il dettaglio di una pietanza ovvero le seguenti informazioni:
    \begin{itemize}
        \item Il nome della pietanza;
        \item Gli ingredienti che compongono la pietanza e la loro quantità;
        \item Gli eventuali allergeni contenuti nella pietanza;
        \item Il prezzo della pietanza.
    \end{itemize}
\end{enumerate}
\end{itemize}

\textbf{UC25.1.2-Ricerca pietanza}
\begin{itemize}
\item \textbf{Attore principale:} Cliente.
\item \textbf{Precondizioni:}  Il cliente sta visualizzando la lista delle pietanza (si veda UC25.1).
\item \textbf{Postcondizioni:} Il cliente visualizza la lista delle pietanze corrispondenti ai criteri da lui inseriti.
\item \textbf{Scenario principale:}
\begin{enumerate}
    \item Il cliente seleziona la funzionalità di ricerca di un ristorante;
    \item Il cliente può effettuare la ricerca inserendo uno o più parametri, corrispondenti ai seguenti criteri:
    \begin{itemize}
        \item Il nome della pietanza (si veda UC25.1.2.1);
        \item Per allergeni non contenuti all'interno della pietanza (si veda UC25.1.2.2).
    \end{itemize}
    \item Il sistema filtra la lista di pietanze secondo i criteri inseriti;
    \item Il cliente visualizza la lista di pietanze che rispettano i criteri da lui inseriti.
\end{enumerate}
\end{itemize}

\textbf{UC25.1.2.1-Ricerca pietanza per nome}
\begin{itemize}
\item \textbf{Attore principale:} Cliente.
\item \textbf{Precondizioni:}  Il cliente sta visualizzando la lista delle pietanza (si veda UC25.1).
\item \textbf{Postcondizioni:} Il cliente visualizza la lista delle pietanze corrispondenti alla ricerca per nome da lui inserito.
\item \textbf{Scenario principale:}
\begin{enumerate}
    \item Il cliente seleziona la funzionalità di ricerca di un ristorante;
    \item Il cliente inserisce il testo che deve essere contenuto nel nome;
    \item Il sistema filtra la lista di pietanze secondo il nome inserito;
    \item Il cliente visualizza la lista di pietanze corrispondenti al nome da lui inserito.
\end{enumerate}
\end{itemize}

\textbf{UC25.1.2.1-Ricerca pietanza per allergeni}
\begin{itemize}
\item \textbf{Attore principale:} Cliente.
\item \textbf{Precondizioni:}  Il cliente sta visualizzando la lista delle pietanza (si veda UC25.1).
\item \textbf{Postcondizioni:} Il cliente visualizza la lista delle pietanze non contenenti gli allergeni da lui selezionati.
\item \textbf{Scenario principale:}
\begin{enumerate}
    \item Il cliente seleziona la funzionalità di ricerca di un ristorante;
    \item Il cliente seleziona gli allergeni, di default sono selezionati gli allergeni da lui indicati in fase di creazione del profilo (si veda UC6);
    \item Il sistema filtra la lista di pietanze escludendo quelle che contengono gli allergeni selezionati;
    \item Il cliente visualizza la lista di pietanze non contenenti gli allergeni da lui selezionati.
\end{enumerate}
\end{itemize}

\textbf{UC25.2-Modifica pietanza ordinata}
\begin{itemize}
\item \textbf{Attore principale:} Cliente.
\item \textbf{Precondizioni:} 
\begin{itemize}
    \item Il cliente sta effettuando un'ordinazione (si veda UC25);
    \item Il cliente ha aggiunto una o più pietanza alla propria ordinazione (si veda UC25.1).
\end{itemize}
\item \textbf{Postcondizioni:} Il cliente ha modificato una pietanza.
\item \textbf{Scenario principale:}
\begin{enumerate}
    \item Il cliente vede la lista delle pietanze da lui ordinate (si veda UC25.1);
    \item Il cliente entra nel menù di modifica della pietanza;
    \item Il cliente può effettuare le seguenti operazioni di modifica:
    \begin{itemize}
        \item Modificare la quantità della pietanza ordinata (si veda UC25.2.1);
        \item Rimuovere la pietanza ordinata (si veda UC25.2.2);
        \item Aggiungere un ingrediente alla pietanza (si veda UC25.2.3);
        \item Rimuovere un ingrediente dalla pietanza (si veda UC25.2.4);
    \end{itemize}
    \item Il cliente dopo aver modificato la pietanza conferma le sue modifiche;
    \item Il sistema aggiorna le informazioni sull'ordinazione del cliente.
\end{enumerate}
\end{itemize}

\textbf{UC25.2.1-Modifica quantità pietanza}
\begin{itemize}
\item \textbf{Attore principale:} Cliente.
\item \textbf{Precondizioni:} 
\begin{itemize}
    \item Il cliente sta effettuando un'ordinazione (si veda UC25);
    \item Il cliente sta effettuando una modifica ad una pietanza ordinata (si veda UC25.2).
\end{itemize}
\item \textbf{Postcondizioni:} Il cliente ha modificato la quantità di una pietanza.
\item \textbf{Scenario principale:}
\begin{enumerate}
    \item Il cliente modifica la quantità della pietanza già ordinata, questo numero non può essere minore di 1.
    \item Il cliente dopo aver modificato la quantità della pietanza conferma le modifiche;
    \item Il sistema aggiorna le informazioni sull'ordinazione del cliente;
    \item Il cliente viene reindirizzato alla schermata di modifica pietanza ordinata (si veda UC25.2).
\end{enumerate}
\end{itemize}

\textbf{UC25.2.2-Rimuovi pietanza}
\begin{itemize}
\item \textbf{Attore principale:} Cliente.
\item \textbf{Precondizioni:} 
\begin{itemize}
    \item Il cliente sta effettuando un'ordinazione (si veda UC25);
    \item Il cliente sta effettuando una modifica ad una pietanza ordinata (si veda UC25.2).
\end{itemize}
\item \textbf{Postcondizioni:} Il cliente ha eliminato una pietanza.
\item \textbf{Scenario principale:}
\begin{enumerate}
    \item Il cliente seleziona l'opzione di eliminazione della pietanza;
    \item Il sistema aggiorna le informazioni sull'ordinazione del cliente;
    \item Il cliente viene reindirizzato alla schermata di modifica pietanza ordinata (si veda UC25.2).
\end{enumerate}
\end{itemize}

\textbf{UC25.2.3-Aggiungi ingrediente}
\begin{itemize}
\item \textbf{Attore principale:} Cliente.
\item \textbf{Precondizioni:} 
\begin{itemize}
    \item Il cliente sta effettuando un'ordinazione (si veda UC25);
    \item Il cliente sta effettuando una modifica ad una pietanza ordinata (si veda UC25.2).
\end{itemize}
\item \textbf{Postcondizioni:} Il cliente ha aggiunto uno o più ingredienti ad una pietanza.
\item \textbf{Scenario principale:}
\begin{enumerate}
    \item Il cliente seleziona l'opzione di aggiunta ingredienti della pietanza;
    \item Il cliente seleziona gli ingredienti che desidera aggiungere alla pietanza dalla lista definita dal ristoratore (si veda UC38);
    \item Il cliente conferma le aggiunte di ingredienti alla pietanza;
    \item Il sistema aggiorna le informazioni sull'ordinazione del cliente;
    \item Il cliente viene reindirizzato alla schermata di modifica pietanza ordinata (si veda UC25.2).
\end{enumerate}
\end{itemize}

\textbf{UC25.2.4-Rimuovi ingrediente}
\begin{itemize}
\item \textbf{Attore principale:} Cliente.
\item \textbf{Precondizioni:} 
\begin{itemize}
    \item Il cliente sta effettuando un'ordinazione (si veda UC25);
    \item Il cliente sta effettuando una modifica ad una pietanza ordinata (si veda UC25.2).
\end{itemize}
\item \textbf{Postcondizioni:} Il cliente ha rimosso uno o più ingredienti da una pietanza.
\item \textbf{Scenario principale:}
\begin{enumerate}
    \item Il cliente seleziona l'opzione di rimozione ingredienti dalla pietanza;
    \item Il cliente seleziona gli ingredienti che desidera rimuovere dalla pietanza;
    \item Il cliente conferma la rimozione degli ingredienti;
    \item Il sistema aggiorna le informazioni sull'ordinazione del cliente;
    \item Il cliente viene reindirizzato alla schermata di modifica pietanza ordinata (si veda UC25.2).
\end{enumerate}
\end{itemize}

\textbf{UC26-Annullamento ordinazione}
\begin{itemize}
\item \textbf{Attore principale:} Cliente.
\item \textbf{Precondizioni:} 
\begin{itemize}
    \item Il cliente ha effettuato l'accesso al sistema;
    \item Il cliente ha effettuato una prenotazione (si veda UC19);
    \item Il cliente è nella sezione di riepilogo di una prenotazione singola (si veda UC22) e la prenotazione deve essere nello stato "Accettata" (si veda UC33).
\end{itemize}
\item \textbf{Postcondizioni:} L'ordinazione selezionata è stata annullata.
\item \textbf{Scenario principale:}
\begin{enumerate}
    \item Il cliente ha ordinato un insieme di pietanze che non desidera più;
    \item Il cliente seleziona un'ordinazione;
    \item Il cliente seleziona l'opzione di annullamento dell'ordinazione;
    \item Il sistema aggiorna le informazioni sull'ordinazione del cliente;
\end{enumerate}
\end{itemize}
