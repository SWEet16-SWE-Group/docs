
\textbf{UC3 - Modifica account}
\begin{itemize}
    \item \textbf{Attore principale: }Utente autenticato;
    \item \textbf{Precondizioni: }L'utente ha effettuato il login o registrazione;
    \item \textbf{Postcondizioni: }L'account ha modificato con successo uno o più attributi dell'account;
    \item \textbf{Scenario principale:}
        \begin{enumerate}
            \item L'utente sceglie l'opzione di modificare l'account;
            \item Il sistema presenta all'utente un form con due campi precompilati, uno per la mail e uno per la password;
            \item L'utente modifica uno dei campi o entrambi;
            \item L'utente effettua il submit del form;
            \item Il sistema comunica all'utente che la modifica è andata a buon fine.
        \end{enumerate}
\end{itemize}

\textbf{UC4 - Logout}
\begin{itemize}
    \item \textbf{Attore principale: }Utente autenticato;
    \item \textbf{Precondizioni: }L'utente ha effettuato il login o registrazione;
    \item \textbf{Postcondizioni: }L'utente non è più riconosciuto dal sistema;
    \item \textbf{Scenario principale:}
        \begin{enumerate}
            \item L'utente sceglie l'opzione di logout;
            \item L'utente viene reindirizzato alla home del sito.
        \end{enumerate}
\end{itemize}

\textbf{UC5 - Creazione profilo}
\begin{itemize}
    \item \textbf{Attore principale: }Utente autenticato;
    \item \textbf{Precondizioni: }L'utente ha effettuato il login o registrazione;
    \item \textbf{Postcondizioni: }L'utente possiede un profilo con cui accedere alle funzionalità fulcro del sistema;
    \item \textbf{Scenario principale:}
        \begin{enumerate}
            \item L'utente sceglie l'opzione di creazione di un profilo;
            \item Il sistema presenta un form a seconda che il tipo di profilo scelto sia Cliente o Ristoratore;
            \item L'utente riempe il form;
            \item L'utente fa il submit del form;
            \item Il sistema valida la correttezza dei dati;
            \item L'utente ha la possibilità di selezionare il profilo generato.
        \end{enumerate}
\end{itemize}

\textbf{UC6 - Creazione profilo cliente}
\begin{itemize}
    \item \textbf{Attore principale: }Utente autenticato;
    \item \textbf{Precondizioni: }L'utente ha effettuato il login o registrazione;
    \item \textbf{Postcondizioni: }L'utente possiede un profilo con cui accedere alle funzionalità fulcro del sistema;
    \item \textbf{Scenario principale:}
        \begin{enumerate}
            \item L'utente sceglie l'opzione di creazione di un profilo;
            \item Il sistema presenta un form con 3 campi: username (obbligatorio), nome e cognome (facoltativi);
            \item L'utente riempe il form;
            \item L'utente fa il submit del form;
            \item Il sistema valida la correttezza dei dati;
            \item L'utente ha la possibilità di selezionare il profilo generato.
        \end{enumerate}
\end{itemize}

\textbf{UC7 - Creazione profilo ristoratore}
\begin{itemize}
    \item \textbf{Attore principale: }Utente autenticato;
    \item \textbf{Precondizioni: }L'utente ha effettuato il login o registrazione;
    \item \textbf{Postcondizioni: }L'utente possiede un profilo con cui accedere alle funzionalità fulcro del sistema;
    \item \textbf{Scenario principale:}
        \begin{enumerate}
            \item L'utente sceglie l'opzione di creazione di un profilo;
            \item Il sistema presenta un form con 3 campi obbligatori: nome del ristorante, recapito telefonico, indirizzo;
            \item L'utente riempe il form;
            \item L'utente fa il submit del form;
            \item Il sistema valida la correttezza dei dati;
            \item L'utente ha la possibilità di selezionare il profilo generato.
        \end{enumerate}
\end{itemize}

\textbf{UCE7 - Campi già presenti}
\begin{itemize}
    \item \textbf{Scenario alternativo:}
        \begin{enumerate}
            \item Il sistema rileva che almeno uno dei campi presenta un dato sono già registrato database;
            \item L'utente visualizza il messaggio di errore sopra al form con i campi precompilati con i valori
              inseriti precedentemente.
        \end{enumerate}
\end{itemize}
