\section{Casi d'uso}
\subsection{Scopo}
\subsection{Struttura dei casi d'uso}
\subsection{Attori}

\subsection{Lista dei casi d'uso}

\textbf{UC1-Registrazione account}
\begin{itemize}
    \item \textbf{Attore principale: }utente non registrato.
    \item \textbf{Precondizioni: }l'utente è connesso al sistema.
    \item \textbf{Postcondizioni: }l'account dell'utente e le informazioni ad esso collegate, sono registrati nel sistema.
    \item \textbf{Scenario principale:} 
        \begin{itemize}
            \item L'utente sceglie l'opzione di registrazione di un account;
            \item Il sistema presenta all'utente un form con due campi da compilare,uno per la password
            e uno per l'indirizzo di posta elettronica;
            \item L'utente compila i due campi;
            \item L'utente effettua il submit del form;
            \item Il sistema comunica all'utente che la registrazione è andata 
            a buon fine.
            \item Il cliente visualizza un profilo di tipo cliente creato di default dal sistema.
        \end{itemize}
    \item \textbf{Estensioni:}
        \begin{itemize}
                \item UCE1.1-Campo mancante;
                \item UC1.2-Email già registrata nel sistema.
        \end{itemize}
\end{itemize}
    
\textbf{UCE1.1-Campo mancante}
\begin{itemize}
    \item \textbf{Descrizione: }per effettuare la registrazione,devono essere inseriti un valore per la password che per l'email.
    \item \textbf{Scenario alternativo:}
    \begin{itemize}
        \item Al momento della conferma,il sistema rileva che uno dei campi (od entrambi) risulta non compilato;
        \item Il sistema comunica la natura dell'errore all'utente;
        \item Il cliente visualizza i campi della password e della email;se è stato precedentemente inserito un valore in uno
        dei due campi,esso è presente.
    \end{itemize}
\end{itemize}

\textbf{UC1.2-Email già registrata nel sistema}
\begin{itemize}
    \item \textbf{Descrizione: }l'utente non può registrare un account inserendo una email già usata da un altro utente.
    \item \textbf{Scenario alternativo:}
    \begin{itemize}
        \item L'email inserita dall'utente al momento della registrazione risulta già presente nel sistema;
        \item Il sistema comunica la natura dell'errore all'utente;
        \item Il cliente è reindirizzato alla homepage di registrazione.
    \end{itemize}
\end{itemize}
\break

\textbf{UC3-Modifica account }
\begin{itemize}
    \item \textbf{Attore principale:} utente autenticato.
    \item \textbf{Precondizioni:} l'utente è autenticato all'interno del sistema.
    \item \textbf{Postcondizioni:} le modifiche fatte alle informazioni dell'account dell'utente sono
    salvate dal sistema.
    \item \textbf{Scenario principale:}
        \begin{itemize}
            \item L'utente seleziona l'opzione di modifica dei dati del suo account;
            \item L'utente può scegliere se modificare sia la password che l'email o solo una
            di esse;
            \item L'utente inserisce la nuova email;
            \item L'utente inserisca la nuova password;
            \item L'utente conferma le modifiche fatte;
            \item Il sistema comunica all'utente che la modifica è avvenuta con successo.
        \end{itemize}
\end{itemize}

\textbf{UC4-Logout}
\begin{itemize}
    \item \textbf{Attore principale:} utente autenticato.
    \item \textbf{Precondizioni:} l'utente è autenticato presso il sistema.
    \item \textbf{Postcondizioni:} l'utente non è più autenticato all'interno del sistema.
    \item \textbf{Scenario principale:}
    \begin{itemize}
        \item L'utente seleziona l'opzione di logout;
        \item Il sistema chiede all'utente la conferma della scelta;
        \item L'utente conferma la scelta;
        \item Il sistema re-indirizza l'utente alla pagina di autenticazione.
    \end{itemize}
\end{itemize}

\textbf{UC5-Creazione profilo}
\begin{itemize}
    \item \textbf{Attore principale:} utente autenticato.
    \item \textbf{Precondizioni:} l'utente è connesso al sistema e sta visualizzando la lista dei suoi profili.
    \item \textbf{Postcondizioni:} il profilo creato dall'utente,insieme a tutte le relative informazioni,
    è salvato dal sistema nella lista dei profili dell'utente.
    \item \textbf{Scenario principale:}
    \begin{itemize}
        \item L'utente selezione l'opzione di creazione di un nuovo profilo;
        \item Il sistema chiede all'utente di scegliere se creare un profilo di tipo cliente
        o ristoratore;
        \item L'utente sceglie la tipologia di profilo;
        \item L'utente inserisce i dati necessari per la creazione del profilo scelto;
        \item L'utente sceglie l'opzione di conferma dei dati inseriti;
        \item Il sistema comunica all'utente che la creazione del profilo è avvenuta con successo;
        \item L'utente visualizza la lista dei suoi profili,ove è stato aggiunto il profilo appena creato.
    \end{itemize}
    \item \textbf{Generalizzazioni:}
        \begin{itemize}
            \item UC6-Creazione profilo cliente;
            \item UC7-Creazione profilo ristoratore.
        \end{itemize}
\end{itemize}

\textbf{UC6-Creazione profilo cliente}
\begin{itemize}
    \item \textbf{Attore principale:} utente autenticato.
    \item \textbf{Precondizioni:} l'utente è autenticato presso il sistema e sta visualizzando
    la lista dei suoi profili.
    \item \textbf{Postcondizioni:} il profilo "cliente" viene salvato dal sistema nella lista dei profili 
    dell'utente.
    \item \textbf{Scenario principale:}
    \begin{itemize}
        \item L'utente selezione l'opzione di creazione di un nuovo profilo;
        \item Il sistema chiede all'utente di scegliere se creare un profilo di tipo cliente
        o ristoratore;
        \item L'utente sceglie la tipologia "cliente";
        \item L'utente inserisce i seguenti dati:
        \begin{itemize}
            \item Il nome;
            \item Il cognome;
            \item Lo username;
            \item Le eventuali allergie ed intolleranze;
        \end{itemize}
        \item L'utente sceglie l'opzione di conferma dei dati inseriti;
        \item Il sistema comunica all'utente che la creazione del profilo è avvenuta con successo;
        \item L'utente visualizza la lista dei suoi profili,ove è stato aggiunto il profilo appena creato.
    \end{itemize}
\end{itemize}

\textbf{UC7-Creazione profilo ristoratore}
\begin{itemize}
    \item \textbf{Attore principale:} utente autenticato.
    \item \textbf{Precondizioni:} l'utente è autenticato presso il sistema e sta visualizzando la lista dei suoi profili.
    \item \textbf{Postcondizioni:} il profilo-ristoratore appena creato è salvato nella lista dei profili dell'utente.
    \item \textbf{Scenario principale:}
    \begin{itemize}
        \item L'utente selezione l'opzione di creazione di un nuovo profilo;
        \item Il sistema chiede all'utente di scegliere se creare un profilo di tipo cliente
        o ristoratore;
        \item L'utente sceglie la tipologia "ristoratore";
        \item L'utente inserisce i seguenti dati:
        \begin{itemize}
            \item il nome del ristorante;
            \item l'indirizzo;
            \item il recapito telefonico;
            \item il numero di coperti disponibili;
            \item l'elenco delle tipologie di cucine proposte.
        \end{itemize}
        \item L'utente sceglie l'opzione di conferma dei dati inseriti;
        \item Il sistema comunica all'utente che la creazione del profilo è avvenuta con successo;
        \item L'utente visualizza la lista dei suoi profili,ove è stato aggiunto il profilo appena creato.
    \end{itemize}
        \item \textbf{Estensioni:}
        \begin{itemize}
                \item UCE7.1-Campo mancante;
                \item UCE7.2-Recapito telefonico già presente;
                \item UCE7.3-Indirizzo già presente.
        \end{itemize}
\end{itemize}

\textbf{UCE7.1-Campo mancante}
\begin{itemize}
    \item \textbf{Descrizione: }al momento della conferma dei dati inseriti,nessun campo relativo al ristorante può essere vuoto.
    \item \textbf{Scenario alternativo:}
    \begin{itemize}
        \item Il sistema verifica che l'utente non ha inserito i valori relativi ad uno o più campi di
        compilazione;
        \item Il sistema comunica l'errore all'utente ,specificandone la natura;
        \item L'utente visualizza tutti i campi relativi alla creazione del profilo; sono presenti i dati inseriti preceden.
    \end{itemize}
\end{itemize}

\textbf{UCE7.2-Recapito telefonico già presente}
\begin{itemize}
    \item \textbf{Descrizione: }nel sistema non possono essere presenti ristoranti, afferenti allo stesso ristoratore
    o a diversi ristoratori, con lo stesso recapito telefonico.
    \item \textbf{Scenario alternativo:}
    \begin{itemize}
        \item Il sistema rileva che è già stato registrato un ristorante con lo stesso recapito telefonico
        inserito dall'utente;
        \item Il sistema comunica all'utente la necessità di modificare il recapito telefonico;
        \item L'utente visualizza i campi con i valori da lui precedentemente inseriti, escluso quello relativo al recapito telefonico
        essendo da ricompilare con un nuovo valore.
    \end{itemize}
\end{itemize}

\textbf{UCE7.3-Indirizzo già presente}
\begin{itemize}
    \item \textbf{Descrizione: }nel sistema non possono essere presenti ristoranti, afferenti allo stesso ristoratore
    o a diversi ristoratori, con lo stesso indirizzo.
    \item \textbf{Scenario alternativo:}
    \begin{itemize}
        \item Il sistema rileva che è già stato registrato un ristorante con lo stesso indirizzo
        inserito dall'utente;
        \item Il sistema comunica all'utente la necessità di modificare l'indirizzo;
        \item L'utente visualizza i campi con i valori da lui precedentemente inseriti, escluso quello relativo al'indirizzo
        essendo da ricompilare con un nuovo valore.
    \end{itemize}
\end{itemize}

\textbf{UC8-Cancellazione profilo}
\begin{itemize}
\item \textbf{Attore principale:} utente autenticato.
\item \textbf{Precondizioni:} l'utente è autenticato presso il sistema e sta visualizzando la lista dei suoi profili.
\item \textbf{Postcondizioni:} il profilo eliminato è rimosso dalla lista dei profili dell'utente.
\item \textbf{Scenario principale:}
\begin{itemize}
    \item L'utente selezione l'opzione di eliminazione di un profilo;
    \item Il sistema chiede all'utente di scegliere quale profilo eliminare;
    \item L'utente sceglie il profilo;
    \item L'utente sceglie l'opzione di conferma ;
    \item Il sistema comunica all'utente che la distruzione del profilo è avvenuta con successo;
    \item L'utente visualizza la lista dei suoi profili dalla quale è stato rimosso il profilo appena creato.
\end{itemize}
\end{itemize}

\textbf{UC9-Selezione profilo cliente}
\begin{itemize}
\item \textbf{Attore principale:} utente autenticato.
\item \textbf{Precondizioni:} l'utente è autenticato presso il sistema.
\item \textbf{Postcondizioni:} l'utente visualizza le informazioni relative al profilo cliente selezionato.
\item \textbf{Scenario principale:}
\begin{itemize}
    \item L'utente sta visualizzando la lista dei suoi profili;
    \item L'utente seleziona un profilo di tipo cliente;
    \item L'utente visualizza una lista casuale di ristoranti scelti dal sistema.
\end{itemize}
\end{itemize}

\textbf{UC10-Selezione profilo ristoratore}
\begin{itemize}
\item \textbf{Attore principale:} utente autenticato.
\item \textbf{Precondizioni:} l'utente è autenticato presso il sistema e possiede una profilo ristoratore.
\item \textbf{Postcondizioni:} l'utente visualizza le informazioni relative al profilo selezionato.
\item \textbf{Scenario principale:}
\begin{itemize}
    \item L'utente sta visualizzando la lista dei suoi profili;
    \item L'utente seleziona un profilo di tipo ristoratore;
    \item L'utente visualizza la dashboard relativa al ristorante del profilo selezionato.
\end{itemize}
\end{itemize}

\textbf{UC11-Modifica profilo }
\begin{itemize}
\item \textbf{Attore principale:} utente autenticato.
\item \textbf{Precondizioni:} l'utente è stato autenticato dal sistema e sta visualizzando la lista dei suoi profili.
\item \textbf{Postcondizioni:} le modifiche apportate ai dati relativi al profilo sono salvate dal sistema.
\item \textbf{Scenario principale:}
\begin{itemize}
    \item L'utente seleziona l'opzione di modifica del profilo;
    \item L'utente seleziona il profilo da modificare;
    \item L'utente visualizza i campi-dati modificabili del profilo;
    \item L'utente modifica uno o più dei campi-dati;
    \item L'utente conferma al sistema la o le modifiche effettuate;
    \item Il sistema comunica all'utente che la modifica è avvenuta con successo.
    \item L'utente visualizza la lista dei suoi profili.
\end{itemize}
\end{itemize}

%Nella descrizione dei casi d'uso di creazione e modifica dei profili cliente e ristoratore
%è ovviamente sempre presente la fase di compilazione dei vari campi, con anche relativi errori come
%recapito già presente ,ecc...che sia da scrivere un sottocaso d'uso "compilazione form" tipo comune a 
%casi d'uso veri e propri?
\textbf{UC-12: Modifica profilo cliente}
\begin{itemize}
\item \textbf{Attore principale:} 
\item \textbf{Precondizioni:} 
\item \textbf{Postcondizioni:} 
\item \textbf{Scenario principale:}
\begin{itemize}
    \item 
\end{itemize}
\end{itemize}

\textbf{UC-13: Modifica profilo ristoratore}
\begin{itemize}
\item \textbf{Attore principale:} 
\item \textbf{Precondizioni:} 
\item \textbf{Postcondizioni:} 
\item \textbf{Scenario principale:}
\begin{itemize}
    \item 
\end{itemize}
\end{itemize}

\textbf{UC14-Logout dal profilo}
\begin{itemize}
\item \textbf{Attore principale:} cliente/ristoratore.
\item \textbf{Precondizioni:} l'utente ha selezionato uno dei profili afferenti al suo account.
\item \textbf{Postcondizioni:} l'utente è indirizzato alla pagina di selezione del profilo.
\item \textbf{Scenario principale:}
\begin{itemize}
    \item L'utente seleziona l'opzione di logout dal profilo precedentemente selezionato;
    \item Il sistema chiede la conferma all'utente della volontà di tornare alla scelta dei profili;
    \item Il cliente conferma la scelta;
    \item Il cliente visualizza la lista dei profili del suo account, potendone selezionare uno.
\end{itemize}
\end{itemize}

\textbf{UC15-Login}
\begin{itemize}
\item \textbf{Attore principale:} utente non riconosciuto.
\item \textbf{Precondizioni:} l'utente è connesso al sistema.
\item \textbf{Postcondizioni:} l'utente è autenticato presso il sistema.
\item \textbf{Scenario principale:}
\begin{itemize}
    \item L'utente sceglie l'opzione di autenticazione;
    \item L'utente inserisce la sua password;
    \item l'utente inserisce la sua email;
    \item L'utente conferma i dati inseriti al sistema;
    \item Il sistema verifica l'esistenza di un account con i suddetti dati;
    \item L'utente visualizza la lista dei profili afferenti al suo account.
\end{itemize}
    \item \textbf{Estensione: }UCE14-Email o password non corretta.
\end{itemize}

\textbf{UCE14.1-Email o password non corretta}
\textbf{Scenario secondario:}
\begin{itemize}
    \item Dopo la conferma dei dati inseriti dall'utente,il sistema verifica 
    che l' email non è presente a sistema;
    \item Il sistema comunica l'errore all'utente,specificando l'incorrettezza dell'
    email.
\end{itemize}

\textbf{UC-Ricerca ristorante}
\begin{itemize}
\item \textbf{Attore principale:}utente non riconosciuto.
\item \textbf{Precondizioni:} l'utente è connesso al sistema.
\item \textbf{Postcondizioni:} l'utente visualizza la lista dei ristoranti corrispondenti ai criteri inseriti
dell'utente.
\item \textbf{Scenario principale:}
\begin{itemize}
    \item L'utente seleziona la funzionalità di ricerca di un ristorante;
    \item L'utente può effettuare la ricerca inserendo uno o più parametri ,corrispondenti
    ai seguenti criteri:
    \begin{itemize}
        \item Il nome del ristorante (vedi UC-Ricerca per nome);
        \item La città del ristorante (vedi UC-Ricerca per città);
        \item La valutazione media del ristorante (vedi UC-Ricerca per valutazione);
        \item La tipologia di cucina (vedi UC-Ricerca per tipologia di cucina);
        \item L orario ;
        \item Il giorno di calendario; 
    \end{itemize}
    \item L'utente conferma i valori inseriti ed effettua la ricerca;
    \item L'utente visualizza la lista dei ristoranti rispettano i criteri desiderati dall'utente.
\end{itemize}
\end{itemize}

\textbf{UC-Ricerca ristorante per nome}
\begin{itemize}
\item \textbf{Attore principale:}utente non riconosciuto.
\item \textbf{Precondizioni:} l'utente è connesso al sistema.
\item \textbf{Postcondizioni:} l'utente visualizza la lista dei ristoranti corrispondenti 
alla ricerca per nome effettuata dal cliente.
\item \textbf{Scenario principale:}
\begin{itemize}
    \item L'utente seleziona la funzionalità di ricerca di un ristorante;
    \item L'utente inserisce il nome secondo il quale effettuare la ricerca;
    \item L'utente visualizza la lista dei ristoranti corrispondenti al nome inserito dall'utente.
\end{itemize}
\end{itemize}

\textbf{UC-Ricerca ristorante per città}
\begin{itemize}
\item \textbf{Attore principale:}utente non riconosciuto.
\item \textbf{Precondizioni:} l'utente è connesso al sistema.
\item \textbf{Postcondizioni:} l'utente visualizza la lista dei ristoranti corrispondenti alla città inserita
dall'utente.
\item \textbf{Scenario principale:}
\begin{itemize}
    \item L'utente seleziona la funzionalità di ricerca di un ristorante;
    \item L'utente inserisce la città come parametro di ricerca;
    \item L'utente dà la conferma ed effettua la ricerca;
    \item L'utente visualizza la lista dei ristoranti corrispondenti alla città inserita.
\end{itemize}
\end{itemize}

\textbf{UC-Ricerca ristorante per valutazione}
\begin{itemize}
\item \textbf{Attore principale:}utente non riconosciuto.
\item \textbf{Precondizioni:} l'utente è connesso al sistema.
\item \textbf{Postcondizioni:} l'utente visualizza la lista dei ristoranti la cui valutazione è maggiore o uguale alla
valutazione inserita dall'utente.
\item \textbf{Scenario principale:}
\begin{itemize}
    \item L'utente seleziona la funzionalità di ricerca di un ristorante;
    \item L'utente inserisce il valore della valutazione che desidera;
    \item L'utente conferma il valore inserito ed effettua la ricerca;
    \item  L'utente visualizza la lista dei ristoranti con valutazione maggiore o uguale a quella da lui inserita.
\end{itemize}
\end{itemize}

\textbf{UC-Ricerca ristorante per tipologia di cucina}
\begin{itemize}
\item \textbf{Attore principale:}utente non riconosciuto.
\item \textbf{Precondizioni:} l'utente è connesso al sistema.
\item \textbf{Postcondizioni:} l'utente visualizza la lista dei ristoranti che offrono la tipologia di cucina
inserita dall'utente.
\item \textbf{Scenario principale:}
\begin{itemize}
    \item L'utente seleziona la funzionalità di ricerca di un ristorante;
    \item L'utente seleziona la tipologia di cucina alla quale è interessato;
    \item L'utente conferma la scelta ed effettua la ricerca; 
    \item L'utente visualizza la lista dei ristoranti che offrono la cucina da egli cercata.
\end{itemize}
\end{itemize}

\textbf{UC-Ricerca ristorante per orario}
\begin{itemize}
\item \textbf{Attore principale:}utente non riconosciuto.
\item \textbf{Precondizioni:} l'utente è connesso al sistema.
\item \textbf{Postcondizioni:} l'utente visualizza la lista dei ristoranti che hanno posti disponibili nell'orario inserito 
dall'utente.
\item \textbf{Scenario principale:}
\begin{itemize}
    \item L'utente seleziona la funzionalità di ricerca di un ristorante;
    \item L'utente seleziona l'orario ;
    \item L'utente dà la conferma ed effettua la ricerca;
    \item L'utente visualizza la lista dei ristoranti che hanno posti disponibili nell'orario da lui
    selezionato;
\end{itemize}
\end{itemize}

\textbf{UC-Ricerca ristorante per data}
\begin{itemize}
\item \textbf{Attore principale:}utente non riconosciuto.
\item \textbf{Precondizioni:} l'utente è connesso al sistema.
\item \textbf{Postcondizioni:} l'utente visualizza la lista dei ristoranti che hanno posti disponibili
nella data selezionata dall'utente.
\item \textbf{Scenario principale:}
\begin{itemize}
    \item L'utente seleziona la funzionalità di ricerca di un ristorante;
    \item L'utente seleziona la data;
    \item L'utente dà la conferma ed effettua la ricerca;
    \item L'utente visualizza la lista dei ristoranti che hanno posti disponibili
    nella data selezionata dall'utente.
\end{itemize}
\end{itemize}

\textbf{UC-Visualizzazione lista ristoranti}
\begin{itemize}
\item \textbf{Attore principale:}
\item \textbf{Precondizioni:} 
\item \textbf{Postcondizioni:} 
\item \textbf{Scenario secondario:}
\begin{itemize}
    \item 
\end{itemize}
\end{itemize}

\textbf{UC-Visualizzazione ristorante}
\begin{itemize}
\item \textbf{Attore principale:} utente non riconosciuto/cliente.
\item \textbf{Precondizioni:} l'utente è connesso al sistema ed ha selezionato un ristorante
dalla lista di ristoranti che stava precedentemente visualizzando.
\item \textbf{Postcondizioni:} l'utente visualizza le informazioni relative al ristorante.
\item \textbf{Scenario principale:}
\begin{itemize}
    \item L'utente visualizza le seguenti informazioni relative al ristorante:
    \begin{itemize}
        \item Il nome;
        \item L'indirizzo;
        \item Il recapito telefonico;
        \item L'orario di servizio;
    \end{itemize}
    \item L'utente può scegliere se visualizzare anche il menù (vedi UC-Visualizzazione menù);
    \item L'utente può scegliere se visualizzare le recensioni (vedi UC-Visualizzazione recensioni).
\end{itemize}
\end{itemize}

\textbf{UC-Visualizzazione menù}
\begin{itemize}
\item \textbf{Attore principale:} utente non riconosciuto.
\item \textbf{Precondizioni:} l'utente è connesso al sistema e sta visualizzando
le informazioni relative ad un ristorante.
\item \textbf{Postcondizioni:} l'utente visualizza il menù del ristorante.
\item \textbf{Scenario principale:}
\begin{itemize}
    \item L'utente seleziona l'opzione di visualizzazione del menù;
    \item L'utente visualizza,oltre alle informazioni del ristorante, anche il relativo menù.
\end{itemize}
\end{itemize}

\textbf{UC-Visualizzazione recensioni}
\begin{itemize}
\item \textbf{Attore principale:} utente non riconosciuto.
\item \textbf{Precondizioni:} l'utente è connesso al sistema e sta visualizzando
le informazioni relative ad un ristorante.
\item \textbf{Postcondizioni:} l'utente visualizza le recensioni rilasciate da altri utenti,relative al
ristorante.
\item \textbf{Scenario principale:}
\begin{itemize}
    \item L'utente seleziona l'opzione di visualizzazione delle recensioni;
    \item L'utente visualizza,oltre alle informazioni del ristorante, anche la lista delle ultime dieci recensioni 
    in ordine temporale, rilasciate da altri utenti;ogni recensione è composta da :
    \begin{itemize}
        \item Un voto da 1 a 5 relativo al servizio;
        \item Un voto da 1 a 5 relativo al prezzo;
        \item Un voto da 1 a 5 relativo al cibo;
        \item Un'eventuale commento testuale.
    \end{itemize}
    %l'utente può chiedere al sistema di vedere recensioni anche più vecchie?
\end{itemize}
\end{itemize}

\textbf{UC-}
\begin{itemize}
\item \textbf{Attore principale:}
\item \textbf{Precondizioni:} 
\item \textbf{Postcondizioni:} 
\item \textbf{Scenario secondario:}
\begin{itemize}
    \item 
\end{itemize}
\end{itemize}

