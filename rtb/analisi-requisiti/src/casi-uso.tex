\section{Casi d'uso}
\subsection{Scopo}

La presente sezione ha come obiettivo l'identificazione e la descrizione di tutti i casi d'uso individuati dall'analisi del gruppo sul capitolato proposto.
    
\subsection{Attori}
Come concordato con il proponente, la web app$^{G}$ deve essere utilizzata sia da utenti clienti che da utenti ristoratori, sono stati quindi identificati per il Sistema i seguenti attori:
\begin{itemize}
    \item \textbf{Utente non riconosciuto:} è un utente che non ha effettuato l'accesso al Sistema. Può essere sia un utente non registrato sia un utente registrato che non ha ancora effettuato l'accesso.\\
    Può ricercare specifici ristoranti e visualizzarli;
    \item \textbf{Utente autenticato:} è un utente che ha effettuato l'accesso al Sistema ma non ha ancora selezionato un profilo.\\ 
    Può creare, modificare o eliminare profili cliente o ristoratore;
    \item \textbf{Cliente:} è un utente che ha effettuato l'accesso al Sistema ed ha selezionato un profilo cliente.\\
    Può effettuare le operazioni di prenotazione ed ordinazione e le loro attività correlate;
    \item \textbf{Ristoratore:} è un utente che ha effettuato l'accesso al Sistema ed ha selezionato un profilo ristoratore.\\
    Può gestire il proprio ristorante e le prenotazioni ad esso associate.
\end{itemize}

% oltre ad aggiungere l'immagine che andrà a mostrare le relazioni tra gli attori (non ci sono relazioni...)
% a mio parere sarebbe utile fare una descrizioni più approfindita sui "profili", ovvero spiegare che un utente autenticato può avere più profili ecc.
\pagebreak
\subsection{Lista dei casi d'uso}

% registazione - login - navigazione del sito
\input{src/casi-uso.d/0-utente-non-riconosciuto.tex}
% profili
\input{src/casi-uso.d/1-utente-autenticato.tex}
% cliente
\input{src/casi-uso.d/2-cliente-prenotazione.tex}
\input{src/casi-uso.d/3-cliente-ordinazione.tex}

\subsubsection{UC27-Selezione modalità di divisione del conto}
\begin{figure}[h] \includegraphics[scale=1]{uc27.png} \end{figure}
\begin{itemize}
\item \textbf{Attore principale:} Cliente.
\item \textbf{Precondizioni:} Il cliente deve avere:
  \begin{itemize}
    \item Una prenotazione accettata dal ristoratore;
    \item Limite temporale di modifica degli ordini deve essere stato superato;
    \item Deve essere stata fatta almeno una ordinazione;
    \item Deve essere stata scelta la modalità di divisione del conto.
  \end{itemize}
\item \textbf{Postcondizioni:} Il conto viene diviso a seconda della modalità scelta.
\item \textbf{Scenario principale:}
\begin{enumerate}
    \item Il cliente clicca sulla selezione della modalità di divisione del conto;
    \item Il cliente clicca sceglie se pagamento equo o proporzionale;
    \item Il cliente viene reindirizzato alla visualizzazione della singola prenotazione.
\end{enumerate}
\end{itemize}

\subsubsection{UCE27-Errore selezione già effettuata}
\begin{itemize}
\item \textbf{Attore principale:} Cliente.
\item \textbf{Descrizione:} Uno solo dei clienti può effettuare la selezione della modalità di divisione del conto.
  il primo che preme, sceglie per tutti.
\item \textbf{Scenario alternativo:}
\begin{enumerate}
    \item Il cliente clicca sulla selezione della modalità di divisione del conto;
    \item Il sistema mostra un messaggio di errore;
    \item Il cliente viene reindirizzato visualizzazione del conto di ogni cliente.
\end{enumerate}
\end{itemize}

\subsubsection{UC28-Pagamento in app}
\begin{figure}[h] \includegraphics[scale=1]{uc28.png} \end{figure}
\begin{itemize}
\item \textbf{Attore principale:} Cliente.
\item \textbf{Attore secondario:} Processore di pagamenti.
\item \textbf{Precondizioni:} Il cliente deve avere:
  \begin{itemize}
    \item Una prenotazione accettata dal ristoratore;
    \item Limite temporale di modifica degli ordini deve essere stato superato;
    \item Deve essere stata fatta almeno una ordinazione;
    \item Deve essere stata scelta la modalità di divisione del conto.
  \end{itemize}
\item \textbf{Postcondizioni:} Il sistema segna il cliente / le ordinazioni come pagati.
\item \textbf{Scenario principale:}
\begin{enumerate}
    \item Il cliente seleziona la prenotazione che vuole pagare;
    \item Il cliente clicca sul tasto paga;
    \item Il sistema mostra un form con un menù a tendina chiedente il metodo di pagamento
      (Carta di credito, carta di debito, Paypal, etc);
    \item Il cliente clicca sul tipo desiderato;
    \item Il sistema crea un richiesta di pagamento tramite il metodo desiderato e
      apre una scheda dove il cliente può pagare;
    \item Il cliente paga;
    \item Il sistema riceve la conferma del pagamento;
    \item Il sitema reindirizza il cliente alla pagina di visualizzazione della singola prenotazione;
    \item Il cliente visualizza la pagina della singola prenotazione con ciò che ha pagato segnato come tale;
\end{enumerate}
    \item \textbf{Estensioni:}
        \begin{itemize}
                \item UCE28-Errore di pagamento.
        \end{itemize}
\end{itemize}

\subsubsection{UCE28-Errore di pagamento}
\begin{itemize}
\item \textbf{Descrizione: } Il processore di pagamento ha dato un errore.
\item \textbf{Scenario alternativo:}
\begin{enumerate}
    \item Al momento di ricezione della conferma viene ottenuto un errore;
    \item Il sistema comunica che il pagamento non è andato a buon fine;
    \item Il cliente visualizza il form della selezione del tipo di pagamento desiderato.
\end{enumerate}
\end{itemize}

\subsubsection{UC29-Inserimento di feedback e recensioni}
\begin{figure}[h] \includegraphics[scale=1]{uc29.png} \end{figure}
\begin{itemize}
\item \textbf{Attore principale:} Cliente.
\item \textbf{Precondizioni:}
  \begin{itemize}
    \item Il cliente sta visualizando una prenotazione accettata;
    \item La prenotazione è stata pagata.
  \end{itemize}
\item \textbf{Postcondizioni:} La recensione è stata salvata nella lista di recensioni del ristorante.
\item \textbf{Scenario principale:}
\begin{enumerate}
    \item Il cliente clicca sul tasto per lasciare una recensione;
    \item Il sistema mostra un form contente tre voti da una a cinque stelle e un campo di testo opzionale, rispettivamente per:
  \begin{itemize}
    \item Menù;
    \item Servizio;
    \item Prezzo.
  \end{itemize}
    \item Il cliente può riempire i tre voti (obbligatori) e il campo di testo (facoltativo);
    \item Il cliente fa il submit del form;
    \item Il sistema reindirizza alla pagina del ristorante dove è possibile vedere la propria recensione.
\end{enumerate}
\end{itemize}

\pagebreak
\subsubsection{UC30-Visualizzazione notifica cambio stato prenotazione}
\begin{figure}[h] \includegraphics[scale=1]{uc30.png} \end{figure}
\begin{itemize}
\item \textbf{Attore principale:} Cliente.
\item \textbf{Attore secondario:} Ristoratore.
\item \textbf{Precondizioni:} Il cliente ha una prenotazione in attesa di essere accettata.
\item \textbf{Postcondizioni:} Il cliente visualizza una notifica con il nuovo stato della prenotazione
\item \textbf{Scenario principale:}
\begin{enumerate}
    \item Il ristoratore accetta o rifiuta la prenotazione;
    \item Al cliente compare una notifica contenente il nuovo stato della prenotazione (accettata o rifiutata).
\end{enumerate}
\end{itemize}

\subsubsection{UC31-Visualizzazione notifica ricezione messaggio in chat}
\begin{figure}[h] \includegraphics[scale=1]{uc31.png} \end{figure}
\begin{itemize}
\item \textbf{Attore principale:} Cliente.
\item \textbf{Precondizioni:}
\begin{itemize}
    \item Il cliente ha avviato la chat con il ristoratore (si veda UC49);
    \item Il ristoratore ha inviato un messaggio al cliente (si veda UC50.1).
\end{itemize}
\item \textbf{Postcondizioni:} Il cliente visualizza una notifica relativa alla ricezione di un messaggio in chat.
\item \textbf{Scenario principale:}
\begin{enumerate}
    \item Il sistema vede che è stato inviato un messaggio dal ristoratore;
    \item Il sistema invia al cliente una notifica relativa al messaggio;
    \item Il cliente visualizza la notifica relativa al messaggio.
\end{enumerate}
\end{itemize}

\pagebreak
\subsubsection{UC32-Visualizzazione notifica accettazione invito da parte di altri utenti}
\begin{figure}[h] \includegraphics[scale=1]{uc32.png} \end{figure}
\begin{itemize}
\item \textbf{Attore principale:} Cliente.
\item \textbf{Precondizioni:} Il cliente ha una prenotazione accetata e ha condiviso il link di invito tramite terze parti.
\item \textbf{Postcondizioni:} Il cliente visualizza una notifica con il nome del profilo di chi ha accettato l'invito.
\item \textbf{Scenario principale:}
\begin{enumerate}
    \item L'utente che ha ricevuto il link di invito seleziona un profilo;
    \item Il sistema aggiunge il profilo selezionato alla prenotazione;
    \item Il sistema notifica il cliente dell'accettazione di invito;
    \item Al cliente mittente compare una notifica contentente il nuovo profilo aggiuntosi alla prenotazione.
\end{enumerate}
\end{itemize}

% ristoratore
\nonstopmode
\subsubsection{UC33-Accettazione prenotazione}
\begin{figure}[h] \includegraphics[scale=1]{uc33.png} \end{figure}
\begin{itemize}
\item \textbf{Attore principale:} Ristoratore;
\item \textbf{Precondizioni:}
\begin{itemize}
        \item Il ristoratore è connesso al sistema e sta visualizzando il dettaglio di una singola prenotazione (si veda UC36-Visualizzazione singola prenotazione);
        \item La prenotazione è nello stato "In attesa".
\end{itemize}
\item \textbf{Postcondizioni:} Lo stato della prenotazione diventa "Accettata";
\item \textbf{Scenario principale:}
\begin{enumerate}
    \item Il ristoratore seleziona l'opzione di accettazione della prenotazione;
    \item Il sistema aggiorna lo stato della prenotazione.
\end{enumerate}
\end{itemize}

\pagebreak
\subsubsection{UC34-Rifiuta prenotazione}
\begin{figure}[h] \includegraphics[scale=1]{uc34.png} \end{figure}
\begin{itemize}
\item \textbf{Attore principale:} Ristoratore;
\item \textbf{Precondizioni:}
\begin{itemize}
        \item Il ristoratore è connesso al sistema e sta visualizzando il dettaglio di una singola prenotazione (si veda UC36-Visualizzazione singola prenotazione);
        \item La prenotazione è nello stato "In attesa".
\end{itemize}
\item \textbf{Postcondizioni:} Lo stato della prenotazione diventa "Rifiutata";
\item \textbf{Scenario principale:}
\begin{enumerate}
    \item Il ristoratore seleziona l'opzione di rifiuto della prenotazione;
    \item Il ristoratore inserisce opzionalmente le motivazioni del rifiuto della prenotazione;
    \item Il sistema aggiorna lo stato della prenotazione.
\end{enumerate}
\end{itemize}

\subsubsection{UC35-Visualizzazione lista prenotazioni}
\begin{figure}[h] \includegraphics[scale=1]{uc35.png} \end{figure}
\begin{itemize}
\item \textbf{Attore principale:} Ristoratore;
\item \textbf{Precondizioni:} Il ristoratore è connesso al sistema;
\item \textbf{Postcondizioni:} Il ristoratore visualizza la lista delle prenotazioni (in qualsiasi stato esse si trovino) suddivise per giorno;
\item \textbf{Scenario principale:}
\begin{enumerate}
    \item Il sistema mostra le prenotazioni raggruppate per giorno;
    \item Per ogni giorno il ristoratore può consultare:
    \begin{itemize}
        \item La lista delle prenotazioni di quel particolare giorno;
        \item La lista degli ingredienti necessari per tale giorno (si veda UC35.1).
    \end{itemize}
    \item Il sistema mostra la lista delle prenotazioni inerente ad un giorno nei seguenti modi:
    \begin{itemize}
        \item Di default, vengono mostrate come prime le prenotazioni nello stato "Accettata";
        \item Il ristoratore può inoltre applicare un filtro in base allo stato della prenotazione.
    \end{itemize}
\end{enumerate}
\end{itemize}

\subsubsection{UC54-Visualizzazione dettagli ingredienti giornata}
\begin{figure}[h] \includegraphics[scale=1]{uc54.png} \end{figure}
\begin{itemize}
\item \textbf{Attore principale:} Ristoratore;
\item \textbf{Precondizioni:} Il ristoratore si trova nella propria dashboard;
\item \textbf{Postcondizioni:} Il ristoratore visualizza la lista degli ingredienti;
\item \textbf{Scenario principale:}
\begin{enumerate}
    \item Il ristoratore seleziona la funzionalità per vedere il dettaglio degli ingredienti necessari per la giornata selezionata da un calendario;
    \item Il sistema mostra la lista degli ingredienti relativi alla giornata, relativamente alle prenotazioni nello stato "Accettata";
    \item Il ristoratore visualizza la lista degli ingredienti.
\end{enumerate}
\end{itemize}

\subsubsection{UC36-Visualizzazione singola prenotazione}
\begin{figure}[h] \includegraphics[scale=1]{uc36.png} \end{figure}
\begin{itemize}
    \item \textbf{Attore principale:} Ristoratore;
    \item \textbf{Precondizioni:} Il ristoratore si trova nella sezione "Visualizzazione lista prenotazioni" (si veda UC35);
    \item \textbf{Postcondizioni:} Il ristoratore visualizza le informazioni dettagliate della singola prenotazione;
    \item \textbf{Scenario principale:}
    \begin{enumerate}
        \item Il ristoratore seleziona una prenotazione da visualizzare in dettaglio;
        \item Il sistema mostra i dettagli e le informazioni relative alla prenotazione selezionata dal ristoratore:
        \begin{itemize}
            \item Username del profilo che ha effettuato la prenotazione;
            \item Giorno e orario della prenotazione;
            \item Numero di persone;
            \item Stato della prenotazione;
            \item Lista totale degli ingredienti per la ordinazione (si veda UC36.1);
            \item Lista delle ordinazioni attuali, sia confermate che non (si veda UC36.2);
            \item Lista dei clienti facenti parte della prenotazione (si veda UC36.3).
        \end{itemize}
    \end{enumerate}
\end{itemize}

\textbf{UC36.1-Visualizzazione lista totale degli ingredienti}
\begin{itemize}
    \item \textbf{Attore principale:} Ristoratore;
    \item \textbf{Precondizioni:}
    \begin{itemize}
        \item Il ristoratore si trova nella sezione "Visualizzazione singola prenotazione" (si veda UC36);
        \item Il ristoratore seleziona "Lista ingredienti per questa prenotazione".
    \end{itemize}
    \item \textbf{Postcondizioni:} Il ristoratore visualizza la lista degli ingredienti per la singola prenotazione;
    \item \textbf{Scenario principale:}
    \begin{enumerate}
        \item Il ristoratore visualizza la lista degli ingredienti che servono per completare quella prenotazione, ogni ingrediente ha una sua quantità espressa in grammi.
    \end{enumerate}
\end{itemize}

\textbf{UC36.2-Visualizzazione lista ordinazioni attuali}
\begin{itemize}
    \item \textbf{Attore principale:} Ristoratore;
    \item \textbf{Precondizioni:} Il ristoratore si trova nella sezione "Visualizzazione singola prenotazione" (si veda UC36);
    \item \textbf{Postcondizioni:} Il ristoratore visualizza la lista delle ordinazioni per la singola prenotazione;
    \item \textbf{Scenario principale:}
    \begin{enumerate}
        \item Il ristoratore seleziona l'opzione "Ordinazioni per questa prenotazione";
        \item Il ristoratore visualizza la lista delle ordinazioni che sono state effettuate dai clienti collegati a questa prenotazione nei seguenti modi:
        \begin{itemize}
            \item Di default, vengono mostrate come prime le ordinazioni nello stato "Confermata";
            \item Il ristoratore può inoltre applicare un filtro in base allo stato dell'ordinazione.
        \end{itemize}
    \end{enumerate}
\end{itemize}

\pagebreak

\textbf{UC36.3-Visualizzazione lista Clienti}
\begin{itemize}
    \item \textbf{Attore principale:} Ristoratore;
    \item \textbf{Precondizioni:}
    \begin{itemize}
        \item Il ristoratore si trova nella sezione "Visualizzazione singola prenotazione" (si veda UC36);
        \item Il ristoratore seleziona "Lista clienti".
    \end{itemize}
    \item \textbf{Postcondizioni:} Il ristoratore visualizza la lista dei clienti collegati alla singola prenotazione;
    \item \textbf{Scenario principale:}
    \begin{enumerate}
        \item Il ristoratore visualizza la lista dei clienti collegati a questa prenotazione nei seguenti modi:
        \begin{itemize}
            \item Di default, vengono mostrati per primi i clienti le cui ordinazioni sono nello stato "Confermata";
            \item Il ristoratore può inoltre applicare un filtro in base allo stato dell'ordinazione del cliente.
        \end{itemize}
    \end{enumerate}
\end{itemize}

\subsubsection{UC37-Modifica menù}
\begin{figure}[h] \includegraphics[scale=1]{uc37.png} \end{figure}
\begin{itemize}
    \item \textbf{Attore principale:} Ristoratore;
    \item \textbf{Precondizioni:} Il ristoratore ha effettuato l'accesso al sistema;
    \item \textbf{Postcondizioni:} Le modifiche apportate al menù sono salvate;
    \item \textbf{Scenario principale:}
    \begin{enumerate}
        \item Il ristoratore visualizza le pietanze contenute nel menù del ristorante;
        \item Il ristoratore può eseguire le seguenti operazioni di modifica:
        \begin{itemize}
           \item Aggiungere una nuova pietanza al menù (si veda UC37.1);
           \item Modificare una pietanza preesistente (si veda UC37.2);
           \item Rimuovere una pietanza dal menù (si veda UC37.3);
           \item Aggiungere una nuova sezione al menù (si veda UC37.4);
           \item Modificare una sezione del menù (si veda UC37.5);
           \item Rimuovere una sezione dal menù (si veda UC37.6).
        \end{itemize}
        \item Il ristoratore dopo aver apportato modifiche al menù le conferma;
        \item Il sistema aggiorna le informazioni sul menù del ristorante.
    \end{enumerate}
\end{itemize}

\textbf{UC37.1-Nuova pietanza}
\begin{itemize}
    \item \textbf{Attore principale:} Ristoratore;
    \item \textbf{Precondizioni:} Il ristoratore si trova nella sezione di modifica del menù (si veda UC37);
    \item \textbf{Postcondizioni:} Il ristoratore ha inserito una nuova pietanza al menù;
    \item \textbf{Scenario principale:}
    \begin{enumerate}
        \item Il ristoratore seleziona l'opzione di creazione di una nuova pietanza;
        \item Il ristoratore inserisce le informazioni relative alla pietanza:
        \begin{itemize}
            \item Il nome della pietanza;
            \item Seleziona gli ingredienti che compongono la pietanza dalla lista degli ingredienti, con la loro quantità espressa in grammi (si veda UC38 per la creazione di tale lista);
            \item Seleziona la sezione del menù a cui questo piatto appartiene.
        \end{itemize}
        \item Il ristoratore conferma l'aggiunta di una nuova pietanza al menù;
        \item Il sistema aggiorna il menù con la nuova pietanza;
        \item Il ristoratore viene reindirizzato alla schermata di modifica menù (si veda UC37).
    \end{enumerate}
\end{itemize}

\textbf{UC37.2-Modifica pietanza}
\begin{itemize}
    \item \textbf{Attore principale:} Ristoratore;
    \item \textbf{Precondizioni:} Il ristoratore si trova nella sezione di modifica del menù (si veda UC37);
    \item \textbf{Postcondizioni:} Il ristoratore ha modificato una pietanza al menù;
    \item \textbf{Scenario principale:}
    \begin{enumerate}
        \item Il ristoratore seleziona l'opzione di modifica di una pietanza;
        \item Il ristoratore modifica le informazioni relative alla pietanza:
        \begin{itemize}
            \item Il nome della pietanza;
            \item Seleziona gli ingredienti che compongono la pietanza dalla lista degli ingredienti, con la loro quantità espressa in grammi (si veda UC38 per la creazione di tale lista);
            \item Seleziona la sezione del menù a cui questo piatto appartiene.
        \end{itemize}
        \item Il ristoratore conferma le modifiche della pietanza;
        \item Il sistema aggiorna il menù con le modifiche apportate alla pietanza;
        \item Il ristoratore viene reindirizzato alla schermata di modifica menù (si veda UC37).
    \end{enumerate}
\end{itemize}

\textbf{UC37.3-Rimuovi pietanza}
\begin{itemize}
    \item \textbf{Attore principale:} Ristoratore;
    \item \textbf{Precondizioni:} Il ristoratore si trova nella sezione di modifica del menù (si veda UC37);
    \item \textbf{Postcondizioni:} La pietanza non è più presente nel menù;
    \item \textbf{Scenario principale:}
    \begin{enumerate}
        \item Il ristoratore seleziona l'opzione di eliminazione di una pietanza;
        \item Il sistema aggiorna il menù con l'eliminazione della pietanza;
        \item Il ristoratore viene reindirizzato alla schermata di modifica menù (si veda UC37).
    \end{enumerate}
\end{itemize}


\textbf{UC37.4-Nuova sezione menù}
\begin{itemize}
    \item \textbf{Attore principale:} Ristoratore;
    \item \textbf{Precondizioni:} Il ristoratore si trova nella sezione di modifica del menù (si veda UC37);
    \item \textbf{Postcondizioni:} Il ristoratore ha inserito una nuova sezione al menù;
    \item \textbf{Scenario principale:}
    \begin{enumerate}
        \item Il ristoratore seleziona l'opzione di creazione di una sezione;
        \item Il ristoratore inserisce il nome della nuova sezione;
        \item Il ristoratore conferma l'aggiunta di una nuova sezione al menù;
        \item Il sistema aggiorna il menù con la nuova seziona inserita.
    \end{enumerate}
\end{itemize}

\textbf{UC37.5-Modifica sezione menù}
\begin{itemize}
    \item \textbf{Attore principale:} Ristoratore;
    \item \textbf{Precondizioni:} Il ristoratore si trova nella sezione di modifica del menù (si veda UC37);
    \item \textbf{Postcondizioni:} Il ristoratore ha modificato una sezione al menù;
    \item \textbf{Scenario principale:}
    \begin{enumerate}
        \item Il ristoratore seleziona l'opzione di modifica di una sezione;
        \item Il ristoratore modifica il nome della sezione;
        \item Il ristoratore conferma le modifiche apportate alla sezione;
        \item Il sistema aggiorna il menù con la sezione modificata dal ristoratore.
    \end{enumerate}
\end{itemize}


\textbf{UC37.6-Rimuovi sezione menù}
\begin{itemize}
    \item \textbf{Attore principale:} Ristoratore;
    \item \textbf{Precondizioni:} Il ristoratore si trova nella sezione di modifica del menù (si veda UC37);
    \item \textbf{Postcondizioni:} La sezione non è più presente nel menù;
    \item \textbf{Scenario principale:}
    \begin{enumerate}
        \item Il ristoratore seleziona l'opzione di eliminazione di una sezione;
        \item Il sistema aggiorna il menù con l'eliminazione della sezione.
    \end{enumerate}
\end{itemize}

\subsubsection{UC38-Modifica lista degli ingredienti}
\begin{figure}[h] \includegraphics[scale=1]{uc38.png} \end{figure}
\begin{itemize}
    \item \textbf{Attore principale:} Ristoratore;
    \item \textbf{Precondizioni:} Il ristoratore ha effettuato l'accesso al sistema;
    \item \textbf{Postcondizioni:} Le modifiche apportate sono state salvate;
    \item \textbf{Scenario principale:}
    \begin{enumerate}
        \item Il ristoratore visualizza la lista degli ingredienti del ristorante;
        \item Il ristoratore può eseguire le seguenti operazioni:
        \begin{itemize}
           \item Aggiungere un nuovo ingrediente alla lista (si veda UC38.1);
           \item Rimuovere un ingrediente presente nella lista (si veda UC38.2).
        \end{itemize}
    \item Il ristoratore conferma le modifiche apportate alla lista degli ingredienti;
    \item Il sistema aggiorna la lista degli ingredienti;
    \item Il ristoratore viene reindirizzato alla dashboard.
    \end{enumerate}
\end{itemize}

\pagebreak
\textbf{UC38.1-Nuovo ingrediente}
\begin{itemize}
    \item \textbf{Attore principale:} Ristoratore;
    \item \textbf{Precondizioni:} Il ristoratore si trova nella sezione di modifica della lista ingredienti (si veda UC38);
    \item \textbf{Postcondizioni:} Viene inserito un nuovo ingrediente nella lista degli ingredienti;
    \item \textbf{Scenario principale:}
    \begin{enumerate}
        \item Il ristoratore seleziona l'opzione di creazione di un nuovo ingrediente;
        \item Il ristoratore compila il seguente form contenente le informazioni relative alla pietanza:
        \begin{itemize}
            \item Il nome dell'ingrediente;
            \item Opzionalmente seleziona gli allergeni contenuti all'interno di quell'ingrediente da una lista fornita dal sistema.
        \end{itemize}
        \item Il ristoratore conferma la creazione di un nuovo ingrediente;
        \item Il sistema aggiorna la lista degli ingredienti con il nuovo ingrediente;
        \item Il ristoratore viene reindirizzato alla schermata di modifica della lista ingredienti (si veda UC38).
    \end{enumerate}
\end{itemize}


\textbf{UC38.2-Rimuovi ingrediente}
\begin{itemize}
    \item \textbf{Attore principale:} Ristoratore;
    \item \textbf{Precondizioni:} Il ristoratore si trova nella sezione di gestione della lista ingredienti (si veda UC38);
    \item \textbf{Postcondizioni:} L'ingrediente viene rimosso dalla lista degli ingredienti;
    \item \textbf{Scenario principale:}
    \begin{enumerate}
        \item Il ristoratore seleziona l'opzione di eliminazione un ingrediente;
        \item Il sistema aggiorna la lista con l'eliminazione dell'ingrediente;
        \item Il ristoratore viene reindirizzato alla schermata di modifica della lista ingredienti (si veda UC38).
    \end{enumerate}
\end{itemize}
\pagebreak
\subsubsection{UC39-Visualizzazione recensioni ristorante} % analizzare più in dettaglio (grazie al *****)
\begin{figure}[h] \includegraphics[scale=1]{uc39.png} \end{figure}
\begin{itemize}
\item \textbf{Attore principale:} Ristoratore;
\item \textbf{Precondizioni:} Il ristoratore si trova nella dashboard;
\item \textbf{Postcondizioni:} Il ristoratore visualizza le ultime 5 recensioni fatte in ordine temporale decrescente dai clienti;
\item \textbf{Scenario principale:}
\begin{enumerate}
    \item Il ristoratore seleziona la funzionalità di visualizzazione delle recensioni fatte dai clienti sulla loro esperienza presso il ristorante;
    \item Il sistema presenta la lista delle recensioni fatte in ordine temporale decrescente;
    \item Il ristoratore visualizza le seguenti informazioni per ogni recensione:
    \begin{itemize}
        \item Il nome del profilo che l'ha rilasciata;
        \item La timestamp di rilascio;
        \item Un voto da 1 a 5 sul menù;
        \item Un voto da 1 a 5 sul servizio;
        \item Un voto da 1 a 5 sul prezzo;
        \item Un eventuale commento testuale.
    \end{itemize}
\end{enumerate}
\end{itemize}

\subsubsection{UC40-Gestione pagamento conto}
\begin{figure}[h] \includegraphics[scale=.7]{uc40.png} \end{figure}
\begin{itemize}
    \item \textbf{Attore principale:} Ristoratore;
    \item \textbf{Precondizioni:}
    \begin{itemize}
        \item Il ristoratore è connesso al sistema e si trova nella sezione di visualizzazione di una singola prenotazione (si veda UC36);
        \item Un cliente ha selezionato la modalità di divisione del conto (si veda UC27).
    \end{itemize}
    \item \textbf{Postcondizioni:} Rispettando la modalità di divisione del conto, una parte o il totale del conto della prenotazione è segnato come pagato;
    \item \textbf{Scenario principale:}
    \begin{enumerate}
        \item Il sistema mostra gli ordini o le persone nello stato "Pagato" e "Non pagato", elencando per prime quelle nello stato "Non pagato";
        \item Il ristoratore può selezionare un ordine o una persona e segnarla come pagata;
        \item Il sistema aggiorna lo stato dell'ordine o della persona a "Pagato";
        \item Quando tutte le ordinazioni o persone hanno pagato allora la prenotazione va nello stato "Pagata".
    \end{enumerate}
    \item \textbf{Specializzazioni:}
        \begin{itemize}
            \item UC41-Pagamento equo;
            \item UC42-Pagamento proporzionale.
        \end{itemize}
\end{itemize}

\subsubsection{UC41-Gestione pagamento equo}
\begin{itemize}
    \item \textbf{Descrizione:} Pagamento equo del totale di tutti gli ordini, ovvero "alla romana";
    \item \textbf{Attore principale:} Ristoratore;
    \item \textbf{Precondizioni:}
    \begin{itemize}
        \item Il ristoratore è connesso al sistema e si trova nella sezione di visualizzazione di una singola prenotazione (si veda UC36);
        \item Un cliente ha selezionato la modalità di divisione del conto equa (si veda UC27).
    \end{itemize}
    \item \textbf{Postcondizioni:} Il o i profili selezionati sono segnati come pagati;
    \item \textbf{Scenario principale:}
    \begin{enumerate}
        \item Il sistema mostra le persone nello stato "Pagato" e "Non pagato", elencando per prime quelle nello stato "Non pagato";
        \item Il ristoratore può selezionare una persona o più persone e segnarle come pagate;
        \item Il sistema aggiorna lo stato delle persone a "Pagato";
        \item Quando tutte le persone hanno pagato allora la prenotazione va nello stato "Pagata".
    \end{enumerate}
\end{itemize}

\subsubsection{UC42-Gestione pagamento proporzionale}
\begin{itemize}
    \item \textbf{Descrizione:} Pagamento proporzionale delle ordinazioni, ovvero i clienti selezionano gli ordini che vogliono pagare (anche di altri clienti);
    \item \textbf{Attore principale:} Ristoratore:
    \begin{itemize}
        \item Il ristoratore è connesso al sistema e si trova nella sezione di visualizzazione di una singola prenotazione (si veda UC36);
        \item Un cliente ha selezionato la modalità di divisione del conto proporzionale (si veda UC27).
    \end{itemize}
    \item \textbf{Postcondizioni:} Una o più ordinazioni selezionate sono segnate come pagate;
    \item \textbf{Scenario principale:}
    \begin{enumerate}
        \item Il sistema mostra le ordinazioni nello stato "Pagato" e "Non pagato", elencando per prime quelle nello stato "Non pagato";
        \item Il ristoratore può selezionare una o più ordinazioni e segnarle come pagate;
        \item Il sistema aggiorna lo stato delle ordinazioni a "Pagato";
        \item Quando tutte le ordinazioni sono segnate come pagate allora la prenotazione va nello stato "Pagato".
    \end{enumerate}
\end{itemize}

\subsubsection{UC43-Visualizzazione notifica cliente ha confermato l'ordinazione}
\begin{figure}[h] \includegraphics[scale=1]{uc43.png} \end{figure}
\begin{itemize}
\item \textbf{Attore principale:} Ristoratore;
\item \textbf{Precondizioni:} Il cliente ha confermato l'ordinazione (si veda UC23.3);
\item \textbf{Postcondizioni:} Il ristoratore visualizza una notifica relativa alla conferma dell'ordinazione;
\item \textbf{Scenario principale:}
\begin{enumerate}
    \item Il sistema vede che è stata confermata un'ordinazione dal cliente;
    \item Il sistema invia al ristoratore una notifica relativa alla conferma dell'ordinazione;
    \item Il ristoratore visualizza la notifica relativa all'ordinazione.
\end{enumerate}
\end{itemize}

\subsubsection{UC44-Visualizzazione notifica cliente ha annullato la prenotazione}
\begin{figure}[h] \includegraphics[scale=1]{uc44.png} \end{figure}
\begin{itemize}
\item \textbf{Attore principale:} Ristoratore;
\item \textbf{Precondizioni:} Il cliente ha annullato la prenotazione (si veda UC21);
\item \textbf{Postcondizioni:} Il ristoratore visualizza una notifica relativa all'annullamento della prenotazione;
\item \textbf{Scenario principale:}
\begin{enumerate}
    \item Il sistema vede che è stata annullata una prenotazione dal cliente;
    \item Il sistema invia al ristoratore una notifica relativa all'annullamento della prenotazione;
    \item Il ristoratore visualizza la notifica relativa all'annullamento della prenotazione.
\end{enumerate}
\end{itemize}

\pagebreak
\subsubsection{UC45-Visualizzazione notifica nuova prenotazione in arrivo}
\begin{figure}[h] \includegraphics[scale=1]{uc45.png} \end{figure}
\begin{itemize}
\item \textbf{Attore principale:} Ristoratore;
\item \textbf{Precondizioni:} Il cliente ha effettuato una prenotazione (si veda UC19);
\item \textbf{Postcondizioni:} Il ristoratore visualizza una notifica relativa alla prenotazione di un tavolo;
\item \textbf{Scenario principale:}
\begin{enumerate}
    \item Il sistema vede che è stata effettuata una prenotazione dal cliente;
    \item Il sistema invia al ristoratore una notifica relativa ad una nuova prenotazione;
    \item Il ristoratore visualizza la notifica relativa alla nuova prenotazione.
\end{enumerate}
\end{itemize}

\subsubsection{UC46-Visualizzazione notifica nuova recensione}
\begin{figure}[h] \includegraphics[scale=1]{uc46.png} \end{figure}
\begin{itemize}
\item \textbf{Attore principale:} Ristoratore;
\item \textbf{Precondizioni:} Il cliente ha lasciato una recensione (si veda UC29);
\item \textbf{Postcondizioni:} Il ristoratore visualizza una notifica relativa alla recensione;
\item \textbf{Scenario principale:}
\begin{enumerate}
    \item Il sistema vede che è stata lasciata una recensione dal cliente;
    \item Il sistema invia al ristoratore una notifica relativa alla nuova recensione;
    \item Il ristoratore visualizza la notifica relativa alla nuova recensione.
\end{enumerate}
\end{itemize}

\pagebreak
\subsubsection{UC47-Visualizzazione notifica cliente ha pagato in app}
\begin{figure}[h] \includegraphics[scale=1]{uc47.png} \end{figure}
\begin{itemize}
\item \textbf{Attore principale:} Ristoratore;
\item \textbf{Precondizioni:} Il cliente ha effettuato un pagamento in app (si veda UC28);
\item \textbf{Postcondizioni:} Il ristoratore visualizza una notifica relativa al pagamento in app;
\item \textbf{Scenario principale:}
\begin{enumerate}
    \item Il sistema vede che è stata pagata un'ordinazione dal cliente;
    \item Il sistema invia al ristoratore una notifica relativa ad un pagamento;
    \item Il ristoratore visualizza la notifica relativa al pagamento.
\end{enumerate}
\end{itemize}

\subsubsection{UC48-Visualizzazione notifica ricezione messaggio in chat}
\begin{figure}[h] \includegraphics[scale=1]{uc48.png} \end{figure}
\begin{itemize}
\item \textbf{Attore principale:} Ristoratore;
\item \textbf{Precondizioni:}
\begin{itemize}
    \item Il cliente ha avviato la chat con il ristoratore (si veda UC49);
    \item Il cliente ha inviato un messaggio al ristoratore (si veda UC50.1).
\end{itemize}
\item \textbf{Postcondizioni:} Il ristoratore visualizza una notifica relativa alla ricezione di un messaggio in chat;
\item \textbf{Scenario principale:}
\begin{enumerate}
    \item Il sistema vede che è stato inviato un messaggio dal cliente;
    \item Il sistema invia al ristoratore una notifica relativa al messaggio;
    \item Il ristoratore visualizza la notifica relativa al messaggio.
\end{enumerate}
\end{itemize}



% chat
\pagebreak
\subsubsection{UC49-Avvia chat}
\begin{figure}[h] \includegraphics[scale=1]{uc49.png} \end{figure}
\begin{itemize}
\item \textbf{Attore principale:} Cliente.
\item \textbf{Precondizioni:} Il cliente è nella pagina di visualizzazione di un ristorante.
\item \textbf{Postcondizioni:} Diventa possibile comunicare con il ristoratore.
\item \textbf{Scenario principale:}
\begin{enumerate}
    \item Il cliente ha dei dubbi che vuole chiarire;
    \item Il cliente preme il bottone di avvio della chat;
    \item Il sistema crea la chat;
    \item Viene visualizzato un bottone di apertura della chat al posto dell'avvio chat;
    \item Diventa possibile comunicare bidirezionalmente tra il cliente e il ristoratore.
\end{enumerate}
\end{itemize}

% TODO aggiornare l'immagine
\subsubsection{UC50-Visualizzazione lista chat}
\begin{figure}[h] \includegraphics[scale=1]{uc50.png} \end{figure}
\begin{itemize}
\item \textbf{Attore principale:} Cliente / Ristoratore.
\item \textbf{Precondizioni:} L'utente sta visualizzando la propria dashboard.
\item \textbf{Postcondizioni:} L'utente visualizza le possibili chat con cui interagire.
\item \textbf{Scenario principale:}
\begin{enumerate}
    \item L'utente seleziona la funzionalità di visualizzazione della chat.
    \item L'utente visualizza l'elenco di chat avviate;
    \item Per ogni chat vengono visualizzate le seguenti informazioni:
      \begin{itemize}
        \item Il nome profilo dell'altra parte;
        \item La timestamp dell'ultimo messaggio inviato;
        \item I primi 20 caratteri dell'ultimo messaggio inviato.
      \end{itemize}
\end{enumerate}
\end{itemize}

% TODO aggiungere l'immagine
\subsubsection{UC51-Apertura chat}
\begin{figure}[h] \includegraphics[scale=1]{uc50.png} \end{figure}
\begin{itemize}
\item \textbf{Attore principale:} Cliente / Ristoratore.
\item \textbf{Precondizioni:} L'utente sta visualizzando la lista di chat interagibili.
\item \textbf{Postcondizioni:} L'utente visualizza la cronologia dei messaggi.
\item \textbf{Scenario principale:}
\begin{enumerate}
    \item L'utente seleziona chat con cui vuole interagire.
    \item L'utente visualizza la cronologia di messaggi inviati in ordine temporale decrescente;
    \item Per ogni mesaggio vengono visualizzate le seguenti informazioni:
      \begin{itemize}
        \item Il nome profilo dell'altra parte;
        \item La timestamp del messaggio;
        \item Il contenuto del messaggio.
      \end{itemize}
\end{enumerate}
\end{itemize}

% TODO aggiornare l'immagine
\textbf{UC52-Invio messaggio}
\begin{itemize}
\item \textbf{Attore principale:} Cliente / Ristoratore
\item \textbf{Precondizioni:} È stata aperta la chat.
\item \textbf{Postcondizioni:} È stato inviato il messaggio.
\item \textbf{Scenario principale:}
\begin{enumerate}
    \item L'attore principale scrive il messaggio nella barra di testo messa a disposizione;
    \item L'attore conferma l'invio;
    \item Il messaggio viene salvato nel database in forma criptata;
    \item Il destinatario riceve il messaggio.
\end{enumerate}
\end{itemize}

