\nonstopmode

\section{Introduzione}

\subsection{Scopo del documento}

Il presente documento ha lo scopo di illustrare le funzionalità del prodotto da sviluppare,
le quali sono state individuate, a partire dalla disamina del capitolato, tramite colloqui con
il proponente e tramite l'identificazione e l'analisi dei bisogni degli utenti finali.\\
Oltre alle funzionalità, nel documento sono illustrati i \emph{requisiti funzionali}$^{G}$, prestazionali, di
vincolo e di qualità che il gruppo ha individuato ed analizzato allo scopo di soddisfare le
funzionalità stesse offerte dal prodotto.

\subsection{Riferimenti}
\subsubsection{Riferimenti normativi}

\begin{itemize}
    \item Regolamento del progetto didattico: \\
    \url{https://www.math.unipd.it/~tullio/IS-1/2023/Dispense/PD2.pdf}
  \item \emph{Capitolato d’appalto}$^{G}$ C3 - Easy Meal: \\
    \url{https://www.math.unipd.it/~tullio/IS-1/2023/Progetto/C3.pdf}
    %\item Norme di progetto vs. 1.0.0 % aggiungere link a NdP una volta completate.
\end{itemize}

\subsubsection{Riferimenti informativi}

\begin{itemize}
    \item Analisi e descrizione delle funzionalità:Use Case e relativi diagrammi \url{https://www.math.unipd.it/~rcardin/swea/2022/Diagrammi%20Use%20Case.pdf};
    \item Kim Hamilton, Russ Miles \emph{Learning UML 2.0}, Addison Wesley;
    \item Martin Fowler, \emph{UMl Distilled}, Pearson Education;
    \item Glossario: \url{https://github.com/SWEet16-SWE-Group/docs/blob/main/RTB/Documentazione%20Esterna/Glossario.pdf}.
\end{itemize}

Tutti i riferimenti (normativi e informativi) a risorse web soggette a variazione sono stati consultati il 2024/03/25.
