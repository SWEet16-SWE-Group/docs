\section{Descrizione }
\subsection{Obiettivi del prodotto}
\subsection{Funzioni del prodotto}

Svariate sono le funzioni che il prodotto deve offrire agli utenti finali,
elencate di seuito ed accompagnate da un breve descrizione:

\begin{itemize}
    \item Creazione e gestione di un profilo: l'utente può creare e gestire (cioè modificare e/o 
    cancellare i dati ad esso relativi) uno o più profili di tipo cliente e/o ristoratore
    \item Ordinazione collaborativa di una o più pietanze: i clienti afferenti alla stessa prenotazione
    possono ordinare ognuno una o più pietanze ,le quali saranno visualizzate in un unico carrello 
    \item Suddivisione del conto: il cliente che ha effettuato la prenotazione può scegliere la modalità secondo la quale 
    dividere il conto prima del pagamento, con gli altri clienti afferenti alla sua stessa prenotazione
    \item Rilascio recensione : il cliente deve poter rilasciare una recensione riguardo la sua
    esperienza presso un ristorante
    \item Comunicazione con un ristoratore: il cliente può stabilire un canale di comunicazione con un
    ristoratore
    \item Inserimento e gestione di un ristorante: il ristoratore può inserire e gestire (modificare e/o cancellare)
    i dati relativi ad un o più ristoranti da lui amministrati 
    \item Gestione del menù: il ristoratore può gestire il menù del proprio ristorante,modificando le pietanze in esso 
    presenti,inserendone di nuove od eliminandole
    \item Comunicazione con il cliente: il ristoratore ,analogamente al cliente,può instaurare un canale di comunicazione
    con il cliente, mandandogli e ricevendo messaggi visualizzati all'interno di un'apposita chat
    \item Gestione delle prenotazioni: il ristoratore può gestire le richieste di prenotazioni (accettandole o rifiutandole), e 
    di quelle accettate, modificarne eventualmente i dati su richiesta del cliente afferente alla prenotazione stessa
    \item Gestione delle ordinazioni: il ristoratore può visualizzare le ordinazione dei clienti afferenti ad una
    una prenotazione ,essere notificato quando queste vengono effettuate ed eventualmente modificarle 
    \item Consultazione dello stato di pagamento: il ristoratore può visualizzare lo stato di pagamento del conto relativo ad una
    particolare prenotazione
    \item Sommario degli ingredienti: il ristoratore può visualizzare l'insieme degli ingredienti a lui ncessario per soddisfare 
    le ordinazioni delle prenotazioni  relative ad un periodo di uno o più giorni di calendario
\end{itemize}