\section{Requisiti}
In seguito allo studio del documento di capitolato,ai colloqui avuti con il proponente,alle sessioni di 
brainstorming fra i membri del gruppo e all'analisi degli casi d'uso,sono stati individuati molteplici requisiti 
che verranno di seguito elencati , suddivisi in due categorie :
\begin{itemize}
    \item Requisiti funzionali: sono strettamente legati alle funzionalità che il prodotto deve offrire agli utenti,
    al modo in cui il prodotto deve reagire a determinati input, agli output che deve produrre e al suo comportamento in determinate situazioni;
    \item Requisiti non funzionali: rappresentano i vincoli che il prodotto deve rispettare,legati alle caratteristiche tipiche di
    ogni prodotto software ,come ad es. la scalabilità,la sicurezza, la performance .
\end{itemize}
\subsection{Requisiti funzionali}
I requisiti di seguito elencati presentano un codice identificativo univoco,il quale indica se il relativo requisito è
obbligatorio (RFO), facoltativo (RFF), una breve descrizione e lo scenario d'uso di riferimento.
\begin{center}
    \begin{longtblr}{
        colspec={|X[1.5cm]|X[12cm]|X[2.5cm]|},
        row{odd}={bg=white},
        row{even}={bg=lightgray}
        }
     \hline
     \textbf{Codice} & \textbf{Descrizione} & \textbf{Fonte} \\
     \hline
     RFO1 & L'utente non riconosciuto deve poter registrare un nuovo account & UC1 \\ 
     \hline
     RFO2 & L'utente non riconosciuto deve visualizzare un mess
     aggio d'errore se la regitrazione non va a buon fine  & UCE1.1 e UCE1.2 \\  
     \hline
     RFO3 & L'utente non riconosciuto deve poter effettuare il login & UC2 \\   
     \hline
     RFO4 & L'utente non riconosciuto deve visualizzare un  messaggio
     se il login non va a buon fine & UCE21 \\
     \hline
     RFO5 & L'utente autenticato deve poter modificare le informazioni del
     suo account & UC14 \\
     \hline
     RFO6 & L'utente non riconosciuto deve poter effettuare il logout & UC4 \\
     \hline
     RFO7 & L'utente autenticato deve poter creare un profilo di tipo cliente & UC6\\
     \hline
     RFO8 & L'utente autenticato deve poter creare un profilo di tipo ristoratore & UC7\\
     \hline
     RFO9 & L'utente autenticato deve visualizzare un messaggio d'errore se la creazione del
     profilo di tipo ristoratore non va a buon fine & UCE7.1 e UCE7.2 \\
     \hline
     RFO10 & L'utente autenticato deve poter cancellare un profilo & UC8 \\
     \hline
     RFO11 & L'utente deve poter selezionare un qualunque profilo afferente al suo account & UC9 e UC10\\
     \hline
     RFO12 & Il cliente deve poter modificare le informazioni relative al suo profilo & UC12\\
     \hline
     RFO13 & Il ristoratore deve poter modificare le informazioni relative al suo profilo & UC13\\
     \hline
     RFO14 & Il cliente e il ristoratore devono poter uscire dal loro profilo & UC15 \\
     \hline 
     RF015 & L'utente non riconosciuto e il cliente devono poter visualizzare una lista di ristoranti & UC16 \\
     \hline
     RFO16 & L'utente non riconosciuto e il cliente devono poter ricercare un ristorante & UC17\\
     \hline 
     RF017 & L'utente non riconosciuto e il cliente devono poter applicare dei filtri alla ricerca di
     un ristorante & UC17\\
     \hline 
     RFO18 & L'utente non riconosciuto e il cliente devono poter visualizzare le informazioni relative
     ad un particolare ristorante, il suo menù e le relative recensioni & UC18 \\
     \hline
     RFO19 & Il cliente deve poter effettuare una prenotazione presso un ristorante & UC19 \\
     \hline
     RFO20 & Il cliente deve poter consultare la lista delle sue prenotazioni & UC20 \\
     \hline 
     RFO21 & Il cliente deve poter annullare una sua prenotazione & UC21 \\
     \hline
     RFO22 & Il cliente deve poter visualizzare le informazioni relative ad una sua
     prenotazione in particolare & UC22 \\
     \hline 
     RF023 & Il cliente deve poter invitare altri clienti a partecipare ad una sua prenotazione & UC23\\
     \hline
     RF024 & Il cliente deve poter accettare l'invito ricevuto , a partecipare ad una 
     prenotazione & UC24 \\
     \hline
     RFO25 & Il cliente deve poter effettuare un'ordinazione nel contesto di una sua prenotazione & UC25 \\
     \hline
     RFO26 & Nell'effettuare l'ordinazione ,il cliente deve poter apportare modifiche alle pietanze del ristorante,
     aggiungendo o rimuovendo ingredienti & UC25.2 \\
     \hline
     RFO27 & Il cliente deve poter annullare un'ordinazione & UC26 \\
     \hline
     RFO28 & Il cliente deve poter selezionare la modalità secondo la quale dividere il conto con
     gli altri clienti afferenti alla stessa prenotazione & UC27 \\
     \hline 
     RFO29 & Il cliente deve visualizzare un messaggio d'errore se la suddetta modalità è già stata scelta da un altro cliente afferente alla stessa prenotazione & UCE27 \\
     \hline 
     RFO30 & Il cliente deve poter effettuare il pagamento tramite l'app & UC28 \\
     \hline
     RFO31 & Il cliente deve visualizzare un messaggio d'errore nel caso il pagamento
     non sia andato a buon fine & UCE28 \\
     \hline
     RFO32 & Il cliente deve poter rilasciare feedback e una recensione relativi alla sua esperienza
     in un ristorante & UC29 \\
     \hline
     RFO33 & Il cliente deve ricevere una notifica quando una richiesta di prenotazione
     é accettata o rifiutata & UC30 \\
     \hline
     RFO34 & Il cliente e il ristoratore devono ricevere una notifica in concomitanza della 
     della ricezione di un messaggio & UC48 \\
     \hline 
     RFO35 & Il cliente deve ricevere una notifica quando un cliente da egli invitato a partecipare 
     ad una prenotazione, accetta l'invito & UC32\\ 
     \hline 
     RFO36 & Il ristoratore deve poter accettare la richiesta di prenotazione
     da parte di un cliente & UC33\\
     \hline
     RF037 & Il ristoratore deve poter rifiutare la richiesta di prenotazione 
     da parte di un cliente & UC34\\
     \hline 
     RFO38 & Il ristoratore deve poter visualizzare le prenotazioni presso il suo ristorante 
     & UC35 \\
     \hline
     RFO39 & Il ristoratore deve poter visualizzare i dettagli relativi ad una singola prenotazione & UC36\\
     \hline 
     RFO40 & Il ristoratore deve poter vedere gli ingredienti necessari per soddisfare gli ordinati
     afferenti ad una singola prenotazione & UC36.1\\
     \hline
     RFO41 & Il ristoratore deve poter vedere le ordinazioni effettuate dai clienti che partecipano alla stessa
     prenotazione & UC36.2\\
     \hline
     RFO42 & Il ristoratore deve poter visualizzare lo stato del pagamento del conto relativo ad una prenotazione
     cioè se e quante ordinazioni o quote rimangono da pagare & UC36.4\\
     \hline
     RFO43 & Il ristoratore deve poter modificare il menù di un suo ristorante aggiungengo,modificando e/o
      eliminando le pietanze e le sezioni del menù stesso & UC37\\
     \hline
     RFO44 & Il ristoratore deve poter modificare la lista degli ingredienti presenti nelle pietanze del menù,
     aggiungendone di nuovi od eliminandone di già presenti & UC38 \\
     \hline
     RFO45 & Il ristotore deve,in fase di aggiunta di un nuovo ingrediente, poter associare specifici allergeni 
     al suddetto ingrediente &  UC38.2.1\\
     \hline
     RFO46 & Il ristoratore deve poter visualizzare le recensioni relative al suo ristorante & UC39 \\
     \hline
     RFO47 & Il ristorante deve poter gestire manualmente il pagamento del conto relativo ad una 
     prenotazione & UC40\\
     \hline
     RFO48 & Il ristoratore deve ricevere una notifica quando un cliente conferma un'ordinazione & UC43 \\
     \hline
     RFO49 & Il ristoratore deve ricevere una notifica quando un cliente annulla una prenotazione & UC44 \\
     \hline
     RFO50 & Il ristoratore deve ricevere una notifica quando è effettuata una nuova richiesta di prenotazione
     presso il suo ristorante & UC45\\
     \hline 
     RFO51 & Il ristoratore deve ricevere una notifica quando un cliente rilascia una nuova recensione
     relativa ad un suo ristorante & UC46\\
     \hline 
     RFO52 & Il ristoratore deve ricevere una notifica quando un cliente effettua un pagamento
     tramite l'app & UC47\\
     \hline 
     RFO53 & Il ristoratore e il cliente devono poter visualizzare la/le loro chat & UC50\\
     \hline 
     RFO54 & Il cliente deve poter aprire un canale di comunicazione tramite chat verso un ristoratore
     & UC49 \\
    \hline
    \end{longtblr}
    \end{center}
\subsection{Requisiti non funzionali}
I requisiti non funzionali presentano anch'essi un codice identificativo univoco, che indica se il 
relativo requisito é obbligatorio (RNFO), desiderabile (RNFD) o facoltativo (RNOF); vengono inoltre indicati
esplicitamente la caratteristica del prodotto che il requisito tratta e la fonte d'ispirazione di quest'ultimo.
\begin{center}
    \begin{tblr}[X]{
        cols={|X[1.5cm]|X[1.5cm]|X[8cm]|X[2.5cm]|},
        row{odd}={bg=white},
        row{even}={bg=lightgray}
        }
        \hline 
        \textbf{Codice} \textbf{Caratteristica} & \textbf{Descrizione} & \textbf{Fonte}\\
        \hline 
        RNFO1 & Portabilità & Il prodotto deve essere una web resposive app , quindi poter essere eseguito su tutti i principali browser, quali Chrome, Firefox, Microsof Edge
        e Safari; deve offrire all'utente un'esperienza  coerente e piacevole su qualsiasi dispositivo, garantendo che il contenuto e le funzionalità siano accessibili
         e ben visualizzati indipendentemente dalle dimensioni dello schermo & Capitolato\\
         \hline 
         RNFO2 & Testabilità & Il prodotto deve presentare un design che faciliti il testing, essendo richiesta dal 
         proponente una copertura di test del codice maggiore o uguale all' 80 per cento  & Capitolato\\
         \hline 
         RNFD3 & Sicurezza & Le comunicazioni tra il cliente e il server devono essere crittografate, data la natura sensibile dei dati 
         inseriti dagli utenti & Capitolato\\
         \hline
         RNFO4 & Sicurezza & Viene implementato un sistema di autenticazione e autorizzazione per controllare l'accesso ai dati
         degli utenti tramite l'utilizzo di protocolli di autenticazione & Capitolato\\
         RNFD5 & Privacy & Viene richiesto all' utente un esplicito permesso per memorizzare i messaggi delle sue chat & Capitolato \\
         \hline 
         RNFF6 & Usabilità & L'app deve essere intuitiva e facile da usare , con un'interfaccia utente chiara e accessibile 
         che faciliti la navigazione e l'interazione; l'utente deve poterusufruire di tutte le funzionalità già al primo
         utilizzo & Brainstorming interno \\
         \hline
    \end{tblr}
\end{center}P

