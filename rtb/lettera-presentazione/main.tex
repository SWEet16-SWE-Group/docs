\documentclass[a4paper, 11pt]{article}
\usepackage{graphicx} % Required for inserting images
\usepackage{amsmath}
\usepackage{geometry}
\usepackage{hyperref}
\usepackage{setspace}
\usepackage{array}
\usepackage[usenames, dvipsnames]{xcolor}
\usepackage{colortbl}
\usepackage{tabularray}
\usepackage[italian]{babel}
\definecolor{darkgreen}{RGB}{18,94,40}
\definecolor{lightgreen}{RGB}{179,255,179}
\definecolor{moregreen}{RGB}{153,255,143}

 \geometry{
 a4paper,
 left=25mm,
 right=25mm,
 top=20mm,
 bottom=20mm,
}

\setlength{\parskip}{1em}
\setlength{\parindent}{0pt}
\graphicspath{{media/}{../media/}}

\setcounter{secnumdepth}{-2}

\begin{document}

\begin{minipage}{0.35\linewidth}
    \includegraphics[width=\linewidth]{Logo_Università_Padova.svg.png}
\end{minipage}\hfil
\begin{minipage}{0.55\linewidth}
\textbf{Università degli Studi di Padova} \\
Laurea in Informatica \\
Corso di Ingegneria del Software \\
Anno Accademico 2023/2024
\end{minipage}

\vspace{5mm}

\begin{minipage}{0.35\linewidth}
    \includegraphics[width=\linewidth]{logo rotondo.jpg}
\end{minipage}\hfil
\begin{minipage}{0.55\linewidth}
\textbf{Gruppo:} SWEet16 \\
\textbf{Email:}
\href{mailto:sweet16.unipd@gmail.com}{\nolinkurl{sweet16.unipd@gmail.com}}
\end{minipage}

\vspace{15mm}

\begin{center}
\begin{Huge}
        \textbf{Lettera di presentazione} \\
        \vspace{4mm}

\end{Huge}

\vspace{20mm}

\begin{large}
\begin{spacing}{1.4}
\begin{tabular}{c c c}
   Redattori: & Alex S. & \\
   Verificatori: & Alberto M., Iulius S. & \\
   Amministratore: & Alex S. & \\
   Destinatari: & T. Vardanega & R. Cardin \\
   Versione: & 1.0.0 &
\end{tabular}
\end{spacing}
\end{large}
\end{center}

\pagebreak

Egregi prof. Vardanega e Cardin\\
il gruppo SWEet16 è lieto di presentare la propria candidatura alla revisione di avanzamento Requirements and Technology Baseline (RTB), 
nell'impegno di consegnare il progetto da Voi commissionato, denominato "Easy Meal" (capitolato C3), proposto dall'azienda Imola Informatica.

\par

La documentazione relativa alla valutazione dei capitolati e verbali
interni ed esterni sono presenti al link:
\begin{center}
    \url{https://github.com/SWEet16-SWE-Group/docs/tree/main/RTB}
\end{center}
dove è possibile trovare:
\begin{itemize}
  \item Verbali interni ed esterni;
  \item Norme di progetto;
  \item Piano di progetto;
  \item Piano di qualifica;
  \item Analisi dei requisiti.
\end{itemize}

\begin{center}
\begin{table}[h]
\begin{tblr}{
colspec={X[5cm]X[5cm]},
row{odd}={bg=moregreen},
row{even}={bg=lightgreen},
row{1}={bg=darkgreen, fg=white}
}
    Nome & Matricola \\
    Alberto Cinel & 1142833 \\
    Bilal El Moutaren & 2053470 \\
    Alberto Michelazzo & 2010007 \\
    Alex Scantamburlo & 2042326 \\
    Iulius Signorelli & 2012434 \\
    Giovanni Zuliani & 595900 \\
\end{tblr}
\end{table}
\end{center}

\par

Numerose criticità sono emerse durante la fase di RTB, dovute ad un utilizzo delle risorse a disposizione del gruppo non sempre ottimale, efficace e/o efficiente; in particolare,
il gruppo non è riuscito a valorizzare la figura del Responsabile, fondamentale soprattutto nella suddivisione ed assegnazione dei compiti tra i vari membri. Il gruppo si è trovato
costretto a constatare tardivamente una deviazione sempre più importante dalla pianificazione originaria del progetto.
\par

Per questi motivi, la data di consegna del progetto è stata modificata. La nuova data di consegna
prevista è il 2024-06-15. Siamo certi che questo nuovo calendario ci permetterà di consegnare il
progetto nei tempi previsti senza comprometterne la qualità.

\par

Riteniamo che queste modifiche siano essenziali per garantire il successo del progetto e la Sua
soddisfazione finale.

\par

Cordiali saluti,\\
Il gruppo \emph{SWEet16}

\end{document}
