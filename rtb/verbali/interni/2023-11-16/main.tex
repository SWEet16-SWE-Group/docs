\documentclass[a4paper, 11pt]{article}
\usepackage{graphicx} % Required for inserting images
\usepackage{amsmath}
\usepackage{geometry}
\usepackage{hyperref}
\usepackage{setspace}
\usepackage{xcolor}
\usepackage{colortbl} 
\usepackage{tabularray}
\definecolor{darkgreen}{RGB}{18,94,40}
\definecolor{lightgreen}{RGB}{179,255,179}
\definecolor{moregreen}{RGB}{153,255,143}
 \geometry{
 a4paper,
 left=25mm,
 right=25mm,
 top=20mm,
 bottom=20mm,
 }

\setlength{\parskip}{1em}
\setlength{\parindent}{0pt}
\graphicspath{{../../../media/}}

\begin{document}

\begin{minipage}{0.35\linewidth}
    \includegraphics[width=\linewidth]{Logo_Università_Padova.svg.png}
\end{minipage}\hfil
\begin{minipage}{0.55\linewidth}
\textbf{Università degli Studi di Padova} \\
Laurea in Informatica \\
Corso di Ingegneria del Software \\
Anno Accademico 2023/2024
\end{minipage}

\vspace{5mm}

\begin{minipage}{0.35\linewidth}
    \includegraphics[width=\linewidth]{logo rotondo.jpg}
\end{minipage}\hfil
\begin{minipage}{0.55\linewidth}
\textbf{Gruppo:} SWEet16 \\
\textbf{Email:} 
\href{mailto:sweet16.unipd@gmail.com}{\nolinkurl{sweet16.unipd@gmail.com}}
\end{minipage}

\vspace{15mm}

\begin{center}
\begin{Huge}
        \textbf{Verbale Interno} \\
        \vspace{4mm}
        \textbf{16 Novembre 2023}
\end{Huge}

\vspace{20mm}

\begin{large}
\begin{spacing}{1.4}
\begin{tabular}{c c c}
   Redattori:  &  Alberto M. & \\
   Verificatori: & Alberto C.& \\
   Amministratore: &  Alberto C. & \\
   Destinatari: & T. Vardanega & R. Cardin \\  
   Versione: & 1.0.0 & 
\end{tabular}
\end{spacing}
\end{large}
\end{center}

\pagebreak


\begin{huge}
    \textbf{Registro delle modifiche}
\end{huge}
\vspace{5pt}

\begin{tblr}{
colspec={|X[1.5cm]|X[2cm]|X[2.5cm]|X[2.5cm]|X[5cm]|},
row{odd}={bg=white},
row{even}={bg=lightgray},
row{1}={bg=black,fg=white}
}
    Versione & Data & Autore & Verificatore & Descrizione \\
    \hline
    1.0.0 &  2023/11/20 & Alberto M. & & Approvazione per rilascio \\
    \hline 
     0.1.0 & 2023/11/19 & Alberto M. & Alberto C. & Prima stesura del documento \\
     \hline
\end{tblr}

\pagebreak

\section{Partecipanti}
Di seguito i nomi dei partecipanti con le rispettive matricole: \\
\vspace{5mm}

\begin{table}[h]
\begin{tblr}{
colspec={X[5cm]X[5cm]},
row{odd}={bg=moregreen},
row{even}={bg=lightgreen},
row{1}={bg=darkgreen,fg=white}
}
    Nome & Matricola \\
    Alberto Cinel & 1142833 \\
    Bilal El Moutaren & 2053470 \\
    Alberto Michelazzo & 2010007 \\
    Alex Scantamburlo & 2042326 \\
    Iulius Signorelli & 2012434 \\
    Giovanni Zuliani & 595900 
\end{tblr}
\end{table}

\vspace{10pt}

\textbf{Inizio incontro:} Ore 21:00 \newline
\textbf{Fine incontro:} Ore 22:30  \newline

\pagebreak

\section{Sintesi ed elaborazione incontro}

Il primo punto di questo incontro è stato riguardo l'allineamento delle conoscenze per l'utilizzo di GitHub come ITS: la maggior parte del gruppo ha preso una buona confidenza con le tecnologie, anche se qualcuno ha ancora qualche dubbio sul loro utilizzo.

Proseguendo si è discusso della lezione rovesciata di oggi (16/11/23), cercando di trarne spunti per poter lavorare meglio da qui in avanti.

Parte centrale di questo incontro è stata la scelta per l'implementazione del PoC, è stata fatta una discussione se utilizzarlo come base per l'MVP oppure se solo come un prodotto "usa e getta". \\
Dopo aver ascoltato i pareri di tutti è stato deciso di andare a creare un prodotto che servirà soprattuto a noi per prendere confidenza con l'argomento del progetto e le tecnologie.

E' inoltre stata fatta una discussione preliminare riguardante le funzionalità da implementare nel PoC, discussione che verrà ampliata anche con la proponente.

Per quanto riguarda la stesura dei documenti si è deciso di andare ad utilizzare Google Docs per la stesura collaborativa, e solo in una fase successiva "tradurli" in LaTeX. \\
E' stato deciso di cominciare a scrivere le Norme di Progetto, per riuscire a dare degli standard per la stesura dei documenti, e per mettere in forma scritta il nostro Way of Working.

L'incontro si è concluso con la programmazione di un incontro interno per martedì 21.

\end{document}



