\section{Descrizione}
\subsection{Obiettivi del prodotto}

Il prodotto, la cui idea nasce nel contesto del settore della ristorazione,
 consiste in una responsive \emph{webapp}$^{G}$ e si pone molteplici obiettivi:
\begin{itemize}
  \item Offire ai \emph{clienti}$^{G}$ la possibilità di effettuare la \emph{prenotazione}$^{G}$ e l'\emph{ordinazione}$^{G}$ presso un ristorante in maniera intuitiva e divertente, potendo collaborare fra di loro e personalizzare le pietanze;
    \item Supportare il \emph{ristoratore}$^{G}$ nello svolgimento delle propria attività, nello specifico nella gestione delle prenotazioni e delle ordinazioni;
    \item Facilitare la comunicazione tra il cliente e il ristoratore;
    \item Permettere al cliente di recensire la propria esperienza e contemporaneamente al ristoratore di avere feedback sulla propria attività.
\end{itemize}

\subsection{Funzioni del prodotto}

Svariate sono le funzioni che il prodotto deve offrire agli utenti finali,
elencate di seguito ed accompagnate da una breve descrizione:

\begin{itemize}
    \item \textbf{Creazione e gestione di un profilo:} L'utente può creare e gestire (cioè modificare e/o
      cancellare i dati ad esso relativi) uno o più \emph{profili}$^{G}$ di tipo cliente e/o ristoratore;
    \item \textbf{Ordinazione collaborativa di una o più pietanze:} I clienti afferenti alla stessa prenotazione
      possono ordinare ognuno una o più pietanze, le quali saranno visualizzate in un unico \emph{carrello}$^{G}$;
    \item \textbf{Suddivisione del conto:} Il cliente che ha effettuato la prenotazione può scegliere la modalità secondo la quale
      dividere il \emph{conto}$^{G}$ prima del pagamento, con gli altri clienti afferenti alla sua stessa prenotazione;
    \item \textbf{Rilascio recensione:} Il cliente deve poter rilasciare una recensione riguardo la sua esperienza presso un ristorante;
    \item \textbf{Comunicazione con un ristoratore:} Il cliente può stabilire un canale di comunicazione con un ristoratore;
    \item \textbf{Inserimento e gestione di un ristorante:} Il ristoratore può inserire e gestire (modificare e/o cancellare) i dati relativi ad un o più ristoranti da lui amministrati;
    \item \textbf{Gestione del menù:} Il ristoratore può gestire il menù del proprio ristorante, modificando le pietanze in esso presenti, inserendone di nuove od eliminandole;
    \item \textbf{Comunicazione con il cliente:} Il ristoratore, analogamente al cliente, può instaurare un canale di comunicazione
      con il cliente, mandandogli e ricevendo messaggi visualizzati all'interno di un'apposita \emph{chat}$^{G}$;
    \item \textbf{Gestione delle prenotazioni:} Il ristoratore può gestire le richieste di prenotazioni (accettandole o rifiutandole), e di quelle accettate, modificarne eventualmente i dati su richiesta del cliente afferente alla prenotazione stessa;
    \item \textbf{Gestione delle ordinazioni:} Il ristoratore può visualizzare le ordinazione dei clienti afferenti ad una una prenotazione, essere notificato quando queste vengono effettuate ed eventualmente modificarle;
    \item \textbf{Consultazione dello stato di pagamento:} Il ristoratore può visualizzare lo stato di pagamento del conto relativo ad una particolare prenotazione;
  \item \textbf{Sommario degli ingredienti:} Il ristoratore può visualizzare l'insieme degli \emph{ingredienti}$^{G}$ a lui necessari per soddisfare le ordinazioni delle prenotazioni relative ad un periodo di uno o più giorni di calendario.
\end{itemize}
