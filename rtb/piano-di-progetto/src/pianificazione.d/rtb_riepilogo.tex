\textbf{Preventivo economico totale}

\begin{tblr}{
    colspec={|X[5cm]|X[3.5cm]|X[1.5cm]|X[3.5cm]},
    row{odd}={bg=white},
    row{even}={bg=lightgray},
    row{1}={bg=black, fg=white},
    row{8}={bg=black, fg=white}
}

    Ruolo & Costo orario (€/h) & N. Ore & Costo totale (€) \\ \hline
    Responsabile & 30,00 & 21 & 630,00 \\ \hline
    Amministratore & 20,00 & 18 & 360,00 \\ \hline
    Analista & 25,00 & 67 & 1675,00 \\ \hline
    Progettista & 25,00 & 26 & 650,00 \\ \hline
    Programmatore & 15,00 & 30 & 450,00 \\ \hline
    Verificatore & 15,00 & 49 & 735,00 \\ \hline
    Totale & \SetCell[c=1]{c} & 211 & 4500,00 \\ \hline

    \end{tblr}

\textbf{Consuntivo economico finale}

\begin{tblr}{
    colspec={|X[5cm]|X[3.5cm]|X[1.5cm]|X[3.5cm]},
    row{odd}={bg=white},
    row{even}={bg=lightgray},
    row{1}={bg=black, fg=white},
    row{8}={bg=black, fg=white}
}

    Ruolo & Costo orario (€/h) & N. Ore & Costo totale (€) \\ \hline
    Responsabile & 30,00 & 19 & 570,00 \\ \hline
    Amministratore & 20,00 & 20 & 400,00 \\ \hline
    Analista & 25,00 & 70 & 1750,00 \\ \hline
    Progettista & 25,00 & 30 & 750,00 \\ \hline
    Programmatore & 15,00 & 40 & 600,00 \\ \hline
    Verificatore & 15,00 & 61 & 915,00 \\ \hline
    Totale & \SetCell[c=1]{c} & 240 & 4985,00 \\ \hline

    \end{tblr}
\pagebreak
\textbf{Preventivo orario totale}

\begin{tblr}{
    colspec={|X[5cm]|X[.5cm]|X[.5cm]|X[.5cm]|X[.5cm]|X[.5cm]|X[.5cm]|X[3.5cm]},
    row{odd}={bg=white},
    row{even}={bg=lightgray},
    row{1}={bg=black, fg=white},
    row{8}={bg=black, fg=white}
}

    Nominativo & Re & Am & An & Pg & Pr & Vf & Ore Totali \\ \hline
    Alberto C. & - & - & 5 & 6 & 15 & 6 & 32 \\ \hline
    Bilal El M. & 4 & 5 & 13 & 6 & - & 15 & 43 \\ \hline
    Alberto M. & 8 & 2 & 19 & - & - & 7 & 36 \\ \hline
    Alex S. & 4 & 11 & 5 & 5 & 5 & 5 & 35 \\ \hline
    Iulius S. & - & - & 25 & - & - & 10 & 35 \\ \hline
    Giovanni Z. & 5 & - & - & 9 & 10 & 6 & 30 \\ \hline
    Totale & 21 & 18 & 67 & 26 & 30 & 49 & 211\\ \hline

\end{tblr}

\textbf{Consuntivo orario finale}

\begin{tblr}{
    colspec={|X[5cm]|X[.5cm]|X[.5cm]|X[.5cm]|X[.5cm]|X[.5cm]|X[.5cm]|X[3.5cm]},
    row{odd}={bg=white},
    row{even}={bg=lightgray},
    row{1}={bg=black, fg=white},
    row{8}={bg=black, fg=white}
}

    Nominativo & Re & Am & An & Pg & Pr & Vf & Ore Totali \\ \hline
    Alberto C. & - & 2 & 3 & 6 & 20 & 9 & 40\\ \hline
    Bilal El M. & 2 & - & 10 & 6 & - & 11 & 29\\ \hline
    Alberto M. & 5 & 5 & 21 & - & - & 10 & 41\\ \hline
    Alex S. & 7 & 13 & 10 & 5 & 5 & 10 & 50\\ \hline
    Iulius S. & - & - & 23 & 4 & - & 13 & 40\\ \hline
    Giovanni Z. & 5 & - & 3 & 9 & 15 & 8 & 40\\ \hline
    Totale & 19 & 20 & 70 & 30 & 40 & 61 & 240\\ \hline

\end{tblr}

\textbf{Gestione dei ruoli} \\
Durante la fase di RTB, il 29\% delle risorse orario è stato dedicato al ruolo di Analista, il 17\% a quello di Programmatore, il 25\% a quello di Verificatore, mentre
solo l'8 \% per le figure rispettivamente del Responsabile e dell'Amministratore; il 13\% per la figura del Progettista.
\textbf{Retrospettiva finale}
Analizzando le gestione da parte del gruppo dei ruoli, salta subito all'occhio un forte sbilanciamento, in parte abbastanza naturale nella fase di RTB, a favore di alcuni ruoli a discapito di altri: Abbondanti risorse sono state spese
per la figura dell'Analista in quanto la produzione dei documenti in generale, e dell'Analisi dei Requisiti in particolare, si è dimostrata un compito assai più oneroso di quanto si fosse preventivato
inizialmente; di questo fatto, è considerata una diretta conseguenza la quantità di risorse dedicate alla verifica e alla validazione della documentazione e del codice del PoC. Il gruppo riconosce retrospettivamente
che troppe poche risorse sono state dedicate in primis alla figura del Responsabile: Un'attenzione evidentemente non sufficiente ai compiti di gestione del gruppo (soprattutto per quanto riguarda la comunicazione),di assegnazione
dei compiti (che ha comportato una differenza importante nella quantità di lavoro profuso da alcuni membri) e
di coordinamento delle risorse impiegate, è considerata dai membri una delle cause principali delle molteplici criticità emerse durante tutta la RTB.
%cosa mettere? Gestione ruoli di tutta la RTB? Retrospettiva totale, anche sui difetti? Spiegare le differenze di ore? Perchè tanta analisi?
% Previsione per la PB: Molte + ore per il progettista, per i programmatori e per i verificatori;