\nonstopmode
\section{Introduzione}
\subsection{Scopo del documento}
Lo scopo del presente documento è delineare in dettaglio le attività necessarie per lo sviluppo del progetto denominato \emph{Easy Meal}.\\
Si analizzeranno i rischi, si illustrerà il modello di sviluppo adottato, si assegneranno i compiti ai membri del gruppo e si presenteranno i preventivi e i consuntivi di periodo.\\
L'obiettivo è raggiungere i risultati in modo efficiente ed efficace attraverso le informazioni qui riportate.
\subsection{Scopo del prodotto}
Lo scopo dell'applicazione è quello di creare una piattaforma che permetta di gestire e semplificare
il processo di \emph{prenotazione}$^{G}$ di tavoli all'interno dei ristoranti. \\
Sarà inoltre possibile anticipare l'esperienza culinaria visionando prima il menù ed
andando ad effettuare la propria \emph{ordinazione}$^{G}$ prima di arrivare al ristorante. \\
Il prodotto offre inoltre un’esperienza di ordinazione delle \emph{pietanze}$^{G}$ collaborativa e coinvolgente,
permettendo di condividerla con amici ed, in caso di dubbi, interagire direttamente con lo staff del ristorante.
L’idea è una piattaforma \emph{SaaS (Software as a Service)}$^{G}$, in cui saranno presenti due tipi di utenti:

\begin{itemize}
\item \emph{Clienti}$^{G}$: Utente registrato all’interno dell’applicazione, può cercare ristoranti, effettuare prenotazioni, ordinazioni e inserire feedback e recensioni;
\item \emph{Ristoratori}$^{G}$: Utente registrato all’interno dell’applicazione, può gestire uno o più ristoranti, controllando le prenotazioni e le ordinazioni dei clienti ed i menù del/i ristorante/i.
\end{itemize}
La piattaforma dovrà essere disponibile attraverso una \emph{Webapp}$^{G}$ accessibile da qualsiasi dispositivo, esso sia \emph{Desktop}$^{G}$ o \emph{Mobile}$^{G}$.

\subsection{Glossario}
Al fine di evitare possibili ambiguità o incomprensioni riguardanti la terminologia usata nel documento, è stato deciso di adottare un glossario in cui vengono riportate le varie definizioni. \\
In questa maniera in esso verranno posti tutti i termini specifici del dominio d’uso con relativi significati. \\
La presenza di un termine all’interno del glossario viene indicata applicando una " $^{G}$ " ad apice della parola.

\pagebreak
\subsection{Riferimenti}
\subsubsection{Riferimenti normativi}

\begin{itemize}
\item Regolamento del progetto didattico: \\ \url{https://www.math.unipd.it/~tullio/IS-1/2023/Dispense/PD2.pdf};
\item \emph{Capitolato d’appalto}$^{G}$ C3 - Easy Meal: \\ \url{https://www.math.unipd.it/~tullio/IS-1/2023/Progetto/C3.pdf} \\;
\item ISO/IEC 12207: Processi del ciclo di vita del software;\\ \url{https://www.iso.org/standard/63712.html};
\item ISO/IEC TR 19759: Software Engineering - Guide to the Software Engineering Body of Knowledge; \\ \url{https://www.iso.org/standard/67604.html};
\item IEEE 830: Pratiche raccomandate per la specifica dei requisiti software;\\ \url{https://standards.ieee.org/ieee/830/1222/}.
\end{itemize}

        \subsubsection{Riferimenti informativi}

\begin{itemize}
\item I processi di ciclo di vita del software: \\ \url{https://www.math.unipd.it/~tullio/IS-1/2023/Dispense/T2.pdf};
\item Il ciclo di vita del SW\\ \url{https://www.math.unipd.it/~tullio/IS-1/2023/Dispense/T3.pdf};
\item Gestione di Progetto\\ \url{https://www.math.unipd.it/~tullio/IS-1/2023/Dispense/T4.pdf};
\item Glossario: \\ \url{https://github.com/SWEet16-SWE-Group/docs/blob/main/RTB/Documentazione%20Esterna/Glossario.pdf}.
\end{itemize}

        Tutti i riferimenti (normativi e informativi) a risorse web soggette a variazione sono stati consultati il 2024/03/26.

