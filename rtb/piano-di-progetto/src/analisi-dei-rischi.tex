\section{Analisi dei rischi}
L'obiettivo dell'analisi dei rischi consiste nel prevedere e comprendere
le possibili problematiche che potrebbero emergere durante il corso del progetto.
Queste problematiche vengono classificate in base al loro impatto e vengono
esplorate soluzioni per mitigarle. L'analisi approfondita dei rischi mira a ottimizzare
il progresso del progetto, consentendo l'identificazione di nuovi scenari e il perferzionamento
delle strategie per mitigarli, attraverso un monitoraggio continuo. \\
Le tabelle presentate nelle sezioni successive espongono i rischi individuati attraverso
le seguenti fasi:
\begin{itemize}
    \item \textbf{Identicazione}: individuazione degli eventi potenziali che potrebbero ostacolare il progresso del progetto
    \item \textbf{Analisi}: valutazione delle probabilità di occorrenza dei rischi, valutazione delle conseguenze
            e assegnazione di un indice di gravità per misurare l'impatto sul progetto
    \item \textbf{Pianificazione di controllo}: sviluppo di metodologie per prevenire l'insorgere
            dei rischi identificati e identificazione delle azioni da intraprendere nel caso in cui si verifichino
    \item \textbf{Monitoraggio}: valutazione continua dei rischi durante lo svolgimento del progetto, con l'obiettivo 
            di individuare nuove minacce e aggiornare le informazioni esistenti.
\end{itemize}
I rischi sono stati categorizzati nelle seguenti aree:
\begin{itemize}
    \item Tecnologie adottate
    \item Organizzazione del lavoro
    \item Relazioni interpersonali
\subsection{Tecnologici}
\begin{tblr}{
    colspec={|X[1,c]|X[2,c]|},
    row{odd}={bg=white},
    row{even}={bg=lightgray},
    row{1}={bg=black,fg=white}
    }
    \hline
    \SetCell[c=2]{} Problemi hardware \\
    \hline
    \textbf{Descrizione} & Ogni membro ha a disposizione un computer per il lavoro, che potrebbe essere soggetto a guasti \\
    \textbf{Conseguenze} & Possibili ritardi nell'avanzamento del progetto \\
    \textbf{Probabilità di manifestazione} & Bassa \\
    \textbf{Pericolosità} & Media \\
    \textbf{Precauzioni} & Ogni modifica a file riguardanti il progetto viene sottoposta a backup tramite
                sistema di versionamento remoto\\
    \textbf{Piano di contingenza} & È necessario cercare di svolgere i compiti assegnati utilizzando un altro 
                dispositivo. In caso non fosse possibile, ricorrere ad una ridistribuzione delle attività in modo
                da attenuare il più possibile il ritardo risultante \\
    \hline
    \end{tblr}
    
    \begin{tblr}{
        colspec={|X[1,c]|X[2,c]|},
        row{odd}={bg=white},
        row{even}={bg=lightgray},
        row{1}={bg=black,fg=white}
        }
        \hline
        \SetCell[c=2]{} Strumenti software \\
        \hline
        \textbf{Descrizione} & Il gruppo non ha esperienza con software per la gestione di 
                    un progetto che prevede stato di avanzamento lavori e di backup\\
        \textbf{Conseguenze} & Possibili ritardi nello sviluppo e inconsistenza dei dati \\
        \textbf{Probabilità di manifestazione} & Bassa \\
        \textbf{Pericolosità} & Alta \\
        \textbf{Precauzioni} & Ogni membro deve segnalare le proprie difficoltà\\
        \textbf{Piano di contingenza} & Verificare l'affidabilità degli strumenti scelti controllando
                    anche se dispongono di una documentazione \\
        \hline
    \end{tblr}
    \begin{tblr}{
        colspec={|X[1,c]|X[2,c]|},
        row{odd}={bg=white},
        row{even}={bg=lightgray},
        row{1}={bg=black,fg=white}
        }
        \hline
        \SetCell[c=2]{} Scarsa esperienza tecnologica \\
        \hline
        \textbf{Descrizione} & Tutti i membri del gruppo partecipano per la prima volta allo
                    svolgimento di un progetto complesso, il che potrebbe causare la comparsa
                    di alcuni problemi durante la fase di progettazione\\
        \textbf{Conseguenze} & Possibili errori di progettazione e difficoltà nella comunicazione \\
        \textbf{Probabilità di manifestazione} & Alta \\
        \textbf{Pericolosità} & Alta \\
        \textbf{Precauzioni} & I membri del gruppo devono comunicare tra di loro riguardo alle 
                    difficoltà più evidenti\\
        \textbf{Piano di contingenza} & Ogni membro del gruppo deve colmare le proprie lacune
                    tecnologiche attraverso lo studio individuale, utilizzando anche i canali
                    di comunicazione interni per chiarimenti vari \\
        \hline
    \end{tblr}
    \begin{tblr}{
        colspec={|X[1,c]|X[2,c]|},
        row{odd}={bg=white},
        row{even}={bg=lightgray},
        row{1}={bg=black,fg=white}
        }
        \hline
        \SetCell[c=2]{} Tecnologie da usare \\
        \hline
        \textbf{Descrizione} & La documentazione relativa alle tecnologie è vasta e complessa\\
        \textbf{Conseguenze} & Ritardo generale del progetto \\
        \textbf{Probabilità di manifestazione} & Media \\
        \textbf{Pericolosità} & Alta \\
        \textbf{Precauzioni} & Non sottostimare il tempo richiesto per lo studio di tali tecnologie\\
        \textbf{Piano di contingenza} & Il proponente offre di organizzare un corso dove vengono spiegate
                    le tecnologie che i membri del gruppo non conoscono\\
        \hline
    \end{tblr}
\subsection{Organizzativi}
\begin{tblr}{
    colspec={|X[1,c]|X[2,c]|},
    row{odd}={bg=white},
    row{even}={bg=lightgray},
    row{1}={bg=black,fg=white}
    }
    \hline
    \SetCell[c=2]{} Conflitti decisionali \\
    \hline
    \textbf{Descrizione} & I membri del gruppo possono essere in disaccordo sulle tecnologie da utilizzare\\
    \textbf{Conseguenze} &  Malcontento all'interno del gruppo \\
    \textbf{Probabilità di manifestazione} & Bassa \\
    \textbf{Pericolosità} & Bassa \\
    \textbf{Precauzioni} & Il membro del gruppo esprimerà la sua disapprovazione al responsabile del progetto\\
    \textbf{Piano di contingenza} & Scelta delle tecnologie attraverso un'adeguata indagine tra i membri del gruppo\\
    \hline
\end{tblr}
\begin{tblr}{
    colspec={|X[1,c]|X[2,c]|},
    row{odd}={bg=white},
    row{even}={bg=lightgray},
    row{1}={bg=black,fg=white}
    }
    \hline
    \SetCell[c=2]{} Disponibilità dei membri \\
    \hline
    \textbf{Descrizione} & I membri del gruppo possono non essere disponibili a causa di impegni extra-universitari\\
    \textbf{Conseguenze} &  Possibile ritardo sull'avanzamento del progetto \\
    \textbf{Probabilità di manifestazione} & Media \\
    \textbf{Pericolosità} & Media \\
    \textbf{Precauzioni} & Ogni membro del gruppo è tenuto a comunicare prontamente la propria indisponibilità al fine 
                di garantire un'orgnaizzazione ottimale\\
    \textbf{Piano di contingenza} & Ridistribuzione dei compiti ai membri non impegnati\\
    \hline
\end{tblr}
\subsection{Personali}

