\section{Analisi dei rischi}
L'obiettivo dell'analisi dei rischi consiste nel prevedere e comprendere
le possibili problematiche che potrebbero emergere durante il corso del progetto.
Queste problematiche vengono classificate in base al loro impatto e vengono
esplorate soluzioni per mitigarle. L'analisi approfondita dei rischi mira a ottimizzare
il progresso del progetto, consentendo l'identificazione di nuovi scenari e il perferzionamento
delle strategie per mitigarli, attraverso un monitoraggio continuo. \\
Le tabelle presentate nelle sezioni successive espongono i rischi individuati attraverso
le seguenti fasi:
\begin{itemize}
    \item \textbf{Identicazione}: individuazione degli eventi potenziali che potrebbero ostacolare il progresso del progetto
    \item \textbf{Analisi}: valutazione delle probabilità di occorrenza dei rischi, valutazione delle conseguenze
            e assegnazione di un indice di gravità per misurare l'impatto sul progetto
    \item \textbf{Pianificazione di controllo}: sviluppo di metodologie per prevenire l'insorgere
            dei rischi identificati e identificazione delle azioni da intraprendere nel caso in cui si verifichino
    \item \textbf{Monitoraggio}: valutazione continua dei rischi durante lo svolgimento del progetto, con l'obiettivo 
            di individuare nuove minacce e aggiornare le informazioni esistenti.
\end{itemize}
I rischi sono stati categorizzati nelle seguenti aree:
\begin{itemize}
    \item Tecnologie adottate
    \item Organizzazione del lavoro
    \item Relazioni interpersonali
\subsection{Tecnologici}
\subsection{Organizzativi}
\subsection{Personali}

