
\section{Pianificazione}
\subsection{Verso la RTB}
\subsubsection{Primo periodo} 
Inizio: 
Fine: 
\paragraph{Preventivo}
\begin{tblr}{
    colspec={|X[5cm]|X[.5cm]|X[.5cm]|X[.5cm]|X[.5cm]|X[.5cm]|X[.5cm]|X[3.5cm]},
    row{odd}={bg=white},
    row{even}={bg=lightgray},
    row{1}={bg=black,fg=white},
    row{8}={bg=black,fg=white}
    }
    
    Nominativo & Re & Am & An & Pg & Pr & Vf & Ore Totali \\ \hline
    Alberto C.    & -  & -  & 5  & -  & -  & 1  & 5 \\ \hline
    Bibal El M.   & -  & -  & 4  & -  & -  & 1  & 5 \\ \hline
    Alberto M.    & 3  & 2  & -  & -  & -  & -  & 5 \\ \hline
    Alex S.       & 2  & 3  & -  & -  & -  & -  & 5 \\ \hline
    Iulius S.     & -  & -  & 4  & -  & -  & 1  & 5 \\ \hline
    Giovanni Z.   & -  & -  & 4  & -  & -  & 1  & 5 \\ \hline
    Totale        & 5  & 5  & 17 & 0  & 0  & 4  & 30 \\ \hline
    
\end{tblr}

\paragraph{Attività svolte} 
Le attività svolte dai membri del gruppo nel primo periodo sono state:
\begin{itemize}
    \item Concepimento e stesura della prima versione del Way of Working;
    \item Studio degli standard di riferimento dei seguenti documenti:
    \begin{itemize}
        \item Norme di progetto;
        \item Piano di progetto;
        \item Analisi dei requisiti.
    \end{itemize}
    \item Ricerca delle tecnologie potenzialmente adatte ad essere incluse nello
    stack tecnologico e loro studio;
    \item Individuazione delle tecnologie di supporto nella produzione di documenti 
    e nel versionamento degli stessi e successivamente del codice ;
    \item Studio delle tecnologie di supporto (LaTex e Git).
\end{itemize}
\paragraph{Diagrammi di Gantt}
\paragraph{Consuntivo}
\begin{tblr}{
    colspec={|X[5cm]|X[.5cm]|X[.5cm]|X[.5cm]|X[.5cm]|X[.5cm]|X[.5cm]|X[3.5cm]},
    row{odd}={bg=white},
    row{even}={bg=lightgray},
    row{1}={bg=black,fg=white},
    row{8}={bg=black,fg=white}
    }
    
    Nominativo & Re & Am & An & Pg & Pr & Vf & Ore Totali \\ \hline
    Alberto C.    & -  & -  & 5  & -  & -  & 1  & 5 \\ \hline
    Bibal El M.   & -  & -  & 4  & -  & -  & 1  & 5 \\ \hline
    Alberto M.    & 3  & 2  & -  & -  & -  & -  & 5 \\ \hline
    Alex S.       & 2  & 3  & -  & -  & -  & -  & 5 \\ \hline
    Iulius S.     & -  & -  & 4  & -  & -  & 1  & 5 \\ \hline
    Giovanni Z.   & -  & -  & 4  & -  & -  & 1  & 5 \\ \hline
    Totale        & 5  & 5  & 17 & 0  & 0  & 4  & 30 \\ \hline

\end{tblr}

\paragraph{Gestione dei ruoli} 

\paragraph{Gestione dei rischi}

\subsubsection{Secondo periodo} 
Inizio: 
Fine:
\paragraph{Preventivo}
\paragraph{Attività svolte} 
Le attività svolte dal gruppo nel secondo periodo sono state:
\begin{itemize}
    \item Analisi dei rischi che possono verificarsi durante lo svolgimento dei processi 
    che compongono il progetto e delle relative possibili mitigazioni;
    \item In seguito all'individuazione delle tecnologie , codifica del PoC;
    \item Colloquio con il prof. Cardin per ottenere delucidazioni riguardo la formalità dei requisiti;
    \item Prima stesura dei requisiti funzionali del prodotto da sviluppare;
    \item Prosieguo della stesura delle Norme di Progetto;
    \item Studio delle metriche da adottare al fine di misurare la qualità di prodotto e dei processi (particolare attenzione
    spesa per quest'ultimi);
    \item Webinar organizzato dall'azienda proponente sull'utilizzo di Docker.
\end{itemize}
\paragraph{Gestione dei ruoli}
\paragraph{Diagrammi di Gantt}
\paragraph{Consuntivo}
\paragraph{Gestione dei rischi}

\subsubsection{Terzo periodo}
Inizio:
Fine: 
\paragraph{Preventivo}
\paragraph{Attività svolte} 
Le attività svolte dal gruppo nel terzo periodo sono state:
\begin{itemize}
    \item Stesura della prima versione dell' Analisi dei Requisiti;
    \item Inizio della stesura del Piano di Qualifica;
    \item Creazione di una script per la verifica automatica dell'aderenza dei documenti alle norme tipografiche;
    \item Prosieguo della stesura delle Norme di Progetto;
    \item Migrazione di tutta la documentazione presente nel Google Drive presso la repository su GitHub;
    \item Adozione di nuovi metodi per la gestione della configurazione dei documenti all'interno della repo
    e per una produzione più efficiente degli stessi da parte dei membri del gruppo;
    \item Incontro con il prof. Cardin nel quale è stato presentato il PoC e il documento di Analisi dei Requisiti;
\end{itemize}
\paragraph{Gestione dei ruoli}
\paragraph{Diagrammi di Gantt}
\paragraph{Consuntivo}
\paragraph{Gestione dei rischi}

\subsubsection{Quarto periodo} 
Inizio: 
Fine: 
\paragraph{Preventivo}
\paragraph{Attività svolte}
Le attività svolte dal gruppo nel quarto periodo sono state:
\begin{itemize}
    \item Modifica dell'Analisi dei Requisiti e correzione degli errori come indicato dal prof. Cardin;
    \item Completamento della sezione "Pianificazione" del Piano di Progetto;
    \item Completamento della stesura del Piano di Qualifica;
    \item Completamento delle Norme di Progetto;
    \item Creazione di uno script per individuare i termini presenti nei vari documenti ,che debbono essere presenti
    nel Glossario;
    \item Consuntivo finale della fase RTB;
    \item Preparazione della presentazione per la seconda fase della RTB.
\end{itemize}
\paragraph{Gestione dei ruoli}
\paragraph{Diagrammi di Gantt}
\paragraph{Consuntivo}
\paragraph{Gestione dei rischi}
