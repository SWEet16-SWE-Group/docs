
\section{Pianificazione}
\subsection{Verso la RTB}
\subsubsection{Primo periodo} 
Inizio: \\
Fine: 
\paragraph{Preventivo}
\subparagraph{}
\begin{tblr}{
    colspec={|X[5cm]|X[.5cm]|X[.5cm]|X[.5cm]|X[.5cm]|X[.5cm]|X[.5cm]|X[3.5cm]},
    row{odd}={bg=white},
    row{even}={bg=lightgray},
    row{1}={bg=black,fg=white},
    row{8}={bg=black,fg=white}
    }
    
    Nominativo & Re & Am & An & Pg & Pr & Vf & Ore Totali \\ \hline
    Alberto C.    & -  & -  & 5  & -  & -  & 1  & 5 \\ \hline
    Bibal El M.   & -  & -  & 4  & -  & -  & 1  & 5 \\ \hline
    Alberto M.    & 3  & 2  & -  & -  & -  & -  & 5 \\ \hline
    Alex S.       & 2  & 3  & -  & -  & -  & -  & 5 \\ \hline
    Iulius S.     & -  & -  & 4  & -  & -  & 1  & 5 \\ \hline
    Giovanni Z.   & -  & -  & 4  & -  & -  & 1  & 5 \\ \hline
    Totale        & 5  & 5  & 17 & 0  & 0  & 4  & 30 \\ \hline
    
\end{tblr}

\paragraph{Attività svolte} 
\subparagraph{}
Le attività svolte dai membri del gruppo nel primo periodo sono state:
\begin{itemize}
    \item Concepimento e stesura della prima versione del Way of Working;
    \item Studio degli standard di riferimento dei seguenti documenti:
    \begin{itemize}
        \item Norme di progetto;
        \item Piano di progetto;
        \item Analisi dei requisiti.
    \end{itemize}
    \item Ricerca delle tecnologie potenzialmente adatte ad essere incluse nello
    stack tecnologico e loro studio;
    \item Individuazione delle tecnologie di supporto nella produzione di documenti 
    e nel versionamento degli stessi e successivamente del codice ;
    \item Studio delle tecnologie di supporto (LaTex e Git).
\end{itemize}
\paragraph{Diagrammi di Gantt}
\paragraph{Consuntivo}
\subparagraph{}
\begin{tblr}{
    colspec={|X[5cm]|X[.5cm]|X[.5cm]|X[.5cm]|X[.5cm]|X[.5cm]|X[.5cm]|X[3.5cm]},
    row{odd}={bg=white},
    row{even}={bg=lightgray},
    row{1}={bg=black,fg=white},
    row{8}={bg=black,fg=white}
    }
    
    Nominativo    & Re & Am & An & Pg & Pr & Vf & Ore Totali \\ \hline
    Alberto C.    & -  & -  & 5  & -  & -  & 1  & 5 \\ \hline
    Bibal El M.   & -  & -  & 4  & -  & -  & 1  & 5 \\ \hline
    Alberto M.    & 3  & 2  & -  & -  & -  & -  & 5 \\ \hline
    Alex S.       & 2  & 3  & -  & -  & -  & -  & 5 \\ \hline
    Iulius S.     & -  & -  & 4  & -  & -  & 1  & 5 \\ \hline
    Giovanni Z.   & -  & -  & 4  & -  & -  & 1  & 5 \\ \hline
    Totale        & 5  & 5  & 17 & 0  & 0  & 4  & 30 \\ \hline

\end{tblr}

\paragraph{Gestione dei ruoli} 
\subparagraph{}
In questo primo periodo,il 57\% delle ore di lavoro sono state dedicate alla fase di analisi,
data la necessità di effettuare uno studio preliminare dei documenti che dovranno essere prodotti durante
l'arco del progetto (in particolare le Norme di Progetto e il Piano di Progetto) e quindi dei relativi Standard
internazionali di riferimento; il 17\% delle ore è stato dedicato al ruolo del Responsabile, in cui si sono mossi i primi 
passi nell'ambito della gestione ed organizzazione del carico di lavoro tra i vari membri; un'eguale percentuale di ore 
è stata spesa per l'organizzazione della repository sulla piattaforma GitHub ed infine il 13\% delle ore al ruolo di verificatore,
durante il quale, i membri hanno preso dimestichezza con l'operazione di verifica e validazione dei primi documenti prodotti dal gruppo.

\paragraph{Gestione dei rischi}
\subparagraph{} 
\textbf{Rischio verificatosi:} Scarsa esperienza tecnologica. \\
\textbf{Esito PdC:} Tramite lo studio individuale e la trasmissione di conoscenza dai membri maggiormente esperti delle tecnologie di supporto
per i processi di supporto, a quelli meno esperti, il gruppo è riuscito efficacemente a mettere in pratica il piano di contingenza previsto per il suddetto
rischio.\\
\textbf{Impatto:} L'impatto della scarsa esperienza tecnologica è stato nullo in questo primo periodo,
avendo il gruppo preventivato la necessità di dedicare innanzitutto risorse all'individuazione,allo studio e all'apprendimento
delle tecnologie di supporto, così come di quelle che andranno a far parte dello stack tecnologico del prodotto.\\ 

\subsubsection{Secondo periodo} 
Inizio: \\
Fine:
\paragraph{Preventivo}
\subparagraph{}

\begin{tblr}{
    colspec={|X[5cm]|X[.5cm]|X[.5cm]|X[.5cm]|X[.5cm]|X[.5cm]|X[.5cm]|X[3.5cm]},
    row{odd}={bg=white},
    row{even}={bg=lightgray},
    row{1}={bg=black,fg=white},
    row{8}={bg=black,fg=white}
    }
    
    Nominativo    & Re & Am & An & Pg & Pr & Vf & Ore Totali \\ \hline
    Alberto C.    & -  & -  & -  & -  & 15 & 2  & 17 \\ \hline
    Bibal El M.   & 2  & 3  & 4  & -  & -  & 2  & 11 \\ \hline
    Alberto M.    & 5  & -  & 5  & -  & -  & -  & 10 \\ \hline
    Alex S.       & -  & -  & -  & 5  & 5  & 5  & 15 \\ \hline
    Iulius S.     & -  & -  & 7  & -  & -  & 2  & 9  \\ \hline
    Giovanni Z.   & -  & -  & -  & -  & 10 & -  & 10 \\ \hline
    Totale        & 7  & 3  & 16 & 5  & 30 & 11 & 72 \\ \hline

\end{tblr}

\paragraph{Attività svolte} 
\subparagraph{}
Le attività svolte dal gruppo nel secondo periodo sono state:
\begin{itemize}
    \item Analisi dei rischi che possono verificarsi durante lo svolgimento dei processi 
    che compongono il progetto, e delle relative possibili mitigazioni;
    \item In seguito all'individuazione delle tecnologie ,progettazione e codifica del PoC;
    \item Colloquio con il prof. Cardin per ottenere delucidazioni riguardo lo studio dei requisiti funzionali;
    \item Prima stesura dei requisiti funzionali del prodotto da sviluppare;
    \item Prosieguo della stesura delle Norme di Progetto;
    \item Studio delle metriche da adottare al fine di misurare la qualità di prodotto e dei processi (particolare attenzione
    spesa per quest'ultimi);
    \item Webinar organizzato dall'azienda proponente sull'utilizzo di Docker.
\end{itemize}
\paragraph{Diagrammi di Gantt}
\paragraph{Consuntivo}

\begin{tblr}{
    colspec={|X[5cm]|X[.5cm]|X[.5cm]|X[.5cm]|X[.5cm]|X[.5cm]|X[.5cm]|X[3.5cm]},
    row{odd}={bg=white},
    row{even}={bg=lightgray},
    row{1}={bg=black,fg=white},
    row{8}={bg=black,fg=white}
    }
    
    Nominativo    & Re & Am & An & Pg & Pr & Vf & Ore Totali \\ \hline
    Alberto C.    & -  & -  & -  & -  & 20 & 5  & 25 \\ \hline
    Bibal El M.   & -  & 5  & 4  & -  & -  & 2  & 11 \\ \hline
    Alberto M.    & 5  & -  & 5  & -  & -  & -  & 10 \\ \hline
    Alex S.       & -  & -  & -  & 5  & 5  & 5  & 15 \\ \hline
    Iulius S.     & -  & -  & 7  & -  & -  & 2  & 9  \\ \hline
    Giovanni Z.   & -  & -  & -  & -  & 15 & -  & 15 \\ \hline
    Totale        & 5  & 5  & 16 & 5  & 40 & 14 & 85\\ \hline

\end{tblr}
\paragraph{Gestione dei ruoli}
\subparagraph{}
Nel secondo periodo, il ruolo per il quale è stata spesa la fetta più consistente di ore (47\%),
è stato quello di Programmatore, data la volontà del gruppo di completare entro la fine del periodo 
la codifica del PoC e la relativa fase di test (alla quale è stato dedicato il 16\% delle ore); data la necessità
di proseguire con la stesura della documentazione e con l'analisi dei requisiti funzionali, il 19\% delle ore è stato
dedicato al ruolo di Analista; il 6\% delle ore è stato dedicato rispettivamente alle figure del Responsabile,
dell'Amministratore e del Progettista.

\paragraph{Gestione dei rischi}

\subparagraph{}
\textbf{Rischio verificatosi:} Conflitti decisionali.\\
\textbf{Esito PdC:}Essendosi presentati punti di vista differenti riguardo le tecnologie da includere
nello stack ,sia per quanto riguarda la parte di Front-end  che quella di Back-end, i membri del gruppo 
hanno avviato una discussione nella quale considerare i pro e i contro delle differenti proposte; si è deciso di 
affidare la scelta definitiva ai membri con la maggior expertise nell'ambito dei linguaggi di programmazione.\\
\textbf{Impatto:} L'impatto è stato nullo,grazie alla disponibilità e alla maturità dimostrata da tutti i membri del gruppo
nell'affrontare un conflitto decisionale e  nell'accettarne la risoluzione.  \\

\textbf{Rischio verificatosi:} Tecnologie da usare.\\
\textbf{Esito PdC:} L'azienda proponente ha offerto al gruppo la possibilità di partecipare ad un Webinar
su Docker, grazie al quale molti dubbi di alcuni membri in particolare sono stati risolti ,permettendo un avanzamento 
importante nella codifica del PoC e un allineamento tra tutti i membri sulla conoscenza del sopracitato Docker.
L'utilizzo invece della libreria React si è rivelato più complesso del previsto,richiedendo ore supplementari di studio
da parte dei programmatori della documentazione relativa.\\
\textbf{Impatto:} Importante è stato l'impatto del rischio legato alle tecnologie utilizzate per la parte Front-End del PoC,
avendo richiesto in totale 10 ore in più rispetto a quelle preventivate inizialmente dal gruppo.\\

\subsubsection{Terzo periodo}
Inizio: \\
Fine: 
\paragraph{Preventivo}

\begin{tblr}{
    colspec={|X[5cm]|X[.5cm]|X[.5cm]|X[.5cm]|X[.5cm]|X[.5cm]|X[.5cm]|X[3.5cm]},
    row{odd}={bg=white},
    row{even}={bg=lightgray},
    row{1}={bg=black,fg=white},
    row{8}={bg=black,fg=white}
    }
    
    Nominativo    & Re & Am & An & Pg & Pr & Vf & Ore Totali \\ \hline
    Alberto C.    & -  & -  & 5  & -  & -  & -  & 5 \\ \hline
    Bibal El M.   & -  & -  & 5  & -  & -  & 7  & 12 \\ \hline
    Alberto M.    & -  & -  & 10 & -  & -  & 2  & 12 \\ \hline
    Alex S.       & -  & 5  & 5  & -  & -  & -  & 10 \\ \hline
    Iulius S.     & -  & -  & 10 & -  & -  & 2  & 12  \\ \hline
    Giovanni Z.   & 5  & -  & -  & -  & -  & -  & 10 \\ \hline
    Totale        & 5  & 5  & 35 & 0  & 0  & 11 & 80\\ \hline

\end{tblr}

\paragraph{Attività svolte} 
Le attività svolte dal gruppo nel terzo periodo sono state:
\begin{itemize}
    \item Stesura della prima versione dell' Analisi dei Requisiti;
    \item Inizio della stesura del Piano di Qualifica;
    \item Creazione di una script per la verifica automatica dell'aderenza dei documenti alle norme tipografiche;
    \item Prosieguo della stesura delle Norme di Progetto;
    \item Migrazione di tutta la documentazione presente nel Google Drive presso la repository su GitHub;
    \item Adozione di nuovi metodi per la gestione della configurazione dei documenti all'interno della repo,
    e per una produzione più efficiente degli stessi da parte dei membri del gruppo;
    \item Incontro con il prof. Cardin nel quale è stato presentato il PoC e il documento di Analisi dei Requisiti;
\end{itemize}
\paragraph{Diagrammi di Gantt}
\paragraph{Consuntivo}

\begin{tblr}{
    colspec={|X[5cm]|X[.5cm]|X[.5cm]|X[.5cm]|X[.5cm]|X[.5cm]|X[.5cm]|X[3.5cm]},
    row{odd}={bg=white},
    row{even}={bg=lightgray},
    row{1}={bg=black,fg=white},
    row{8}={bg=black,fg=white}
    }
    
    Nominativo    & Re & Am & An & Pg & Pr & Vf & Ore Totali \\ \hline
    Alberto C.    & -  & 2  & 3  & -  & -  & -  & 5 \\ \hline
    Bibal El M.   & -  & 5  & 2  & -  & -  & 3  & 10 \\ \hline
    Alberto M.    & -  & -  & 12 & -  & -  & 5  & 17 \\ \hline
    Alex S.       & -  & 5  & 10 & -  & -  & 5  & 20 \\ \hline
    Iulius S.     & -  & -  & 12 & -  & -  & 5  & 17  \\ \hline
    Giovanni Z.   & 5  & -  & 3  & -  & -  & 2  & 10 \\ \hline
    Totale        & 5  & 12 & 42 & 0  & 0  & 20 & 79 \\ \hline

\end{tblr}

\paragraph{Gestione dei ruoli}
\subparagraph{}
Nel terzo periodo, il 53\% delle risorse temporali del gruppo è stato speso nel ruolo dell'Analista,data la necessità 
di portare a compimento la stesura della prima versione dell'Analisi dei Requisiti in vista della prima fase della revisione
RTB; di conseguenza il gruppo ha utilizzato il 25\% delle ore per i processi di verifica e validazione del documento sopra citato,processi che il 
gruppo ha attenzionato particolarmente,vista l'importanza del documento; inoltre il 15\% delle ore sono state dedicate alla
migrazione di tutta la documentazione sulla repository in GitHub e alla messa a punto della relativa gestione di configurazione.

\paragraph{Gestione dei rischi}

\subparagraph{}
\textbf{Rischio verificatosi:} Analisi incompleta/carente dei requisiti.\\
\textbf{Esito PdC:} I membri del gruppo deputati alla stesura dell'Analisi dei Requisiti hanno constatato tardivamente
che i requisiti funzionali individuati prima del terzo periodo fossero insufficienti, in termini di numero e di 
profondità dell'analisi stessa; si è deciso quindi di destinare un maggior numero di ore a quest'ultima e di incrementare da
 2 a 3 il numero di componenti deputati ad essa. \\
\textbf{Impatto:} Significativo è stato l'impatto del suddetto rischio,avendo comportato la necessità
da parte del gruppo, di riallocare parte delle risorse orarie,dedicando 7 ore in più alla fase di analisi
cercando però di non aumentare il monte orario complessivo preventivato all'inizio del terzo periodo;sono state
quindi tolte risorse alla stesura degli altri documenti.\\

\subparagraph{}
\textbf{Rischio verificatosi:} Calcolo delle tempistiche.\\
\textbf{Esito PdC:}In seguito al verificarsi del rischio legato ad una carente analisi dei requisiti, il Responsabile ha 
intuito il fatto che si sarebbe reso necessario aumentare non solo le ore da dedicare all'analisi stessa, ma anche all verifica e 
validazione delle sezioni del relativo documento che venivano iterativamente prodotte e corrette; il gruppo è riuscito  a non aumentare
il monte orario preventivato all'inizio del terzo periodo.\\
\textbf{Impatto:} L'impatto è stato significativo; nonostante non direttamente misurabile in un aumento del monte orario preventivato,
si è tradotto in un'importante riallocazione di ore , tolte dal prosieguo della stesura della restante documentazione (in particolare del Piano
di Qualifica).\\

\subsubsection{Quarto periodo} 
Inizio: \\
Fine: 
\paragraph{Preventivo}

\begin{tblr}{
    colspec={|X[5cm]|X[.5cm]|X[.5cm]|X[.5cm]|X[.5cm]|X[.5cm]|X[.5cm]|X[3.5cm]},
    row{odd}={bg=white},
    row{even}={bg=lightgray},
    row{1}={bg=black,fg=white},
    row{8}={bg=black,fg=white}
    }
    
    Nominativo    & Re & Am & An & Pg & Pr & Vf & Ore Totali \\ \hline
    Alberto C.    & -  & -  & 2  & -  & -  & 3  & 5 \\ \hline
    Bibal El M.   & 2  & 2  & 6  & -  & -  & 5  & 15 \\ \hline
    Alberto M.    & -  & -  & 4  & -  & -  & 5  & 9 \\ \hline
    Alex S.       & 2  & 3  & -  & -  & -  & -  & 5 \\ \hline
    Iulius S.     & -  & -  & 4  & -  & -  & 5  & 9  \\ \hline
    Giovanni Z.   & -  & -  & 5  & -  & -  & 5  & 10 \\ \hline
    Totale        & 4  & 5  & 21 & 0  & 0  & 23 & 53 \\ \hline

\end{tblr}

\paragraph{Attività svolte}
Le attività svolte dal gruppo nel quarto periodo sono state:
\begin{itemize}
    \item Modifica dell'Analisi dei Requisiti e correzione degli errori come indicato dal prof. Cardin;
    \item Completamento della sezione "Pianificazione" del Piano di Progetto;
    \item Completamento della stesura del Piano di Qualifica;
    \item Completamento delle Norme di Progetto;
    \item Creazione di uno script per individuare i termini presenti nei vari documenti ,che debbono essere presenti
    nel Glossario;
    \item Stesura della prima versione del Glossario;
    \item Consuntivo finale della fase RTB;
    \item Preparazione della presentazione per la seconda fase della RTB.
\end{itemize}
\paragraph{Diagrammi di Gantt}
\paragraph{Consuntivo}
\subparagraph{}

\begin{tblr}{
    colspec={|X[5cm]|X[.5cm]|X[.5cm]|X[.5cm]|X[.5cm]|X[.5cm]|X[.5cm]|X[3.5cm]},
    row{odd}={bg=white},
    row{even}={bg=lightgray},
    row{1}={bg=black,fg=white},
    row{8}={bg=black,fg=white}
    }
    
    Nominativo    & Re & Am & An & Pg & Pr & Vf & Ore Totali \\ \hline
    Alberto C.    & -  & -  & 2  & -  & -  & 3  & 5 \\ \hline
    Bibal El M.   & 2  & 2  & 6  & -  & -  & 5  & 15 \\ \hline
    Alberto M.    & -  & -  & 4  & -  & -  & 5  & 9 \\ \hline
    Alex S.       & 5  & 5  & -  & -  & -  & -  & 10 \\ \hline
    Iulius S.     & -  & -  & 4  & -  & -  & 5  & 9  \\ \hline
    Giovanni Z.   & -  & -  & 5  & -  & -  & 5  & 10 \\ \hline
    Totale        & 7  & 7  & 21 & 0  & 0  & 23 & 58 \\ \hline

\end{tblr}

\paragraph{Gestione dei ruoli}
\subparagraph{}
Come si può vedere dalla suddetta tabella,in questo quarto periodo il gruppo ha concentrato buona parte del monte-ore nello svolgimento
del ruolo di Analista e Verificatore :per quanto concerne il primo, i membri hanno speso una parte consistente delle proprie
risorse  (36\%) al fine di completare la stesura di tutta la documentazione e di effettuare le correzioni all'Analisi dei Requisiti
in seguito alla revisione del prof. Cardin; di conseguenza, una parte cospicua di risorse (39\%) è stata utilizzata per la fase di verifica 
e validazione della documentazione sopracitata; il gruppo ha inoltre riservato il 12\% delle ore per il Responsabile ,data l'importanza di un efficace
ed efficiente orchestrazione delle attività dei membri del gruppo , ed una quantità uguale di ore alla figura dell'Amministratore, al fine
di evitare potenziali conflitti negli item di versionamento prodotti dai differenti membri.

\paragraph{Gestione dei rischi}
