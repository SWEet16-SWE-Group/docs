\subsection{Account}Insieme di credenziali e dati associati a un singolo utente che consente loro di accedere e interagire con i servizi offerti dalla piattaforma online.
\subsection{Back-end}Parte di un'applicazione o di un sistema informatico che gestisce e elabora i dati e le operazioni logiche lato server.
\subsection{Best practices}Approcci, metodologie o procedure che sono riconosciute come le più efficaci e efficienti per raggiungere un determinato obiettivo o risultato desiderato in un determinato contesto o settore. Queste pratiche sono solitamente identificate attraverso l'esperienza, la ricerca, l'analisi dei dati e l'osservazione delle prestazioni passate.
\subsection{Capitolato}Documento redatto dal proponente in cui vengono specificati i vincoli utente e vincoli tecnologici per lo sviluppo esplorativo di un determinato prodotto software. Serve ad esporre un problema, per cercare di trovarvi una soluzione.
\subsection{Carrello}Funzionalità che consente agli utenti di raccogliere e gestire i prodotti che desiderano acquistare durante la loro esperienza di ordinazione online.Le caratteristiche tipiche di un carrello virtuale sono l'aggiunta e rimozione di un prodotto, la modifica delle quantità, il calcolo del totale e il salvataggio temporaneo dei prodotti.
\subsection{Casi d’uso}Tecnica utilizzata nell'ambito dell'analisi dei requisiti del software per descrivere interazioni o scenari specifici tra gli utenti (attori) e un sistema software. I casi d'uso sono spesso documentati tramite diagrammi UML (Unified Modeling Language), che forniscono una rappresentazione visuale delle interazioni tra attori e sistema.
\subsection{Chat}Interazione testuale o multimediale in tempo reale tra due o più persone attraverso un'applicazione o una piattaforma online dedicata. Facilita la comunicazione immediata e sincrona, indipendentemente dalla distanza geografica.
\subsection{Cliente}Individuo o entità che utilizza un sito web, un'applicazione mobile o una piattaforma digitale per effettuare acquisti di beni o servizi forniti da un'azienda o un'organizzazione tramite il canale online.
\subsection{Consuntivo}Bilancio dei risultati ottenuti a rendiconto di un certo periodo temporale di attività, in termini di tempo e risorse.
\subsection{Conto}Documento cartaceo o elettronico che riepiloga i costi relativi ai pasti consumati e ai servizi forniti in un ristorante o in un locale simile.
\subsection{Coperti} (INSERIRE DEFINIZIONE QUI)
\subsection{Deployment} (INSERIRE DEFINIZIONE QUI)
\subsection{Design}Processo di creazione, pianificazione e organizzazione di elementi in modo da raggiungere uno scopo specifico.
\subsection{Desktop}Area principale dell'interfaccia grafica del sistema operativo dove vengono visualizzate icone, file e altre risorse del computer. Fornisce un'organizzazione visiva degli elementi presenti sul computer e facilita l'accesso rapido alle risorse utili.
\subsection{Docker} (INSERIRE DEFINIZIONE QUI)
\subsection{ExpressJS}Framework web leggero, flessibile e minimalista per Node.js, un runtime JavaScript lato server. È ampiamente utilizzato per la creazione di applicazioni web e API (Application Programming Interface) utilizzando JavaScript sia lato server che lato client.
\subsection{Firefox} (INSERIRE DEFINIZIONE QUI)
\subsection{Front-end}Parte di un'applicazione informatica, di un sito web o di un'applicazione mobile che interagisce direttamente con gli utenti e che è visibile e accessibile attraverso l'interfaccia utente.
\subsection{GitHub} (INSERIRE DEFINIZIONE QUI)
\subsection{Git} (INSERIRE DEFINIZIONE QUI)
\subsection{Google Chrome} (INSERIRE DEFINIZIONE QUI)
\subsection{IDE} (INSERIRE DEFINIZIONE QUI)
\subsection{ITS (Issue Tracking System)}Software o applicazione utilizzata per registrare, gestire e monitorare le issue, i problemi e i compiti all'interno di un progetto. Fornisce un'infrastruttura organizzativa per tenere traccia delle richieste di funzionalità, dei bug, delle richieste di supporto e dei compiti assegnati, consentendo ai membri del team di sviluppo di collaborare in modo efficace e gestire il lavoro in corso.
\subsection{Ingredienti}Sostanze o componenti di origine naturale o sintetica utilizzate nella preparazione di cibo o di altre sostanze chimiche o composti.
\subsection{Issue}Registrazione di un problema, una richiesta di funzionalità o un compito specifico che deve essere affrontato durante il ciclo di sviluppo di un progetto.
\subsection{LaTeX}Linguaggio di markup e sistema di preparazione di documenti potente e flessibile, ampiamente utilizzato per la produzione di documenti tecnici e scientifici di alta qualità.
\subsection{Mobile}Dispositivi informatici che sono progettati per essere portatili e utilizzati in movimento, quali smartphone, tablet e altri dispositivi simili che consentono agli utenti di accedere a internet, utilizzare applicazioni e svolgere una vasta gamma di attività digitali ovunque si trovino.
\subsection{Modello a V}Modello di sviluppo software che prende il nome dalla sua forma a "V" quando viene visualizzato graficamente. Questo modello è una rappresentazione del processo di sviluppo del software che mostra le fasi di sviluppo e i test correlati, organizzati in una struttura a forma di V. Utilizzato in progetti software in cui è richiesta una pianificazione dettagliata e una forte enfasi sulla verifica e sulla validazione del software.
\subsection{NextJS}Framework open-source di React per lo sviluppo di applicazioni web front-end e full-stack. È progettato per semplificare e velocizzare il processo di creazione di applicazioni web moderne, fornendo strumenti e funzionalità avanzate integrate.
\subsection{Ordinazione}Processo di richiesta e acquisizione di beni o servizi da un'organizzazione, un'azienda o un fornitore. Questo processo coinvolge solitamente un cliente che seleziona i prodotti o i servizi desiderati e li richiede al fornitore o al venditore, specificando eventuali dettagli aggiuntivi come quantità, varianti o preferenze.
\subsection{Pietanze}Porzioni di cibo preparate per essere consumate come pasto o come parte di un pasto.
\subsection{PoC (Proof of Concept)}Realizzazione incompleta o abbozzata di un determinato progetto, per provare la fattibilità di idee costituite e tecnologie presenti nel progetto, sulla base di principi/concetti usati come fondamento. Si intende l’idea di prototipo, usato come oggetto di studio.
\subsection{Prenotazione}Processo attraverso il quale i clienti fissano in anticipo un tavolo per consumare un pasto presso un ristorante in una data e un'ora specifiche. Le prenotazioni sono comunemente utilizzate nei ristoranti per garantire ai clienti un posto al momento desiderato e per consentire al ristorante di pianificare e organizzare il servizio in modo efficace. La prenotazione si compone delle seguenti fasi: richiesta di prenotazione, verifica della disponibilità, conferma della prenotazione, arrivo e assegnazione del tavolo.
\subsection{Profilo}Insieme delle informazioni relative a un singolo utente che vengono memorizzate e gestite da un'applicazione o da un servizio. Queste informazioni possono includere dati personali, preferenze, cronologia degli acquisti, attività passate e altro ancora.
\subsection{Proponente}Azienda o ente che propone il capitolato d’appalto per il progetto didattico.
\subsection{Pull request}Funzionalità comune nei sistemi di controllo delle versioni, come Git e GitHub, utilizzata per richiedere la revisione e l'integrazione delle modifiche apportate a un repository di codice sorgente.
\subsection{RTB}Insieme di documenti e informazioni chiave che stabiliscono i requisiti e le basi tecnologiche per un progetto specifico, fornendo una chiara base di partenza e una guida per tutto lo sviluppo.
\subsection{Requisito funzionale}Specifica di ciò che un sistema software deve fare o delle funzionalità che deve possedere per soddisfare le esigenze degli utenti, dei clienti o degli stakeholder.
\subsection{Requisito}Specifica, istanza di un servizio o caratteristica che è necessaria o desiderata per soddisfare un obiettivo o risolvere un problema.
\subsection{Ristoratore}Individuo o un imprenditore che gestisce un ristorante o un'attività di ristorazione.
\subsection{SSL/TLS} (INSERIRE DEFINIZIONE QUI)
\subsection{SaaS (Software as a Service)}Modello di distribuzione del software in cui le applicazioni sono ospitate su cloud da un provider di servizi e rese disponibili agli utenti attraverso Internet. Gli utenti possono accedere al software tramite un browser web o un'interfaccia utente, senza dover installare o mantenere l'applicazione sul proprio dispositivo.
\subsection{Safari} (INSERIRE DEFINIZIONE QUI)
\subsection{Telegram}Applicazione multipiattaforma di messaggistica istantanea basata su cloud che permette una facile comunicazione con gruppi e canali, organizzando la comunicazione sulla base di strumenti e funzionalità automatiche offerte (bot), condividendo facilmente file e messaggi in un canale unico di comunicazione.
\subsection{UML}UML è l'acronimo di Unified Modeling Language (Linguaggio di Modellazione Unificato). Si tratta di un linguaggio standardizzato per la modellazione e la documentazione dei sistemi software. Fornisce un insieme di notazioni grafiche e simboli che consentono agli sviluppatori di visualizzare, progettare e comunicare le diverse aspetti di un sistema software in modo chiaro e conciso.
\subsection{Way of Working}Terminologia che indica il modo di lavorare adottato, da un punto di vista di strumenti, attività e tecnologie, determinato in modo quantificabile e misurato.
\subsection{Webapp}Programma applicativo memorizzato su un server remoto e distribuito su Internet attraverso un'interfaccia browser.
\subsection{WhatsApp}Applicazione multipiattaforma di messaggistica istantanea basata su cloud che che consente agli utenti di inviare messaggi di testo, immagini, video, documenti e messaggi vocali a singoli o a gruppi tramite Internet.
