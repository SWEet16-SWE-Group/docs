\section{Account}
Insieme di credenziali e dati associati a un singolo utente che consente loro di accedere e interagire con i servizi offerti dalla piattaforma online.

\section{Back-end}
Parte di un'applicazione o di un sistema informatico che gestisce e elabora i dati e le operazioni logiche lato server.

\section{Best practices}
Approcci, metodologie o procedure che sono riconosciute come le più efficaci e efficienti per raggiungere un determinato obiettivo o risultato desiderato in un determinato contesto o settore. Queste pratiche sono solitamente identificate attraverso l'esperienza, la ricerca, l'analisi dei dati e l'osservazione delle prestazioni passate.

\section{Capitolato}:
Documento redatto dal proponente in cui vengono specificati i vincoli utente e vincoli tecnologici per lo sviluppo esplorativo di un determinato prodotto software. Tale documento serve infatti ad esporre un problema, per cercare di trovarvi una soluzione.

\section{Carrello}
Funzionalità che consente agli utenti di raccogliere e gestire i prodotti che desiderano acquistare durante la loro esperienza di ordinazione online.Le caratteristiche tipiche di un carrello virtuale sono l'aggiunta e rimozione di un prodotto, la modifica delle quantità, il calcolo del totale e il salvataggio temporaneo dei prodotti.

\section{Casi d’uso}
Tecnica utilizzata nell'ambito dell'analisi dei requisiti del software per descrivere interazioni o scenari specifici tra gli utenti (attori) e un sistema software. I casi d'uso sono spesso documentati tramite diagrammi UML (Unified Modeling Language), che forniscono una rappresentazione visuale delle interazioni tra attori e sistema.

\section{Chat}
Interazione testuale o multimediale in tempo reale tra due o più persone attraverso un'applicazione o una piattaforma online dedicata. Facilita la comunicazione immediata e sincrona, indipendentemente dalla distanza geografica.

\section{Clienti}
Individuo o un'entità che utilizza un sito web, un'applicazione mobile o una piattaforma digitale per effettuare acquisti di beni o servizi forniti da un'azienda o un'organizzazione tramite il canale online.

\section{Consuntivo}: DA DEFINIRE;
\section{Conto}: DA DEFINIRE;
\section{Design}: DA DEFINIRE;
\section{Desktop}: DA DEFINIRE;
\section{Express}: DA DEFINIRE;
\section{Front-end}: DA DEFINIRE;
\section{Ingredienti}: DA DEFINIRE;
\section{Issue}: DA DEFINIRE;
\section{ITS (Issue Tracking System)}: DA DEFINIRE;
\section{LaTeX}: DA DEFINIRE;
\section{Mobile}: DA DEFINIRE;
\section{Modello a V}: DA DEFINIRE;
\section{NextJs}: DA DEFINIRE;
\section{Ordinazione}: DA DEFINIRE;
\section{Pietanze}: DA DEFINIRE;
\section{PoC (Proof of Concept)}: DA DEFINIRE;
\section{Prenotazione}: DA DEFINIRE;
\section{Prenotazioni}: DA DEFINIRE;
\section{Profili}: DA DEFINIRE;
\section{Proponente}: DA DEFINIRE;
\section{Pull request}: DA DEFINIRE;
\section{Requisiti}: DA DEFINIRE;
\section{Requisiti funzionali}: DA DEFINIRE;
\section{Ristoratore}: DA DEFINIRE;
\section{Ristoratori}: DA DEFINIRE;
\section{RTB}: DA DEFINIRE;
\section{SaaS (Software as a Service)}: DA DEFINIRE;
\section{Telegram}: DA DEFINIRE;
\section{UML}: DA DEFINIRE;
\section{Way of Working}: DA DEFINIRE;
\section{Webapp}: DA DEFINIRE;
\section{WhatsApp}: DA DEFINIRE;
