\section{Introduzione e scopo del documento}

    \subsection{Scopo del documento}

    Il presente documento si pone lo scopo di individuare e definire le best practices$^{G}$ e il Way of Working$^{G}$ del progetto che ogni componente del gruppo SWEet16
    si impegna a rispettare durante l’intero svolgimento del progetto Easy Meal. \\
    In questo modo si cercherà di garantire omogeneità e coesione in ogni aspetto del suddetto progetto.
  
    \subsection{Scopo del prodotto}

    Lo scopo dell’applicazione è quello di creare una piattaforma che permetta di gestire e semplificare il processo di prenotazione$^{G}$ di tavoli all’interno dei ristoranti. \\
    Sarà inoltre possibile anticipare l’esperienza culinaria visionando prima il menù ed andando ad effettuare la propria ordinazione$^{G}$ prima di arrivare al ristorante. \\
    Il prodotto offre inoltre un’esperienza di ordinazione delle pietanze$^{G}$ collaborativa e coinvolgente, permettendo di condividerla con amici ed, in caso di dubbi, interagire direttamente con lo staff del ristorante.

    L’idea è una piattaforma SaaS$^{G}$ (Software as a Service), in cui i saranno presenti due tipi di utenti:
    \begin{itemize}
        \item Clienti$^{G}$: utente registrato all’interno dell’applicazione, può cercare ristoranti, effettuare prenotazioni, ordinazioni e inserire feedback e recensioni;
        \item Ristoratori$^{G}$: utente registrato all’interno dell’applicazione, può gestire uno o più ristoranti, controllando le prenotazioni e le ordinazioni dei clienti ed i menù del/i ristorante/i.
    \end{itemize}


    La piattaforma dovrà essere  disponibile attraverso una WebApp$^{G}$ accessibile da qualsiasi dispositivo oppure tramite dispositivo mobile, attraverso dispositivo iOS$^{G}$ oppure Android$^{G}$.

    \subsection{Glossario}

    Al fine di evitare possibili ambiguità o incomprensioni riguardanti la terminologia usata nel documento, è stato deciso di adottare un glossario in cui vengono riportate le varie definizioni.  \\
    In questa maniera in esso verranno posti tutti i termini specifici del dominio d’uso con relativi significati. \\
    La presenza di un termine all’interno del glossario viene indicata applicando una " $^{G}$ " ad apice della parola.


    \subsection{Maturità del documento}

    \subsection{Riferimenti}

        \subsubsection{Riferimenti normativi}

        \subsubsection{Riferimenti informativi}
