\section{Introduzione e scopo del documento}

    \subsection{Scopo del documento}

    Il presente documento si pone lo scopo di individuare e definire le \emph{best practices}$^{G}$ e il \emph{Way of Working}$^{G}$ del progetto che ogni componente del gruppo SWEet16
    si impegna a rispettare durante l’intero svolgimento del progetto Easy Meal. \\
    In questo modo si cercherà di garantire omogeneità e coesione in ogni aspetto del suddetto progetto.

    \subsection{Scopo del prodotto}

    Lo scopo dell’applicazione è quello di creare una piattaforma che permetta di gestire e semplificare il processo di \emph{prenotazione}$^{G}$ di tavoli all’interno dei ristoranti. \\
    Sarà inoltre possibile anticipare l’esperienza culinaria visionando prima il menù ed andando ad effettuare la propria \emph{ordinazione}$^{G}$ prima di arrivare al ristorante. \\
    Il prodotto offre inoltre un’esperienza di ordinazione delle \emph{pietanze}$^{G}$ collaborativa e coinvolgente, permettendo di condividerla con amici ed, in caso di dubbi, interagire direttamente con lo staff del ristorante.

    L’idea è una piattaforma \emph{SaaS (Software as a Service)}$^{G}$, in cui i saranno presenti due tipi di utenti:
    \begin{itemize}
      \item \emph{Clienti}$^{G}$: Utente registrato all’interno dell’applicazione, può cercare ristoranti, effettuare prenotazioni, ordinazioni e inserire feedback e recensioni;
      \item \emph{Ristoratori}$^{G}$: Utente registrato all’interno dell’applicazione, può gestire uno o più ristoranti, controllando le prenotazioni e le ordinazioni dei clienti ed i menù del/i ristorante/i.
    \end{itemize}


    La piattaforma dovrà essere disponibile attraverso una \emph{Webapp}$^{G}$ accessibile da qualsiasi dispositivo, esso sia \emph{Desktop}$^{G}$ o \emph{Mobile}$^{G}$.


    \subsection{Glossario}

    Al fine di evitare possibili ambiguità o incomprensioni riguardanti la terminologia usata nel documento, è stato deciso di adottare un glossario in cui vengono riportate le varie definizioni. \\
    In questa maniera in esso verranno posti tutti i termini specifici del dominio d’uso con relativi significati. \\
    La presenza di un termine all’interno del glossario viene indicata applicando una " $^{G}$ " ad apice della parola.


    \subsection{Maturità del documento}

    Il presente documento è redatto con un approccio incrementale al fine di poter trattare nuove o ricorrenti questioni in modo rapido ed efficiente, sulla base di decisioni concordate tra tutti i membri del gruppo. \\
    Non può pertanto essere considerato definitivo nella sua attuale versione.

    \subsection{Riferimenti}

        \subsubsection{Riferimenti normativi}

        \begin{itemize}
            \item Regolamento del progetto didattico: \\
            \url{https://www.math.unipd.it/~tullio/IS-1/2023/Dispense/PD2.pdf}
          \item \emph{Capitolato d’appalto}$^{G}$ C3 - Easy Meal: \\
            \url{https://www.math.unipd.it/~tullio/IS-1/2023/Progetto/C3.pdf}
        \end{itemize}

        \subsubsection{Riferimenti informativi}

        \begin{itemize}
            \item I processi di ciclo di vita del software: \\
            \url{https://www.math.unipd.it/~tullio/IS-1/2023/Dispense/T2.pdf}
            \item Glossario: \\
            \url{} %aggiungere link a glossario
        \end{itemize}

        Tutti i riferimenti (normativi e informativi) a risorse web soggette a variazione sono stati consultati il 2024/02/19.
