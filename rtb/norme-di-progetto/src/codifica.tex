\subsection{Codifica}
Superata la fase di Progettazione si passa alla fase di Codifica dove i programmatori
realizzano in concreto ciò che è stato deciso nella fase precedente.
Tale attività viene svolta seguendo le norme e le convenzioni descritte qui sotto allo
scopo di garantare codice di qualità che soddisfi le richieste del proponente e sia mantenibile nel tempo.

\subsubsection{Strumenti utilizzati}
    \begin{itemize}
    \item \textbf{Git:} Strumento di versionamento per lo sviluppo incrementale e asincrono del codice;
    \begin{center}
      \url{https://git-scm.com/}
    \end{center}
    \item \textbf{GitHub:} Piattaforma di condivisione del codice dotata di ITS per l'assegnazione dei compiti;
    \begin{center}
      \url{https://github.com/}
    \end{center}
    \item \textbf{Docker:} Strumento di sviluppo del software che garantisce delle fondamenta sempre uguali
      e comuni a tutti i membri del gruppo indipendentemente dal sistema operativo e dalle librerie installate. \\
      Viene utilizzato nella fase di deployment$^{G}$ del software;
      \begin{center}
        \url{https://www.docker.com/}
      \end{center}
    \item \textbf{Virtual Studio Code:} IDE$^{G}$ multipiattaforma versatile e gratuito.
    \begin{center}
      \url{https://code.visualstudio.com}
    \end{center}
    \end{itemize}
