\nonstopmode
\pagebreak % aggiunto per riuscire a caricare la pagina
\subsection{Gestione della Qualità}

\subsubsection{Descrizione, scopo ed aspettative}
La gestione della qualità di progetto è l'insieme dei processi e delle attività che vengono eseguite per garantire che la qualità del prodotto finale
segua standard e metriche rigorose descritte all'interno del Piano di Qualifica.

I suoi obiettivi sono:

\begin{itemize}
\item Comprendere, valutare e gestire le aspettative in modo tale che tutti i requisiti del proponente vengano rispettati;
\item Riuscire ad impostare chiare metriche di qualità e documentare tutte le procedure necessarie al completamento del progetto secondo le aspettative richieste;
\item Sviluppare il prodotto seguendo queste metriche, così da consegnarlo entro tempistiche prestabilite, rispettando il budget scelto e secondo i requisiti e le aspettative del proponente.
\end{itemize}
\subsubsection{Piano di Qualifica}

Per rispettare tutti gli obbiettivi di questo processo si utilizza il Piano di Qualifica, un documento dove vengono descritte le metriche e le modalità utilizzate per valutare la qualità dei prodotti e dei processi,
al cui interno sono presenti:

\begin{itemize}
\item Definizione delle metriche per analizzare la qualità del prodotto;
\item Definizione di un sistema per il controllo della qualità durante tutto il ciclo di vita del progetto;
\item Definizione di un piano di miglioramento per analizzare le prestazioni di qualità e identificare attività per migliorare le stesse.
\end{itemize}

\subsubsection{Denominazione metriche}

Le metriche utilizzate vengono definite come segue:

\begin{center}
    \large{\textbf{M[Tipologia].[Titolo]}}
\end{center}

Dove:

\begin{itemize}
\item Tipologia:
\begin{itemize}
\item PC: Per processo;
\item PD: Per prodotto.
\end{itemize}
\item Titolo: Nome della metrica utilizzata.
\end{itemize}

%% prima bozza della denominazione delle metriche, cambiare in seguito al completamento del PdQ
