\pagebreak
\subsection{Verifica}
\subsubsection{Scopo aspettative e descrizione}
L'obiettivo è stabilire il metodo di verifica per garantire l'assenza
di errori durante lo sviluppo del prodotto, la redazione della documentazione
e il rispetto dei requisiti specificati.
Le aspettative del gruppo \emph{SWEet16} riguardo all'utilizzo di questo processo includono:
\begin{itemize}
    \item Verificare ogni fase seguendo criteri definiti, coerenti e adattabili in caso di necessità;
    \item Condurre una verifica attenta per garantire il successo durante la fase di validazione;
    \item Automatizzare il più possibile le attività svolte durante il processo di verifica;
    \item Rispettare gli obiettivi di copertura indicati nel piano di qualifica.
\end{itemize}
Il processo di verifica si svolge in due forme:
\begin{itemize}
    \item \textbf{Analisi statica:} Non richiede l'esecuzione dell'oggetto di verifica;
    \item \textbf{Analisi dinamica:} Richiede l'esecuzione dell'oggetto di verifica.
\end{itemize}
\subsubsection{Analisi statica}
L'analisi statica implica l'esame del codice prima della sua esecuzione e assicura che il software soddisfi i requisiti
e le specifiche indicate. \\
Poiché non richiede l'esecuzione dell'oggetto in esame, questo tipo di analisi non si applica solo al codice, ma anche alla documentazione.
L'analisi statica comprende due tecniche principali:
\begin{itemize}
    \item \textbf{Walkthrough}: Questa tecnica coinvolge una revisione generale alla ricerca di errori senza presupposti specifici. Si articola nelle seguenti attività:
    \begin{itemize}
        \item Pianificazione;
        \item Lettura;
        \item Discussione;
        \item Correzione degli errori.
    \end{itemize}
    \item \textbf{Inspection}: Questa tecnica comporta una revisione mirata alla ricerca di errori conosciuti e si suddivide nelle seguenti attività:
    \begin{itemize}
        \item Pianificazione;
        \item Definizione di una lista di controllo;
        \item Lettura;
        \item Correzione degli errori.
    \end{itemize}
\end{itemize}
\subsubsection{Analisi dinamica}
L'analisi dinamica è applicabile principalmente al prodotto software
poiché implica l'esecuzione di test, cioè prove sul codice in esecuzione.
Un test ben definito deve:
\begin{itemize}
    \item \textbf{Essere ripetibile:} Dato un determinato input, si deve ottenere sempre lo stesso output per ogni prova eseguita;
    \item Specificare l'ambiente di esecuzione;
    \item Identificare input e output richiesti;
    \item Fornire informazioni utili sui risultati dell'esecuzione.
\end{itemize}
L'automatizzazione sarà realizzata mediante strumenti dedicati non ancora definiti dal gruppo.
\subsubsection{Verifica della documentazione}
La revisione della documentazione comprende:
\begin{itemize}
    \item Verifica dell'ortografia e della grammatica;
    \item Accertamento dell'adeguato rispetto delle norme tipografiche e di formattazione stabilite, descritte nella sezione 3.1.6 di questo documento;
    \item Valutazione della coerenza e della rilevanza dei contenuti redatti.
\end{itemize}
