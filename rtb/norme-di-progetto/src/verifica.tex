
\subsection{Verifica}
\subsubsection{Scopo}
L'obiettivo è stabilire il metodo di verifica per garantire l'assenza
di errori durante lo sviluppo del prodotto, la redazione della documentazione
e il rispetto dei requisiti specificati
\subsubsection{Aspettative}
Le aspettative del gruppo \emph{SWEet16} riguardo all'utilizzo di questo processo includono:
\begin{itemize}
    \item Verificare ogni fase seguendo criteri definiti, coerenti e adattabili in caso di necessità;
    \item Condurre una verifica attenta per garantire il successo durante la fase di validazione;
    \item Automatizzare il più possibile le attività svolte durante il processo di verifica;
    \item Rispettare gli obiettivi di copertura indicati nel \emph{piano di qualifica}.
\end{itemize}
\subsubsection{Descrizione}
Il processo di verifica si svolge in due forme:
\begin{itemize}
    \item \textbf{Analisi statica:} non richiede l'esecuzione dell'oggetto di verifica;
    \item \textbf{Analisi dinamica: } richiede l'esecuzione dell'oggetto di verifica.
\end{itemize}
\subsubsection{Analisi statica}
L'analisi statica si applica a tutti i processi attivi nel progetto e comprende due tecniche principali:
\begin{itemize}
    \item \textbf{Walkthrough}: questa tecnica coinvolge una revisione generale alla ricerca di errori
    senza presupposti specifici. Si articola nelle seguenti attività:
    \begin{itemize}
        \item Pianificazione
        \item Lettura
        \item Discussione
        \item Correzione degli errori
    \end{itemize}
    \item \textbf{Inspection}: questa tecnica comporta una revisione mirata alla ricerca di errori conosciuti
    e si suddivide nelle seguenti attività:
    \begin{itemize}
        \item Pianificazione
        \item Definizione di una lista di controllo
        \item Lettura
        \item Correzione degli errori
    \end{itemize}
\end{itemize}
\subsubsection{Analisi dinamica}

\subsubsubsection{Verifica della documentazione}
