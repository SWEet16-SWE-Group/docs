\subsection{Documentazione}

\subsubsection{Descrizione, scopo e aspettative}

\subsubsection{Ciclo di vita del documento}

\subsubsection{Template}
È stato deciso di utilizzare un template LaTeX % TODO marcare latex con il glossario e con il suo font
come base di partenza per la stesura di ogni documento.

\subsubsection{Documenti prodotti}

    \subsubsubsection{Formali}

    \subsubsubsection{Informali}

\subsubsection{Struttura del documento}

        \subsubsubsection{Prima pagina}

        \subsubsubsection{Registro delle modifiche - Changelog}

        \subsubsubsection{Indice}

        %\subsubsubsection{Piè di Pagina} si può pensare di aggiungere un piè di pagina ai documenti

        \subsubsubsection{Verbali}

\subsubsection{Norme tipografiche}

    \subsubsubsection{Nome del file}

    \subsubsubsection{Stile del testo}

    \subsubsubsection{Glossario}

    \subsubsubsection{Altre norme tipografiche}

\subsubsection{Elementi grafici}

    \subsubsubsection{Immagini}

    \subsubsubsection{Tabelle}

\subsubsection{Strumenti}

    \subsubsubsection{GitHub}
    

    \subsubsubsection{LaTeX WorkShop}
    Estensione per VSCode e fork vari che solleva dal redattore l'impegno di compilare il documento, concendo un workflow più simile
    a quello what you see is what get ma con il vantaggio di rimanere compilato da sorgente.
    L'estensione è poi configurata in sinergia con Git (quindi anche GitHub) per ignorare la presenza di eventuali artefatti binari
    risultati dalla compilazione.

    \subsubsubsection{UML}
