\subsection{Documentazione}

\subsubsection{Scopo}
Lo scopo di questo documento è fornire uno standard da seguire durante il processo di documentazione.

\subsubsection{Ciclo di vita del documento}
Tappe fondamentali del ciclo di vita di ogni documento sono:
  \begin{itemize}
    \item Creazione: Creazione del documento partendo da un template.
    \item Strutturazione: Creazione di file distinti rappresentanti le singole sezioni dell'indice dei contenuti.
    \item Stesura: Fase di scrittura dei contenuti nei documenti. Può essere fatta da uno o più redattori, lavorando
      asincronicamente su diverse sezioni.
    \item Verifica: Una volta che una sezione viene completata entra in fase di verifica, dove uno o più verificatori
      confermeranno le modifiche o rilasceranno dei commenti di correzione, in questo secondo caso il documento torna
      nella fase di stesura.
    \item Approvazione: Rilascio del documento.
  \end{itemize}
  Le prime due fasi sono svolte da tutti i membri del gruppo mentre la terza è affidata esclusivamente al responsabile.

\subsubsection{Template}
È stato deciso di utilizzare un template LaTeX % TODO marcare latex con il glossario e con il suo font
come base di partenza per la stesura di ogni documento.

\subsubsection{Documenti prodotti}
I documenti prodotti si possono suddividere in due macrosezioni.

    \subsubsubsection{Esterni}
    Di interesse principale per il proponente e i committenti.
    \begin{itemize}
      \item Analisi dei requisiti.
      \item Piano di progetto.
      \item Piano di qualifica.
      \item Verbali esterni.
    \end{itemize}

    \subsubsubsection{Interni}
    Di interesse principale solo per membri del gruppo.
    \begin{itemize}
      \item Norme di progetto.
      \item Verbali interni.
    \end{itemize}


\subsubsection{Struttura del documento}

        \subsubsubsection{Prima pagina}
        La prima pagina di ogni documento dovrà contenere:
        \begin{itemize}
          \item Logo di Unipd accompagnato dal nome dell'università, corso di laurea, materia e anno accademico.
          \item Logo del gruppo SWEet16 accompagnato dal nome del gruppo e la rispettiva mail.
          \item Titolo del documento.
          \item Informazioni aggiuntive:
          \begin{itemize}
            \item Redattori
            \item Verificicatori
            \item Destinatari
            \item Versione
          \end{itemize}
        \end{itemize}

        % \subsubsubsection{Registro delle modifiche - Changelog}
        % lo fa già github in automatico, non ci serve, possiamo toglierlo

        \subsubsubsection{Indice}
        Ogni documento presenta un indice di navigazione allo scopo di facilitare quest'ultima e
        dare un'aspettativa di ciò che esso contiene.

        %\subsubsubsection{Piè di Pagina} si può pensare di aggiungere un piè di pagina ai documenti

        \subsubsubsection{Verbali}
        I verbali sono documenti informali non troppo diversi dagli altri documenti, con l'unica differenza
        di non essere soggetti a versionamento.
        La prima pagina è la stessa fornita da template.
        Il resto della struttura è divisa in due:
        \begin{itemize}
          \item Partecipanti
          \begin{itemize}
            \item Tabella dei partecipanti che possono essere i da due a tutti i membri del gruppo più il proponente
            \item Orario di inizio e fine
          \end{itemize}
          \item Sintesi ed elaborazione dell'incontro
        \end{itemize}

\subsubsection{Norme tipografiche}

    \subsubsubsection{Nome del file}
    Il nome di ogni file inizia con la prima lettera maiuscola, tutte le altre minuscole.
    Segue poi dal numero di versione composto da una 'v' e tre numeri separati da un '.'.
    Per i verbali il nome del documento sarà semplicemente la data cui si è tenuto nel formato
    AAAA/MM/GG.

    \subsubsubsection{Stile del testo}
    \begin{itemize}
    % TODO da riempire in un secondo momento
    \item TODO
    \end{itemize}

    \subsubsubsection{Glossario}
    % TODO da ricontrollare anche questo che ho visto c'è anche nell'introduzione
    Il Glossario è un documento che contiene tutte le parole ritenute ambigue date il contesto del progetto.

    I termini presenti sono separati in sezioni indicanti la loro prima lettera, in ordine alfabetico, affiancati
    da una breve descrizione del significato con cui sono usati e eventuali sinonimi.
    La definizione di questi ultimi non ne sarà riportata all'interno.

    I termini e i loro eventuali sinonimi riporteranno una 'G' in apice all'interno dei documenti. La G sarà
    presente solo nella prima occorrenza di quella parola in quel documento.

    \subsubsubsection{Altre norme tipografiche}
    \begin{itemize}
    \item Data in formato AAAA/MM/GG
    % TODO aggiungere altre se ci sono
    \end{itemize}

\subsubsection{Elementi grafici}

    \subsubsubsection{Immagini}
    Le immagini vanno centrate orizzontalmente nella pagina e devono rispettare i margini previsti dal foglio qual ora
    fossero molto grandi.
    Grafici, diagrammi o schemi contano come immagini.

    \subsubsubsection{Tabelle}
    Le tabelle devono occupare tutta la larghezza del foglio indipendentemente dal numero di colonne.
    Inoltre per tabelle troppo grandi, queste vanno spezzate su più pagine e l'intestazione deve essere ripetuta
    per ognuno dei seguenti pezzi.
    Le righe devono essere a colori alternati per facilitarne la lettura.

\subsubsection{Strumenti}

    \subsubsubsection{GitHub}
    Piattaforma di condivisione di codice, già utilizzata per la produzione del software, si presta
    molto bene anche per la produzione della documentazione data la scelta di utilizzare LaTex.
    L'ITS e il meccanismo di versionamento permettono di tener traccia delle modifiche apportate nel tempo
    in modo automatico e con un enorme quantità di informazioni utili.
    % ^ praticamente la stessa cosa del registro delle modifiche ma per i sigma gigachad

    \subsubsubsection{LaTeX WorkShop}
    Estensione per VSCode e fork vari che solleva dal redattore l'impegno di compilare il documento, concedendo un workflow più simile
    a quello what you see is what get ma con il vantaggio di rimanere compilato da sorgente.
    L'estensione è poi configurata in sinergia con Git (quindi anche GitHub) per ignorare la presenza di eventuali artefatti binari
    risultati dalla compilazione.

    \subsubsubsection{StarUML}
    Applicazione desktop scelta per la produzione di diagrammi UML (sia delle classi e che dei casi d'uso) dato il suo essere
    multipiattaforma.
