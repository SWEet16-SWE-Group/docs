\subsection{Documentazione}

\subsubsection{Descrizione, scopo e aspettative}

\subsubsection{Ciclo di vita del documento}

\subsubsection{Template}
È stato deciso di utilizzare un template LaTeX % TODO marcare latex con il glossario e con il suo font
come base di partenza per la stesura di ogni documento.

\subsubsection{Documenti prodotti}

    \subsubsubsection{Formali}

    \subsubsubsection{Informali}

\subsubsection{Struttura del documento}

        \subsubsubsection{Prima pagina}

        \subsubsubsection{Registro delle modifiche - Changelog}

        \subsubsubsection{Indice}
        Ogni documento presenta un indice di navigazione allo scopo di facilitare quest'ultima e
        dare un'aspettativa di ciò che esso contiene.

        %\subsubsubsection{Piè di Pagina} si può pensare di aggiungere un piè di pagina ai documenti

        \subsubsubsection{Verbali}

\subsubsection{Norme tipografiche}

    \subsubsubsection{Nome del file}
    Il nome di ogni file inizia con la prima lettera maiuscola, tutte le altre minuscole.
    Segue poi un carattere di underscore (_) con il numero di versione nel formato: v.<NUM>.<NUM>.<NUM> .
    Per i verbali il nome del documento sarà semplicemente la data cui si è tenuto, con la data nel formato
    AAAA_MM_GG .

    \subsubsubsection{Stile del testo}
    % TODO da fare in secondo momento

    \subsubsubsection{Glossario}
    Il Glossario è un documento che contiene tutte le parole ritenute ambigue date il contesto del progetto.

    I termini presenti sono separati in sezioni indicanti la loro prima lettera, in ordine alfabetico, affiancati
    da una breve descrizione del significato con cui sono usati e eventuali sinonimi.
    La definizione di questi ultimi non ne sarà riportata all'interno.

    I termini e i loro eventuali sinonimi riporteranno una 'G' in apice all'interno dei documenti. La G sarà
    presente solo nella prima occorrenza di quella parola in quel documento.

    \subsubsubsection{Altre norme tipografiche}
    \begin{itemize}
    \item Data in formato ANNO/MM/GG
    % TODO aggiungere altre se ci sono
    \end{itemize}

\subsubsection{Elementi grafici}

    \subsubsubsection{Immagini}
    Le immagini vanno centrate orizzontalmente nella pagina e devono rispettare i margini previsti dal foglio qual'ora
    fossero molto grandi.
    Grafici, diagrammi o schemi contano come immagini.

    \subsubsubsection{Tabelle}
    Le tabelle devono occupare tutta la larghezza del foglio indipendentemente dal numero di colonne.
    Inoltre per tabelle troppo grandi, queste vanno spezzate su più pagine e l'intestazione deve essere ripetuta
    per ognuno dei seguenti pezzi.
    Le righe devono essere a colori alternati per facilitarne la lettura.

\subsubsection{Strumenti}

    \subsubsubsection{GitHub}
    Piattaforma di condivisione del codice di un software dotata di ITS. Già utilizzata per la produzione del software, si presta
    molto bene anche per la produzione della documentazione data la scelta di utilizzare LaTex.

    \subsubsubsection{LaTeX WorkShop}
    Estensione per VSCode e fork vari che solleva dal redattore l'impegno di compilare il documento, concendo un workflow più simile
    a quello what you see is what get ma con il vantaggio di rimanere compilato da sorgente.
    L'estensione è poi configurata in sinergia con Git (quindi anche GitHub) per ignorare la presenza di eventuali artefatti binari
    risultati dalla compilazione.

    \subsubsubsection{StarUML}
    Applicazione desktop scelta per la produzione di diagrammi UML (sia delle classi e che dei casi d'uso) dato il suo essere
    multipiattaforma.
