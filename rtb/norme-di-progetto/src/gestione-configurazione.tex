\pagebreak

\subsection{Gestione della Configurazione}

\subsubsection{Descrizione, scopo ed aspettative}
    \subsubsubsection{Scopo}
    Lo scopo è di supervisionare e regolare in modo organizzato la creazione di 
    documenti e codice. Ogni elemento soggetto a configurazione sarà soggetto a versionamento
    e controllo delle modifiche, al fine di assicurare l'integrità del prodotto 
    nel tempo.
    \subsubsubsection{Descrizione}
    Tutti gli strumenti dedicati alla configurazione utilizzati per la produzione
    di documenti e codice vengono raccolti, organizzati e coordinati. Questo include
    la gestione della struttura e della disposizione dei file all'interno del repository, 
    nonché gli strumenti per il versionamento e il coordinamento.

\subsubsection{Versionamento}
Ogni modifica apportata a un documento genera una nuova versione seguendo il formato
\textbf{[X].[Y].[Z]}, dove:
\begin{itemize}
    \item \textbf{X}: rappresenta la versione approvata dal \emph{Responsabile}, l'unico autorizzato ad incrementarla;
    \item \textbf{Y}: rappresenta la versione approvata dal \emph{Verificatore}, l'unico autorizzato ad incrementarla;
    \item \textbf{Z}: rappresenta la versione dell'ultima modifica.
\end{itemize}
Ogni parte del codice di versione inizia da 0 e ritorna a 0 ogni volta che la componente alla sua sinistra
viene incrementata
\subsubsection{Repository}
    \subsubsubsection{Tecnologie}
    Per il progetto viene usato il sistema di versionamento \emph{Git}, nello specifico con 
    il servizio \emph{GitHub}. Per documenti interni temporanei, invece, è stato utilizzato il servizio 
    \emph{Google Documents}.
    \subsubsubsection{Struttura Repository}
    Il repository contiene il codice sorgente per ogni documento dentro una cartella denominata
    NomeDocumento. Il repository è suddiviso in più branch:
    \begin{itemize}
        \item \textbf{main}: il branch principale che contiene l'ultima versione dei documenti
        \item \textbf{poc-main}: contiene il codice sorgente del \emph{PoC}
        \item \textbf{rtb-documentazione-main}: il branch per la modifica e la creazione di documenti
    \end{itemize}
    \subsubsubsection{Modifiche alla repository}
    Le modifiche ai vari documenti non vanno fatte direttamente nel branch \emph{rtb-documentazione-main},
    poiché porterebbe ad un elevato rischio di incongruenze e conflitti. Per ogni sezione di un documento, vengono
    creati appositi branch dove è possibile apportare modifiche. Successivamente, è necessario 
    effettuare una \textbf{pull request} con verifica obbligatoria per integrare le modifiche nel branch 
    \emph{rtb-documentazione-main}.