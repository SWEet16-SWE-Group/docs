\subsection{Sviluppo}

\subsubsection{Descrizione, scopo ed aspettative}

Il processo di sviluppo contiene le attività e i compiti dello sviluppatore, tra cui le attività per l’Analisi dei Requisiti, la progettazione, la codifica ed i strumenti utilizzati.

Lo scopo del processo di sviluppo è quello di descrivere i compiti e le attività da svolgere per la codifica del prodotto software richiesto. \\
In questa sezione vengono dunque descritte le attività, le norme e le convenzioni adottate per questo processo.

Le aspettative per una corretta applicazione del processo di sviluppo sono:
\begin{itemize}
    \item Realizzare un prodotto finale conforme alle richieste del proponente, come dettagliato dall'Analisi dei Requisiti;
    \item Determinare i vincoli tecnologici;
    \item Determinare gli obiettivi di sviluppo;
    \item Determinare i vincoli di design$^{G}$;
\end{itemize}
\subsubsection{Analisi dei Requisiti}


L’analisi dei requisiti è l’attività preliminare che permette di individuare, a partire da un approfondito studio del capitolato$^{G}$, i requisiti diretti ed indiretti, 
impliciti ed espliciti che il proponente richiede per la realizzazione del prodotto, ed i vari casi d’uso$^{G}$ del prodotto stesso. \\
In questa attività è importante suddividere il problema iniziale in requisiti$^{G}$ quanto più elementari possibile, andando a facilitare il lavoro durante la fase di sviluppo.

    \subsubsubsection{Requisiti}

    \subsubsubsection{Denominazione e Legenda}

    \subsubsubsubsection{Struttura Casi d'Uso}

    \subsubsubsubsection{Denominazione Casi d'Uso}

    \subsubsubsubsection{Struttura dei Requisiti}

    \subsubsubsubsection{Denominazione dei Requisiti}
 