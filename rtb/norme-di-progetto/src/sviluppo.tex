\subsection{Sviluppo}

\subsubsection{Descrizione, scopo ed aspettative}

Il processo di sviluppo contiene le attività e i compiti dello sviluppatore, tra cui le attività per l’Analisi dei Requisiti, la progettazione, la codifica ed i strumenti utilizzati.

Lo scopo del processo di sviluppo è quello di descrivere i compiti e le attività da svolgere per la codifica del prodotto software richiesto. \\
In questa sezione vengono dunque descritte le attività, le norme e le convenzioni adottate per questo processo.

Le aspettative per una corretta applicazione del processo di sviluppo sono:
\begin{itemize}
    \item Realizzare un prodotto finale conforme alle richieste del proponente, come dettagliato dall'Analisi dei Requisiti;
    \item Determinare i vincoli tecnologici;
    \item Determinare gli obiettivi di sviluppo;
    \item Determinare i vincoli di design$^{G}$;
\end{itemize}
\subsubsection{Analisi dei Requisiti}


L’Analisi dei Requisiti è l’attività preliminare che permette di definire chiaramente, grazie all'utilizzo di Analisti, i requisiti diretti ed indiretti, 
impliciti ed espliciti che il proponente richiede per la realizzazione del prodotto, ed i vari casi d’uso$^{G}$ del prodotto stesso. \\
In questa attività è importante suddividere il problema iniziale in requisiti$^{G}$ quanto più elementari possibile, andando a facilitare il lavoro durante la fase di sviluppo.

Al fine di poter fornire una corretta intrepretazione e realizzazione del prodotto, l'attività ha lo scopo di comprendere le specificità del capitolato$^{G}$, sulla base di un confronto mirato con il proponente, interpretando e ampliando la relativa realizzazione.
\pagebreak
    \subsubsubsection{Requisiti}

    I requisiti vengono raccolti da diverse fonti quali:
    \begin{itemize}
        \item Lettura dettagliata del capitolato$^{G}$;
        \item Confronto interno tra i membri del gruppo;
        \item Confronto con il proponente;
        \item Analisi dei casi d’uso$^{G}$.
    \end{itemize}

    \subsubsubsection{Denominazione e Legenda}

    \subsubsubsubsection{Struttura Casi d'Uso}

    I casi d’uso esprimono un comportamento od un modo di utilizzare il prodotto. \\
    Vengono descritti graficamente mediante l’ausilio di diagrammi UML$^{G}$. 

    Ciascun caso d’uso è costituito da:
    \begin{itemize}
        \item Codice identificativo;
        \item Attore Primario;
        \item Precondizioni;
        \item Postcondizioni;
        \item Scenario principale;
        \item Generalizzazione (se esistono);
        \item Estensioni (se esistono).
        
    \end{itemize}

    \subsubsubsubsection{Denominazione Casi d'Uso}

    Ciascun caso d’uso viene classificato univocamente mediante l’utilizzo del seguente schema:
    
    \subsubsubsubsection{Struttura dei Requisiti}

    \subsubsubsubsection{Denominazione dei Requisiti}
 