\subsection{Fornitura}

\subsubsection{Descrizione, scopo ed aspettative}

Il processo di fornitura determina ogni compito, attività e risorsa necessaria al corretto svolgimento del progetto. \\
Tale processo verrà avviato solo in seguito ad un’attenta analisi preliminare delle richieste del proponente$^{G}$, seguito da uno studio di fattibilità delle menzionate
richieste e concluso dalla definizione di un accordo contrattuale con la proponente. \\

Questa sezione ha lo scopo di elencare tutte le metriche, gli strumenti e i documenti utilizzati al fine di realizzare il processo di fornitura.

Le aspettative sono:
\begin{itemize}
    \item Avere una chiara struttura dei documenti;
    \item Definire in modo chiaro le tempistiche di lavoro;
    \item Chiarire eventuali dubbi e stabilire vincoli con il proponente.
\end{itemize}
\subsubsection{Proponente}

Il team si è accordato con il proponente per avere un continuo riscontro sul prosieguo del progetto,  organizzando periodicamente incontri e mantenendosi in contatto asincrono tramite un gruppo Telegram$^{G}$. \\
Questo ci permetterà di poter chiarire velocemente eventuali dubbi, ed avere un riscontro sulla documentazione redatta, oltre che alla verifica dei verbali esterni.

\subsubsection{Documenti}

    \subsubsubsection{Piano di Progetto}

    \subsubsubsection{Piano di Qualifica}

%Aggiungere sezione strumenti?