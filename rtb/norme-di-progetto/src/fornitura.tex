\nonstopmode
\subsection{Fornitura}

\subsubsection{Descrizione, scopo ed aspettative}

Il processo di fornitura determina ogni compito, attività e risorsa necessaria al corretto svolgimento del progetto. \\
Tale processo verrà avviato solo in seguito ad un’attenta analisi preliminare delle richieste del \emph{proponente}$^{G}$, seguito da uno studio di fattibilità delle menzionate
richieste e concluso dalla definizione di un accordo contrattuale con il proponente. \\

Questa sezione ha lo scopo di elencare tutte le metriche, gli strumenti e i documenti utilizzati al fine di realizzare il processo di fornitura.

Le aspettative sono:

\begin{itemize}
\item Avere una chiara struttura dei documenti;
\item Definire in modo chiaro le tempistiche di lavoro;
\item Chiarire eventuali dubbi e stabilire vincoli con il proponente.
\end{itemize}
\subsubsection{Proponente}

Il team si è accordato con l'azienda proponente Imola Informatica per avere un continuo riscontro sul prosieguo del progetto, mantenendosi in contatto asincrono tramite un gruppo \emph{Telegram}$^{G}$. \\
Questo ci permetterà di poter chiarire velocemente eventuali dubbi, ed avere un riscontro sulla documentazione redatta, oltre che alla verifica dei verbali esterni.

Sono inoltre programmati degli incontri sincroni a cadenza variabile, a seguito di una richiesta da parte del gruppo o dall'azienda, utili per chiarimenti sui vincoli o necessità del \emph{capitolato}$^{G}$, oltre che per avere un feedback su quanto prodotto fino a quel momento.

\subsubsection{Documentazione}

Vengono di seguito elencati i documenti che il gruppo SWEet16 consegnerà all'azienda proponente \textit{Imola Informatica} e ai committenti \textit{Prof. Tullio Vardanega} e \textit{Prof. Riccardo Cardin}.

    \subsubsubsection{Analisi dei Requisiti}

    L'Analisi dei Requisiti è un documento che definisce le funzionalità che il prodotto offre ed i requisiti da soddisfare in modo tale che il software sviluppato sia conforme alle richieste del proponente.

    Viene così suddiviso:

\begin{itemize}
\item Descrizione del prodotto, contenente gli obiettivi del prodotto e le sue funzionalità principali;
\item Casi d'uso, i quali descrivono tutti gli scenari di utilizzo del sistema da parte degli utenti;
\item Lista dei requisiti, cioè tutte le richieste e vincoli definiti dal proponente o ricavati dal gruppo. I requisiti sono divisi in obbligatori, desiderabili o opzionali.
\end{itemize}

    \subsubsubsection{Piano di Progetto}

    Il Piano di Progetto verrà redatto durante tutta la durata del progetto, ed andrà a costituire uno strumento utile alla pianificazione di tutte le attività da svolgere, le risorse necessarie e la scadenza del progetto stesso. \\
    È un documento ufficiale soggetto a versionamento ed approvazione, che viene utilizzato per descrivere in modo chiaro e conciso gli obiettivi di progetto e gli elementi necessari per il loro compimento;

    È suddiviso nelle seguenti parti:

\begin{itemize}
\item Analisi dei rischi: Analisi delle problematiche e difficoltà che potrebbero venire a crearsi durante il progetto e metodologie per prevenire e/o risolvere queste problematiche, in modo da minimizzare questi rallentamenti;
\item Modello di sviluppo: Descrizione dell'approccio metodologico e strutturato utilizzato durante lo sviluppo del prodotto;
\item Pianificazione: Definizione dei periodi, con le relative attività di progetto da svolgere;
\item Preventivo: Riepilogo del prospetto economico ed orario totale, diviso per ogni periodo;
\item Consuntivo di periodo: Tracciamento dell’andamento rispetto al preventivo;
\item Attualizzazione dei rischi.
\end{itemize}

    \subsubsubsection{Piano di Qualifica}

    Il Piano di Qualifica rappresenta i compiti e le attività relative al progetto che dovranno essere svolto dal Verificatore all'interno del progetto,
    utili a garantire qualità al prodotto software che si andrà a sviluppare. \\
    Contiene inoltre le metriche e le misure necessarie a garantire che il prodotto finale sia conforme alle specifiche richieste e alle aspettative del committente. \\

    Nello specifico, esso è formato dalle seguenti parti:

\begin{itemize}
\item Qualità di processo: Definizione dei parametri e delle metriche utili a garantire processi di elevata qualità;
\item Qualità di prodotto: Definizione dei parametri e delle metriche per garantire un prodotto di elevata qualità;
\item Test e specifiche: Descrizione dei test necessari per assicurare che i requisiti stabiliti vengano soddisfatti;
\item Resoconto delle attività di verifica ed eventuali criticità riscontrate.
\end{itemize}

    \subsubsubsection{Glossario}

    Il Glossario è un documento di supporto concepito per i membri del gruppo, ma anche per i committenti e l'azienda proponente. \\
    Permette di evitare ambiguità o confusione riguardanti l'utilizzo di terminologia di dominio utilizzata in tutta la documentazione prodotta dal gruppo.

    \subsubsubsection{Lettera di Presentazione}

    La Lettera di Presentazione è il documento con cui il gruppo \textit{SWEet16} esprime la propria volontà di partecipare alla fase di revisione del prodotto software. \\
    Al suo interno vengono elencati i documenti prodotti e messi a disposizione per la visione ai committenti e all'azienda proponente.
    %

%Aggiungere sezione strumenti?
