\pagebreak
\subsection{Gestione dei Processi}

        \subsubsection{Descrizione, scopo ed aspettative}

        Lo scopo del processo è la stesura del documento denominato Piano di Progetto, utilizzato dal gruppo per l'organizzazione e la gestione dei ruoli.\\
        In particolare le attività previste dal processo organizzativo di gestione dei processi sono:

        \begin{itemize}
            \item Assegnazioni dei ruoli e dei compiti;
            \item Comunicazione interne/esterne;
            \item Incontri interni/esterni;
            \item Strumenti di coordinamento;
            \item Strumenti di versionamento;
            \item Analisi dei Rischi e loro mitigazione.
        \end{itemize}

        Le aspettative sono invece:
        \begin{itemize}
            \item Ottenere un'organizzazione efficace tra i membri del gruppo;
            \item Adottare un buon \textit{Way of Working};
            \item Ottenere un'equa distribuzione dei ruoli, attraverso un'efficace rotazione degli stessi.
        \end{itemize}

        \subsubsection{Ruoli di progetto}

        Il Responsabile di Progetto si occupa di suddividere i ruoli tra i membri del gruppo, garantendo che ognuno di essi assuma nel corso del progetto almeno una volta ogni ruolo.\\
        Vengono di seguito descritti i singoli ruoli.

            \subsubsubsection{Responsabile di progetto}

            Il suo compito consiste nel garantire lo svolgimento delle attività pianificate entro i tempi e le modalità previste dal gruppo. \\
            Si occupa anche delle comunicazioni esterne con committenti e proponenti.

            Le sue responsabilità sono:
            \begin{itemize}
                \item Approvare la documentazione nella fase finale del processo di verifica;
                \item Gestire l'assegnazione dei compiti agli altri membri del gruppo;
                \item Organizzare il lavoro in modo da minimizzare la probabilità che si verifichino problemi.
            \end{itemize}

            \subsubsubsection{Amministratore di progetto}

            È la figura che si occupa della gestione degli strumenti necessari all'ambiente di lavoro, le sue responsabilità sono:
            \begin{itemize}
                \item Gestire il sistema di archiviazione e versionamento di documentazione e codice;
                \item Gestire il sistema di configurazione del prodotto;
                \item Mantenere un ambiente di sviluppo efficiente, fornendo strumenti adeguati ai componenti del gruppo;
                \item Salvaguardare la documentazione di progetto;
                \item Redigere ed attuare i piani e le procedure per la gestione della qualità.
            \end{itemize}

            \subsubsubsection{Analista}

            Il suo compito consiste nell'individuare, analizzare e documentare i servizi che il sistema deve fornire. \\
            Le sue responsabilità sono:
            \begin{itemize}
                \item Studiare il problema ed il relativo contesto applicativo;
                \item Comprendere il problema e definire la complessità ed i requisiti;
                \item Redigere il documento denominato Analisi dei Requisiti.
            \end{itemize}

            \subsubsubsection{Progettista}

            Il suo compito consiste nel definire la struttura architetturale del sistema.
            È responsabile delle attività di progettazione, le quali devono portare alla realizzazione di un prodotto che soddisfa i requisiti individuati dagli analisti.\\
            Le sue responsabilità sono:
            \begin{itemize}
                \item Studiare l'architettura più adatta al prodotto da realizzare;
                \item Garantire la qualità del prodotto;
                \item Identificare l'architettura ad alto livello del sistema e l'architetturaa livello di componenti del sistema.
            \end{itemize}

            \subsubsubsection{Programmatore}

            Il suo compito consiste nella scrittura del codice richiesto dallo svolgimento del progetto. \\
            Le sue responsabilità sono:
            \begin{itemize}
                \item Scrivere codice pulito e facile da mantenere, rispettando le Norme di Progetto;
                \item Realizzare gli strumenti per la verifica e la validazione del software.
            \end{itemize}

            \subsubsubsection{Verificatore}
            Il suo compito consiste nella verifica del lavoro svolto dagli altri componenti del progetto, sulla base delle proprie conoscenze tecniche, esperienza e conoscenza delle norme. \\
            Le sue responsabilità sono:
            \begin{itemize}
                \item Controllare che vengano rispettate le norme di progetto, inserite all'interno di questo documento;
                \item Verificare la conformità dei prodotti ai requisiti funzionali e di qualità;
                \item Segnalare eventuali anomalie riscontrate nella verifica in modo che vengano corrette tempestivamente.
            \end{itemize}


        \subsubsection{Procedure}

            \subsubsubsection{Gestione delle comunicazioni}

            Per il coordinamento e le comunicazioni durante l'intera durata del progetto il gruppo \textit{SWEet16} adotterà le seguenti procedure:
            \begin{itemize}
                \item \textbf{Comunicazione interna:} Coinvolge tutti i membri del gruppo;
                \item \textbf{Comunicazione esterna:} Coinvolge proponente e/o committenti.
            \end{itemize}

            \subsubsubsubsection{Comunicazioni interne}

            Le comunicazioni interne avvengono tramite le applicazioni:
            \begin{itemize}
                \item \textbf{\emph{Telegram}$^{G}$:} Permette una rapida comunicazione tra i membri del gruppo e viene utilizzato per risolvere veloci dubbi o organizzare incontri interni;
                \item \textbf{\emph{Discord}$^{G}$:} Utilizzato per comunicazione tramite canale vocale.
            \end{itemize}

            \subsubsubsubsection{Comunicazioni esterne}

            Le comunicazioni esterne vengono gestite dal Responsabile di Progetto tramite l'utilizzo dei seguenti canali:
            \begin{itemize}
                \item \textbf{Posta elettronica:} Tramite l'indirizzo del gruppo \href{mailto:sweet16.unipd@gmail.com}{\nolinkurl{sweet16.unipd@gmail.com}};
                \item \textbf{\emph{Telegram}$^{G}$:} Utilizzato per avere un canale di comunicazione diretto con il proponente per organizzare incontri periodici.
            \end{itemize}

            \subsubsubsection{Gestione degli incontri}

            \subsubsubsubsection{Incontri interni}

            Negli incontri interni i partecipanti sono solo i membri che compongono il gruppo. \\
            Questi incontri vengono svolti almeno una volta a settimana, il Martedì sera o il Giovedì sera, in base alla disponibilità dei membri del gruppo. \\
            In caso servisse più di un incontro per quella settimana questi possono anche essere fatti nel fine settimana.

            Un incontro può essere richiesto da un qualsiasi membro del gruppo al Responsabile di Progetto che poi l'organizzerà in base alla disponibilità degli altri membri. \\
            Una volta fissata la data e l'ora queste vengono comunicate tramite il gruppo \emph{Telegram}$^{G}$.

            La modalità d'incontro è sempre virtuale e non fisica date le non poche difficoltà che ci sarebbero, sia in termini di distanza fisica che di orario.

            Come piattaforma per gli incontri viene utilizzato \emph{Discord}$^{G}$ che permette di avere un canale di comunicazione vocale affidabile e veloce.\\

            Ogni riunione sincrona viene così svolta:
            \begin{itemize}
                \item Precedentemente all'incontro il Responsabile stila un ordine del giorno sui principale punti da vedere durante la riunione;
                \item Il Responsabile decide chi scriverà il Verbale Interno per l'incontro, questa persona si occuperà di prendere appunti sugli argomenti trattati;
                \item Viene fatta una discussione sul lavoro effettuato da ogni membro del gruppo dall'ultimo incontro sincrono;
                \item Viene fatta una discussione sugli eventuali dubbi che sono emersi dall'ultimo incontro;
                \item Si effettua una pianificazione delle prossime attività da svolgere da cui al prossimo incontro;
                \item Se l'incontro viene effettuato il Giovedì sera, si scrive anche il Diario di Bordo per il giorno successivo.
            \end{itemize}

            Alla fine di ogni incontro la persona designata si occupa della stesura del Verbale Interno fornendo una breve descrizione sui punti trattati nell'incontro appena svolto.
            Quest'ultimo dovrà poi essere approvato dal Responsabile.


            \subsubsubsubsection{Incontri esterni}

            Negli incontri esterni i partecipanti sono i membri del gruppo ed il proponente o il committente.
            Questi incontri vengono richiesti dal Responsabile di Progetto se vi è la necessità di chiarimento di una problematica riscontrata da parte del gruppo. \\
            Il Responsabile provvede a formalizzare una richiesta all'entità esterna tramite i canali di comunicazione preposti (canale \emph{Telegram}$^{G}$ per il proponente ed email per il committente).\\

            In caso di richiesta accettata, verranno comunicate ai membri del gruppo, tramite canali di comunicazione interni, le informazioni riguardanti data, ora e le modalità dell'incontro.

            In occasione di ogni incontro esterno viene redatto un verbale da un redattore scelto dal Responsabile. \\
            Il contenuto della riunione deve essere riportato nel Verbale Esterno corrispondente ed approvato dal responsabile e dalla figura esterna (proponente o committente).

        \subsubsubsection{Rotazione dei ruoli}

        È prevista una rotazione dei ruoli a cadenza periodica da parte del gruppo \textit{SWEet16}.
        L'attribuzione dei ruoli viene svolta secondo i seguenti criteri:
        \begin{itemize}
            \item Equità;
            \item Assenza di conflitti;
            \item Continuità.
        \end{itemize}

        \subsubsubsection{Gestione dei compiti e delle task}

        Per la suddivisione dei compiti viene utilizzato GitHub come \emph{ITS}$^{G}$, in particolare questo è il ciclo di vita di una \emph{issue}$^{G}$:

        \begin{itemize}
            \item \textbf{Individuazione:} Viene individuata un'attività a seguito di un verbale interno o esterno, o da un commento su una pull request;
            \item \textbf{Creazione:} La \emph{issue}$^{G}$ viene creata fornendo una descrizione testuale e applicando le \emph{labels}$^{G}$ corrette;
            \item \textbf{Assegnazione:} Il Responsabile di Progetto assegna la \emph{issue}$^{G}$ ad uno o più componenti del gruppo;
            \item \textbf{Completamento:} L'attività viene completata da chi la sta svolgendo, vengono inoltre assegnati uno o più Verificatori per il punto successivo;
            \item \textbf{Verifica:} Un Verificatore effettua un controllo di qualità sulle azioni effettuate, può eventualmente lasciare dei commenti se non ritiene l'attività completata;
            \item \textbf{Accetazione:} A seguito della conferma da parte di un Verificatore il Responsabile effettua a sua volta un controllo e comunica se l'esito dell'attività è positivo o meno.
        \end{itemize}

        \subsection{Strumenti}

        Sezione contenente gli strumenti utilizzati per i processi organizzativi che hanno permesso di attuare in modo pratico ed efficace il lavoro, questi sono:

        \begin{itemize}
            \item \textbf{Github:} Piattaforma di condivisione di codice, utilizzata per il versionamento del software e per la gestione delle issue.\\
            L'\emph{ITS}$^{G}$ e il meccanismo di versionamento permettono di tener traccia delle modifiche apportate nel tempo in modo automatico e con una grande quantità di informazioni utili.
            \begin{center}
                \url{https://github.com}
            \end{center}
            \item \textbf{Discord:} Principale strumento utilizzato dal gruppo per le comunicazioni interne sincrone. Il gruppo ha adibito un server suddiviso in due canali:
                \begin{itemize}
                    \item \textbf{Canale Vocale:} Canale principale utilizzato durante gli incontri interni;
                    \item \textbf{Canale Testuale:} Canale utilizzato per la condivisione di risorse durante gli incontri interni.
                \end{itemize}
                \begin{center}
                    \url{https://discord.com}
                \end{center}
            \item \textbf{Telegram:} Strumento di comunicazione asincrona utilizzato da tutti i membri del gruppo per comunicazioni di varia natura.\\
                    Utilizzato inoltre per comunicazioni veloci con il \emph{proponente}$^{G}$;
                    \begin{center}
                        \url{https://telegram.org/}
                    \end{center}
            \item \textbf{Google Drive:} Strumento utilizzato come directory condivisa per stesura di bozze o veloci appunti. Vengono inoltre salvati al suo interno di Diari di Bordo:
                \begin{center}
                    \url{https://drive.google.com/}
                \end{center}
           \item \textbf{Microsoft Teams:} Piattaforma di videochiamata utilizzato per comunicazione esterne con il proponente o per formazione:
           \begin{center}
            \url{https://teams.microsoft.com/}
           \end{center}
        \end{itemize}

        \subsection{Formazione}

        Ogni membro del gruppo è responsabile della propria formazione circa l'utilizzo degli strumenti decisi per lo sviluppo.
