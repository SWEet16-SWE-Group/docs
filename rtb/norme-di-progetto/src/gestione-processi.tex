\pagebreak
\subsection{Gestione dei Processi}

        \subsubsection{Descrizione, scopo ed aspettative}

        Lo scopo del processo è la stesura del documento denominato Piano di Progetto, utilizzato dal gruppo per l'organizzazione e la gestione dei ruoli.\\
        In particolare le attività previste dal processo organizzativo di gestione dei processi sono:

        \begin{itemize}
            \item Assegnazioni dei ruoli e dei compiti;
            \item Comunicazione interne/esterne;
            \item Incontri interni/esterni;
            \item Strumenti di coordinamento;
            \item Strumenti di versionamento;
            \item Analisi dei Rischi e loro mitigazione.
        \end{itemize}

        Le aspettative sono invece:
        \begin{itemize}
            \item Ottenere un'organizzazione efficace tra i membri del gruppo;
            \item Adottare un buon \textit{Way of Working};
            \item Ottenere un'equa distribuzione dei ruoli, attraverso un'efficace rotazione degli stessi.
        \end{itemize}

        \subsubsection{Ruoli di progetto}

        Il Responsabile di Progetto si occupa di suddividere i ruoli tra i membri del gruppo, garantendo che ognuno di essi assuma nel corso del progetto almeno una volta ogni ruolo.\\
        Vengono di seguito descritti i singoli ruoli.

            \subsubsubsection{Responsabile di progetto}

            Il suo compito consiste nel garantire lo svolgimento delle attività pianificate entro i tempi e le modalità previste dal gruppo. \\ 
            Si occupa anche delle comunicazioni esterne con committenti e proponenti.

            Le sue responsabilità sono:
            \begin{itemize}
                \item Approvare la documentazione nella fase finale del processo di verifica;
                \item Gestire l'assegnazione dei compiti agli altri membri del gruppo;
                \item Organizzare il lavoro in modo da minimizzare la probabilità che si verifichino problemi.
            \end{itemize}

            \subsubsubsection{Amministratore di progetto}

            È la figura che si occupa della gestione degli strumenti necessari all'ambiente di lavoro, le sue responsabilità sono:
            \begin{itemize}
                \item Gestire il sistema di archiviazione e versionamento di documentazione e codice;
                \item Gestire il sistema di configurazione del prodotto;
                \item Mantenere un ambiente di sviluppo efficiente, fornendo strumenti adeguati ai componenti del gruppo;
                \item Salvaguardare la documentazione di progetto;
                \item Redigere ed attuare i piani e le procedure per la gestione della qualità.
            \end{itemize}

            \subsubsubsection{Analista}

            Il suo compito consiste nell'individuare, analizzare e documentare i servizi che il sistema deve fornire. \\
            Le sue responsabilità sono:
            \begin{itemize}
                \item Studiare il problema ed il relativo contesto applicativo;
                \item Comprendere il problema e definire la complessità ed i requisiti;
                \item Redigere il documento denominato Analisi dei Requisiti.
            \end{itemize}

            \subsubsubsection{Progettista}

            Il suo compito consiste nel definire la struttura architetturale del sistema.
            È responsabile delle attività di progettazione, le quali devono portare alla realizzazione di un prodotto che soddisfa i requisiti individuati dagli analisti.\\
            Le sue responsabilità sono:
            \begin{itemize}
                \item Studiare l'architettura più adatta al prodotto da realizzare;
                \item Garantire la qualità del prodotto;
                \item Identificare l'architettura ad alto livello del sistema e l'architetturaa livello di componenti del sistema.
            \end{itemize}

            \subsubsubsection{Programmatore}

            Il suo compito consiste nella scrittura del codice richiesto dallo svolgimento del progetto. \\
            Le sue responsabilità sono:
            \begin{itemize}
                \item Scrivere codice pulito e facile da mantenere, rispettando le Norme di Progetto;
                \item Realizzare gli strumenti per la verifica e la validazione del software.
            \end{itemize}

            \subsubsubsection{Verificatore}
            Il suo compito consiste nella verifica del lavoro svolto dagli altri componenti del progetto, sulla base delle proprie conoscenze tecniche, esperienza e conoscenza delle norme. \\
            Le sue responsabilità sono:
            \begin{itemize}
                \item Controllare che vengano rispettate le norme di progetto, inserite all'interno di questo documento;
                \item Verificare la conformità dei prodotti ai requisiti funzionali e di qualità;
                \item Segnalare eventuali anomalie riscontrate nella verifica in modo che vengano corrette tempestivamente.
            \end{itemize}


        \subsubsection{Procedure}

            \subsubsubsection{Gestione delle comunicazioni}
            
            Per il coordinamento e le comunicazioni durante l'intera durata del progetto il gruppo \textit{SWEet16} adotterà le seguenti procedure:
            \begin{itemize}
                \item \textbf{Comunicazione interna:} coinvolge tutti i membri del gruppo;
                \item \textbf{Comunicazione esterna:} coinvolge proponente e/o committenti.
            \end{itemize}

            \subsubsubsubsection{Comunicazioni interne}

            \subsubsubsubsection{Comunicazioni esterne}

            \subsubsubsection{Gestione degli incontri}

        \subsubsection{Rotazione dei ruoli}

            \subsubsubsection{Metodo organizzativo}