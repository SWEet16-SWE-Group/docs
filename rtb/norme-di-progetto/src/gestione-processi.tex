\pagebreak
\subsection{Gestione dei Processi}

        \subsubsection{Descrizione, scopo ed aspettative}

        Lo scopo del processo è la stesura del documento denominato Piano di Progetto, utilizzato dal gruppo per l'organizzazione e la gestione dei ruoli.\\
        In particolare le attività previste dal processo organizzativo di gestione dei processi sono:

        \begin{itemize}
            \item Assegnazioni dei ruoli e dei compiti;
            \item Comunicazione interne/esterne;
            \item Incontri interni/esterni;
            \item Strumenti di coordinamento;
            \item Strumenti di versionamento;
            \item Analisi dei Rischi e loro mitigazione.
        \end{itemize}

        Le aspettative sono invece:
        \begin{itemize}
            \item Ottenere un'organizzazione efficace tra i membri del gruppo;
            \item Adottare un buon \textit{Way of Working};
            \item Ottenere un'equa distribuzione dei ruoli, attraverso un'efficace rotazione degli stessi.
        \end{itemize}

        \subsubsection{Ruoli di progetto}

        Il Responsabile di Progetto si occupa di suddividere i ruoli tra i membri del gruppo, garantendo che ognuno di essi assuma nel corso del progetto almeno una volta ogni ruolo.\\
        Vengono di seguito descritti i singoli ruoli.

            \subsubsubsection{Responsabile di progetto}

            Il suo compito consiste nel garantire lo svolgimento delle attività pianificate entro i tempi e le modalità previste dal gruppo. \\ 
            Si occupa anche delle comunicazioni esterne con committenti e proponenti.

            Le sue responsabilità sono:
            \begin{itemize}
                \item Approvare la documentazione nella fase finale del processo di verifica;
                \item Gestire l'assegnazione dei compiti agli altri membri del gruppo;
                \item Organizzare il lavoro in modo da minimizzare la probabilità che si verifichino problemi.
            \end{itemize}

            \subsubsubsection{Amministratore di progetto}

            \subsubsubsection{Analista}

            \subsubsubsection{Progettista}

            \subsubsubsection{Programmatore}

            \subsubsubsection{Verificatore}


        \subsubsection{Procedure}

            \subsubsubsection{Gestione delle comunicazioni}

            \subsubsubsubsection{Comunicazioni interne}

            \subsubsubsubsection{Comunicazioni esterne}

            \subsubsubsection{Gestione degli incontri}

        \subsubsection{Rotazione dei ruoli}

            \subsubsubsection{Metodo organizzativo}