\subsection{Progettazione}

\subsubsection{Descrizione, scopo ed aspettative}
La fase di progettazione ha lo scopo di definire le linee essenziali della struttura del progetto in funzione dei requisiti individuati durante la fase di analisi e specificati nell’Analisi dei Requisiti.
Questa attività viene svolta dai progettisti che si occupano di specificare le funzionalità dei sottosistemi, per riuscire a implementare tutti i requisiti specificati.

L'obiettivo che il gruppo si pone durante questa fase del ciclo di vita del software sono:
\begin{itemize}
    \item Trasformazione di tutti i requisiti in specifiche dettagliate che vadano a coprire tutti gli aspetti del sistema;
    \item Creazione di una demo propotipale del sistema per testare le tecnologie e la loro compatibilita, detta Proof of Concept$^{G}$ (da qui in poi "PoC");
    \item Approvazione del passaggio alla fase di sviluppo;
\end{itemize}

\subsubsection{Requirements \& Technology Baseline (RTB)}

\subsubsection{Product Baseline (PB)}