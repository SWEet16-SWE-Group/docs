\subsection{Progettazione}

\subsubsection{Descrizione, scopo ed aspettative}
La fase di progettazione ha lo scopo di definire le linee essenziali della struttura del progetto in funzione dei requisiti individuati durante la fase di analisi e specificati nell’Analisi dei Requisiti. \\
Questa attività viene svolta dai progettisti che si occupano di specificare le funzionalità dei sottosistemi, per riuscire a implementare tutti i requisiti specificati all'interno di un unico sistema.

Gli obiettivi che il gruppo si pone durante questa fase del ciclo di vita del software sono:
\begin{itemize}
    \item Trasformazione di tutti i requisiti in specifiche dettagliate che vadano a coprire tutti gli aspetti del sistema;
    \item Creazione di una demo prototipale del sistema per testare le tecnologie e la loro compatibilità, detta \emph{PoC (Proof of Concept)}$^{G}$ (da qui in poi "PoC");
    \item Approvazione del passaggio alla fase di sviluppo.
\end{itemize}

Questo processo può essere suddiviso in tre fasi distinte:
\begin{enumerate}
    \item Design dell'interfaccia: In questa fase si procede ad un alto livello di astrazione rispetto al funzionamento interno del sistema. L'attenzione è focalizzata sulle tecnologie che verranno utilizzate nella fase di sviluppo software, l'obiettivo di questa fase è la codifica del sopracitato PoC;
    \item Progettazione architetturale: In questa fase si definisce la struttura generale del sistema, andando ad ignorare i dettagli interni dei componenti principali. Vengono inoltre definiti i test di integrazione;
    \item Progettazione dettagliata: In questa fase si vanno a specificare i dettagli di tutti i componenti del prodotto e le specifiche architetturali. Vengono definiti i diagrammi delle classi e i test di unità per ogni componente del prodotto.
\end{enumerate}
