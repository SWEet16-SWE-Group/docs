\nonstopmode
\subsection{Validazione}

\subsubsection{Descrizione, scopo ed aspettative}

La validazione del software è un processo successivo alla fase di verifica che permette di assicurarsi che il software soddisfi i requisiti del committente predefiniti e specificati,
e le richieste e aspettative del proponente. \\

In caso di buona verifica durante tutta la fase di sviluppo, allora il processo di validazione avrà esito positivo. \\
Questo garantisce che il prodotto finale sia in linea rispetto alle aspettative. \\

Le aspettative del gruppo nell'applicazione di questo processo sono:
\begin{itemize}
    \item Rilevare possibili errori ignorati o trascurati durante la fase di verifica;
    \item Assicurarsi che il prodotto finale soddisfi i requisiti specificati all'interno dell'Analisi dei Requisiti.
\end{itemize}

